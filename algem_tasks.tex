\documentclass[a4paper]{article}
\usepackage[utf8]{inputenc}
\usepackage[T2A]{fontenc}
\usepackage[12pt]{extsizes}

\usepackage[english,russian]{babel}
\usepackage[left=10mm, top=10mm, right=10mm, bottom=20mm, nohead, nofoot]{geometry}
\usepackage{amsmath,amsfonts,amssymb} % математический пакет
\headsep=10mm

\usepackage[most]{tcolorbox} % для управления цветом
% НАСТРОЙКИ
%теорема
\definecolor{theorem-color}{gray}{0.90} % уровень прозрачности (1 - максимум)
\newtcolorbox{htheorem}{colback=theorem-color,grow to right by=-4mm,grow to left by=-4mm,
    boxrule=0pt,boxsep=0pt,breakable} % настройки области с изменённым фоном

%определение
\definecolor{def-color}{gray}{0.98}
\newtcolorbox{definit}{colback=def-color,grow to right by=-4mm,grow to left by=-4mm,
    boxrule=0pt,boxsep=0pt,breakable} % настройки области с изменённым фоном

%доказательсвто теоремы
\definecolor{proof-color}{gray}{0.95} % уровень прозрачности (1 - максимум)
\newtcolorbox{hproof}{colback=proof-color,grow to right by=-1mm,grow to left by=-1mm,
    boxrule=0pt,boxsep=0pt,breakable} % настройки области с изменённым фоном

%замечания, следствия
\definecolor{consectary-color}{gray}{0.95} % уровень прозрачности (1 - максимум)
\newtcolorbox{cns}{colback=consectary-color,grow to right by=-4mm,grow to left by=-4mm,
    boxrule=0pt,boxsep=0pt,breakable} % настройки области с изменённым фоном



\usepackage{fancybox,fancyhdr}
\pagestyle{fancy}
\fancyhf{}
\fancyhead[R]{ФТ-104}
\fancyfoot[R]{\thepage}
\fancyhead[L]{алгем}

\usepackage{hyperref}
\hypersetup{colorlinks=true, allcolors=[RGB]{010 090 200}} % цвет ссылок 
\newcommand{\lr}[1]{\left({#1}\right)} % команда для скобок

\title{Конспектик к экзамену по алгему}
\author{Васильев Павел}
%\linespread{1}
\usepackage{amsmath}

\usepackage{graphicx}
\usepackage{ifpdf}
\ifpdf
\DeclareGraphicsRule{*}{mps}{*}{}
\fi
\usepackage{graphicx}
\usepackage{color}
\graphicspath{ {images/} }

%\renewcommand{\familydefault}{\sfdefault}

\begin{document}

\section*{Пару задач по алгему}

\begin{itemize}


\item Доказать формулу «бац минус цаб»: $a \times (b \times c) = b(a \cdot c) - c(a \cdot b)$.
\item Доказать тождество Якоби $(a \times b) \times c + (b \times c) \times a + (c \times a) \times b$ = 0.
\item Пусть $е, f,g$  - базис. Доказать, что вектора $e \times f, g \times e, f \times g$ также образуют базис. (Указание - эту задачу можно решать разными способами, но довольно поучительно перейти к координатам.)
\item Матрица Грама базиса $e_1,e_2,e_3$ - это 3х3-матрица, у которой на месте (i,j) стоит скалярное произведение $e_ie_j$. Доказать, что определитель матрицы Грама отличен от 0. (Указание - с теми средствами, которыми вы располагаете сейчас, решить эту задачу непросто. Но попробуйте! Когда вы узнаете больше про матрицы и определители, эта задача станет совсем простой.)
\item Алиса и Боб по очереди заполняют числами матрицу 2х2. Алиса (которая ходит первой) хочет добиться, чтобы определитель получившейся матрицы был отличен от 0, а Боб хочет добиться, чтобы этот определитель был равен 0. У кого из игроков есть выигрышная стратегия? А если Алиса хочет, чтобы получился нулевой определитель, а Боб - чтобы получился определитель, отличный от 0? Те же вопросы для матрицы 3х3. (Предостережение: для 3х3-матриц задача уже нетривиальна.)
\item Уравнения прямой и плоскости в трехмерном пространстве
\item Исследовать взаимное расположение трех прямых на плоскости. (Здесь и далее под словом "исследовать" понимается следующее: указать условия на коэффициенты уравнений, отвечающие различным с геометрической точки зрения вариантам взаимного расположения задаваемых этими уравнениями объектов).
\item На плоскости даны три параллельные прямые с уравнениями $Ax+By+C=0$, Ax+By+D=0, Ax+By+E=0. Указать необходимое и достаточное условие, при котором вторая прямая проходит между первой и третьей.
\item На плоскости даны две параллельные прямые с уравнениями $Ax+By+C=0, Ax+By+D=0$. Придумать формулу, выражающее расстояние между этими прямыми через коэффициенты $A,B,C,D$. (Система координат - прямоугольная декартова.)
\item На плоскости даны две пересекающиеся и неперпендикулярные прямые с уравнениями $A_1x+B_1y+C_1=0, A_2x+B_2y+C_2=0$. Написать уравнение биссектрисы острого угла между этими прямыми. (Система координат - прямоугольная декартова.)
\item Исследовать взаимное расположение трех плоскостей в пространстве.
\item В пространстве даны три параллельные плоскости с уравнениями $Ax+By+Cz+D=0, Ax+By+Cz+E=0, Ax+By+Cz+F=0$. Указать необходимое и достаточное условие, при котором вторая плоскость проходит между первой и третьей.
\item В пространстве даны две параллельные плоскости с уравнениями $Ax+By+Cz+D=0, Ax+By+Cz+E=0$ и прямая с уравнением $x=x_0+mt, y=y_0+nt, z=z_0+pt$. Указать необходимое и достаточное условие, при котором прямая расположена между плоскостями.
\item Найти сумму $k-х$ степеней всей корней n-й степени из 1.
\item Найти произведение корней n-й степени из 1.
\item Доказать, что модуль произведения двух комплексных чисел равен произведению их модулей.
\item Некоторые натуральные числа (например, 1, 2, 4 или 5) можно представить в виде суммы квадратов двух целых чисел, а некоторые (например, 3, 6 или 7) нельзя. Доказать, что если натуральные числа m и n представимы в виде суммы квадратов двух целых чисел, то и их произведение mn представимо в виде такой суммы.
\item Можно ли ввести на множестве комплексных чисел линейный порядок, продолжающий обычный порядок на множестве действительных чисел и согласованный с операцией сложения? (Согласованность означает, что для любых $x,y,z$, если $x < y$, то $x+z < y+z$.) А линейный порядок, согласованный с операцией умножения? (Здесь согласованность означает, что для любых $x,y,z$, если $x < y$ и $z > 0$, то $xz < yz$.)
\item Абстрактные векторные пространства
\item Доказать, что аксиому $1а=а$ нельзя вывести из остальных аксиом линейного пространства.
\item Доказать, что коммутативность сложения можно вывести из остальных аксиом линейного пространства. (Предостережение: задача нетривиальна.)
\item Доказать, что в обычном трехмерном пространстве любые четыре вектора линейно зависимы.
\item Доказать, что объединение двух подпространств будет подпространством тогда и только тогда, когда одно из этих подпространств содержится в другом.
\item Может ли объединение трех попарно несравнимых подпространств быть подпространством? (Указание: рассмотрите двумерное пространство над двухэлементным полем.)
\item Доказать, что для операций пересечения и суммы подпространств, вообще говоря, не выполняется дистрибутивный закон.
\item Доказать, что для операций пересечения и суммы подпространств выполняется так называемый модулярный закон: если подпространство А содержит подпространство В, то для любого подпространства С пересечение А с суммой В+С равно сумме В и пересечения А с С.
\item Формула для размерности суммы двух подпространств аналогична формуле включений и исключений для двух множеств. Верна ли формула для размерности суммы трех подпространств, построенная по аналогии с формулой включений и исключений для трех множеств? (Указание: рассмотрите три прямые в обычной двумерной плоскости.)
\item Докажите, что линейные многообразия$ x+M и y+N$ равны тогда и только тогда, когда $M = N и x-y$ лежит в М.
\item Ранг матрицы. Теория систем линейных уравнений
\item Доказать, что произвольная матрица ранга r представима в виде суммы r матриц ранга 1.
\item Доказать, что для любых двух матриц одинаковых размеров ранг их суммы не провосходит суммы их рангов.
\item Доказать, что для любой nxs-матрицы A, любой обратимой $n \times n$-матрицы В и любой обратимой sxs-матрицы ранги матриц А и ВАС равны.
\item Доказать, что для системы линейных уравнений следующие условия эквивалентны:
система имеет единственное решение;
ранг основной матрицы равен числу неизвестных;
ранг расширенной матрицы равен числу неизвестных.
\item (Неравенство Сильвестра) Пусть А - линейный оператор, принимающий значения в некотором n-мерном пространстве L, а В - линейный оператор, определенный на L. Доказать, что ранг оператора АВ не меньше $r(A)+r(B)-n$. (Указание: примените теорему о сумме ранга и дефекта к ограничению оператора В на пространство $Im(A)$.)
\item (Принцип наложения решений) Доказать, что если вектор y - решение системы линейных уравнений $Ax=b$, а вектор z - решение системы линейных уравнений $Ax=c$, то вектор $y+z$ будет решением системы линейных уравнений $Ax=b+c$.
\item Пусть квадратная матрица А такова, что система линейных уравнений $Ax=b$ имеет решение при любой правой части b. Доказать, что тогда эта система имеет единственное решение при каждой правой части. (Указание: воспользуйтесь теоремой о сумме ранга и дефекта.)
\item Евклидовы и унитарные пространства. Решение несовместных систем линейных уравнений
\item Верно ли утверждение, обратное к теореме Пифагора, в произвольном евклидовом или унитарном пространстве?
\item Что произойдет, если применить процесс Грама-Шмидта к линейно зависимой системе векторов?
\item Точка Лемуана треугольника - это точка, сумма квадратов расстояний которой до сторон треугольника минимальна. Найдите точку Лемуана треугольника, стороны которого лежат на прямых с уравнениями $A_1x+B_1y+C_1=0, A_2x+B_2y+С_2=0, A_3x+B_3y+С_3=0$. (Указание: примените метод наименьших квадратов.)
\item Две прямые в пространстве заданы уравнениями $x=x_0+mt, y=y_0+nt, z=z_0+pt и x=x_1+qt, y=y_1+rt, z=z_1+st$. Объединим эти 6 уравнений в одну систему. Какие точки будут псевдорешениями этой системы?
\item Пусть определитель nxn-матрицы A равен d. Чему равен определитель матрицы kА?
\item Пусть определитель $n \times n$-матрицы А равен d. Чему равен определитель матрицы, присоединенной к А?
\item Пусть ранг nxn-матрицы А равен r. Чему равен ранг матрицы, присоединенной к А?
\item Доказать, что при перестановке двух строк матрицы в присоединенной матрице происходит такая же перестановка столбцов и все элементы присоединенной матрицы меняют знак.
\item Доказать, что матрица, обратная к верхнетреугольной матрице, сама является верхнетреугольной.
\item (Теорема Гамильтона-Кэли) Пусть А - 2х2-матрица, s - ее след (сумма диагональных элементов), а d - ее определитель. Проверить, что $A^2-sA+dE=0$.
\item Через $tr(A)$ обозначается след матрицы А. Доказать, что удвоенный определитель 2х2-матрицы А равен $tr(A)^2-tr(A^2)$.
\item Привести пример 4х4-матрицы, определитель которой не равен ad-bc, где a - определитель верхнего левого 2х2-блока, b - определитель верхнего правого 2х2-блока, c - определитель нижнего левого 2х2-блока, d - определитель нижнего правого 2х2-блока.
\item Матрица А называется кососимметрической, если ее транспонированнаяматрица равна -А. Доказать, что определитель действительной кососимметрической матрицы нечетного порядка равен 0.
\item Пусть в nxn-матрице А есть такие s строк и t столбцов, что все элементы, стоящие на их пересечении, равны 0 и $s+t>n$. Доказать, что определитель матрицы А равен 0.
\end{itemize}
\end{document}
