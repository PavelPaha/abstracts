\documentclass[a4paper]{article}
\usepackage[utf8]{inputenc}
\usepackage[T2A]{fontenc}
\usepackage[12pt]{extsizes}
\usepackage[normalem]{ulem}
\usepackage{calligra}
\usepackage{pdfpages}
\usepackage{mathrsfs}
\usepackage{mathalfa}


\usepackage[english,russian]{babel}
\usepackage[left=10mm, top=10mm, right=10mm, bottom=20mm, nohead, nofoot]{geometry}
\usepackage{amsmath,amsfonts,amssymb} % математический пакет
\headsep=10mm

\usepackage[most]{tcolorbox} % для управления цветом
% НАСТРОЙКИ
%теорема
\definecolor{theorem-color}{gray}{0.90} % уровень прозрачности (1 - максимум)
\newtcolorbox{htheorem}{colback=theorem-color,grow to right by=-4mm,grow to left by=-4mm,
    boxrule=0pt,boxsep=0pt,breakable} % настройки области с изменённым фоном

%определение
\definecolor{def-color}{gray}{0.98}
\newtcolorbox{definit}{colback=def-color,grow to right by=-4mm,grow to left by=-4mm,
    boxrule=0pt,boxsep=0pt,breakable} % настройки области с изменённым фоном

%доказательсвто теоремы
\definecolor{proof-color}{gray}{0.95} % уровень прозрачности (1 - максимум)
\newtcolorbox{hproof}{colback=proof-color,grow to right by=-1mm,grow to left by=-1mm,
    boxrule=0pt,boxsep=0pt,breakable} % настройки области с изменённым фоном

%замечания, следствия
\definecolor{consectary-color}{gray}{0.95} % уровень прозрачности (1 - максимум)
\newtcolorbox{cns}{colback=consectary-color,grow to right by=-4mm,grow to left by=-4mm,
    boxrule=0pt,boxsep=0pt,breakable} % настройки области с изменённым фоном

\everymath{\displaystyle}


\usepackage{fancybox,fancyhdr}
\pagestyle{fancy}
\fancyhf{}
\fancyhead[R]{ФТ-104}
\fancyfoot[R]{\thepage}
\fancyhead[L]{экзамен алгем 2 семестр}

\usepackage{hyperref}
\hypersetup{colorlinks=true, allcolors=[RGB]{010 090 200}} % цвет ссылок 
\newcommand{\lr}[1]{\left({#1}\right)} % команда для скобок

\title{Конспектик к коллоквиуму по алгему}
\author{Васильев Павел}
%\linespread{1}
\usepackage{amsmath}

\usepackage{graphicx}
\usepackage{ifpdf}
\ifpdf
\DeclareGraphicsRule{*}{mps}{*}{}
\fi
\usepackage{graphicx}
\usepackage{color}
\graphicspath{ {images/} }

%\renewcommand{\familydefault}{\sfdefault}

\begin{document}

\section*{HOW TO заботать экзамен по алгебре (2 семестр)}

Должен признаться, что не планирую делать полный конспект к экзамену по линейной алгебре, так как это слишком долгий процесс.

\subsection*{Список билетов}
\begin{enumerate}
\item Многочлены
\begin{enumerate}
\item Основные понятия теории делимости. Отношение ассоциированности
\item Деление многочленов с остатком
\item Теорема о наибольшем общем делителе. Алгоритм Евклида
\item Существование и однозначность разложения на неприводимые многочлены в кольце многочленов над полем
\item Поле частных области. Рациональные дроби.
\item Кольцо многочленов над областью с однозначным разложением. Лемма Гаусса и ее следствия
\item Однозначность разложения на неприводимые многочлены в кольце многочленов над областью с однозначным разложением
\item Теорема Безу. Корни многочлена.
\item Классификация неприводимых многочленов над полями комплексных и действительных чисел
\item Неприводимые многочлены с целыми коэффициентами. Критерий Эйзенштейна. Алгоритм Кронекера
\item Неприводимые многочлены над полями вычетов
\item Отделение кратных множителей
\item Кратные корни. Число корней многочлена n-й степени
\item Поле разложения многочлена. Конечные поля
\item Симметрические многочлены. Формулы Виета. Основная теорема о симметрических многочленах
\item Лемма о модуле старшего члена. Основная теорема алгебры комплексных чисел.
\end{enumerate}

\item Линейные операторы
\begin{enumerate}
\item Изменение матрицы при замене базиса
\item Собственные числа и собственные значения линейного оператора. Линейные операторы простой структуры
\item Линейные функционалы. Теорема о строении линейного функционала на унитарном (евклидовом) пространстве.
\item Сопряженный оператор. Линейность сопряженного оператора. Свойства операции сопряжения. Матрица сопряженного оператора
\item Теорема Фредгольма. Альтернатива Фредгольма.
\item Нормальный оператор. Теорема о строении нормального оператора.
\item Унитарные и ортогональные операторы.
\item Самосопряженные операторы.
\item Неотрицательные самосопряженные операторы. Квадратные корни из неотрицательных самосопряженных операторов.
\item Полярное разложение оператора на унитарном (евклидовом) пространстве
\item Сингулярные числа и их применения. Теорема Эккарта-Янга
\item Псевдообратный оператор. Нормальное псевдорешение несовместной системы линейных уравнений.
\end{enumerate}

\item Жорданова теория
\begin{enumerate}
\item Разложение Фиттинга. Корневое разложение. Теорема о корневом разложении.
\item Теорема о минимальном многочлене. Теорема Гамильтона-Кэли
\item Жорданов базис нильпотентного оператора
\item Теорема Жордана
\end{enumerate}

\item Квадратичные формы
\begin{enumerate}
\item Метод Лагранжа
\item Закон инерции действительных квадратичных форм
\item Критерий Сильвестра
\end{enumerate}

\item Квадрики на плоскости и в пространстве

\begin{enumerate}
\item Эллипс, гипербола, парабола
\item Упрощение уравнения 2-го порядка от двух переменных. Классификация плоских квадрик
\item Эллипсоиды, гиперболоиды, параболоиды, конусы, цилиндры
\item Упрощение уравнения 2-го порядка от трех переменных. Классификация пространственных квадрик
\end{enumerate}

\end{enumerate}


\section*{Линейные операторы}

\subsection*{Изменение матрицы при замене базиса}

Пусть $V$ - конечномерное векторное пространство над полем $F$, а $P = \{ p_1, p_2, ..., p_n \}$ и $Q = \{ q_1, q_2, ..., q_n \}$ - два базиса этого пространства. \textbf{Матрицей перехода от базиса $P$ к базису $Q$} нызывается $n \times n$ матрицы, $i$-ый столбец которой (для каждого $i=1,...,n$) есть координатный столбец вектора $q_i$ в базисе $P$.

Обозначается как $T_{P \rightarrow Q}$.

\begin{htheorem}\textbf{Предложение}.

Пусть $P$ и $Q$ - два базиса пространства $V$.
Тогда для любого $x \in V$

\[
[x]_P = T_{P \rightarrow Q} [x]_Q
\]

\[
[x]_P = T_{P \rightarrow Q} T_{Q \rightarrow P} [x]_P
\]
\end{htheorem}

\begin{htheorem}\textbf{Предложение}.

Пусть $P$ и $Q$ - два базиса пространства $V$. Матрица $T_{P \rightarrow Q}$ обратима и обратной к ней является матрицаа обратного перехода $T_{Q \rightarrow P}$.
\end{htheorem}

\begin{htheorem}\textbf{Теорема (о замене матрицы)}.

Пусть $V$ и $W$ - конечномерные векторные пространства над полем $F$, $P_1, P_2$ - базисы пространства $V, Q_1, Q_2$ - базисы пространства $W$, а $\mathcal{A}: V \rightarrow W$ - линейный оператор. Тогда 

\[
A_{P_2, Q_2} = T_{Q_2 \rightarrow Q_1} A_{P_1,Q_1} T_{P_1 \rightarrow P_2}
\]

Важный частный случай $W=V$. Тогда $Q_1 = P_1, Q_2=P_2$.
\end{htheorem}

\textbf{Определение.} Квадратные матрицы $A$ и $B$ над некоторым полем $F$ называются подобными над $F$, если существует невырожденная квадратная матрица над $F$ такая, что $B = T^{-1}AT$.

Таким образом, все матрицы одного и тогоже линейного оператора $\mathcal{A}: V \rightarrow V$ подобны между собой.

\subsection*{Собственные числа и собственные значения линейного оператора. Линейные операторы простой структуры}

Пусть $V$ - векторное пространство над полем $F$, а $\mathcal{A}$ - линейный оператор на $V$. Вектор $x \in V$ нызвается \textbf{собственным вектором} оператора $\mathcal{A}$, если $x \neq 0$ и существует скаляр $\lambda \in F$ такой, что 

\[
\mathcal{A}x = \lambda x
\]

\begin{htheorem}\textbf{Замечание}.
Характеристические многочлены подобных матриц равны.

Справка: 

Квадратные матрицы $A$ и $B$ одинакового порядка называются подобными, если существует невырожденная матрица $P$ того же порядка, такая что $B = P^{-1} AP$

Характеристический многочлен матрицы — многочлен, определяющий её собственные значения.
\end{htheorem}

\subsubsection*{Замечания}
У линейного оператора на $n$-мерном пространстве не более $n$ собственных значений (так как у многочлена $n$ степени не более $n$ корней).

У любого линейного оператора обычного трехмерного пространства
есть собственный вектор. Геометрически это отнюдь не очевидно, но сразу
следует из наличия действительного корня у многочленов третьей степени.

В силу основной теоремы алгебры комплексных чисел у любого
оператора на любом конечномерном пространстве над полем C есть
собственные значения и собственные вектора.

\textbf{Алгоритм поиска собственных значений и собственных векторов оператора $\mathcal{A}$}:
\begin{enumerate}
\item Взять матрицу $A$ оператора $\mathcal{A}$ в некотором базисе
\item Вычислить характеристический многочлен $\det(A - \lambda E)$
\item Найти корни характеристического многочлена $\lambda_1, ..., \lambda_k$.
\item Для каждого $\lambda_i$ найти ненулевые решения системы линейных однородных уравнений $(A-\lambda_i E) x = 0$
\end{enumerate}

\begin{htheorem}\textbf{Теорема}.

Собственные вектора, принадлежащие попарно различным собственным
значениям, линейно независимы.
\end{htheorem}

\begin{htheorem}\textbf{Следствие}.

Если у линеуного опретора $\mathcal{A}$ на $n$-мерном пространстве имеется $n$ различных собственных значений, то в $V$ существует базис из собственных векторов оператора $\mathcal{A}$.
\end{htheorem}

\textbf{Определение.} Оператор с $n$ различными собственными значениями нызваеют \textbf{операторами простой структуры}.

В базисе из собственных векторов оператора его матрица диагональна,
причем по диагонали идут собственные значения, которым принадлежат
вектора базиса. Поэтому операторы, допускающие такой базис, называют
приводимыми к диагональному виду или \textbf{диагонализируемыми}.

Из отмеченного выше следствия вытекает, что операторы простой
структуры диагонализируемы.
Обратное, разумеется, неверно: например, тождественный оператор и
нулевой оператор диагонализируемы, так как у каждого из них любой
ненулевой вектор собственный.
Бывают ли недиагонализируемые операторы? Конечно, некоторые
операторы недиагонализируемы из-за нехватки собственных значений.
Например, оператор поворота плоскости на угол $\frac{\pi}{2}$
недиагонализируем.
Но бывают и недиагонализируемые операторы, у которых есть
собственные значения.


\subsection*{Сопряжённые операторы}

Оператора $\mathcal{A}^*$ называется сопряжённым с $\mathcal{A}$, если $\forall x, y \in V$

\[
(\mathcal{A}(x),y) = (x, \mathcal{A}^*(y)
\]

\textit{Свойства}:
\begin{enumerate}
\item $\mathcal{A}^*$ --- линейный оператор
\item $(\mathcal{A} + \mathcal{B})^* = \mathcal{A}^* + \mathcal{B}^*$
\item $(\alpha \mathcal{A})^* = \overline{\alpha}\mathcal{A}^*$
\item $(\mathcal{A} \mathcal{B})^* = \mathcal{B}^* \mathcal{A}^*$
\item $(\mathcal{A}^*)^* = \mathcal{A}$
\end{enumerate}

\begin{htheorem}\textbf{Предложение (матрица сопряжённого оператора)}.

Если линейный оператор $\mathcal{A}: V_1 \rightarrow V_2$ имеет в ортонормированных базисах пространств $V_1$ и $V_2$ матрицу $A$, то сопряжённый ему оператор $\mathcal{A}^*: V_2 \rightarrow V_1$ имеет в тех же базисах матрицу $A^*$

\end{htheorem}


Да, это верно. Ортогональная матрица - это квадратная матрица $Q$, для которой выполняется условие $Q^{T}Q=QQ^{T}=I$, где $^{T}$ обозначает транспонирование матрицы и $I$ - единичная матрица.

Пусть $\vec{v}$ - некоторый вектор, такой что $Q\vec{v} = \vec{0}$, где $\vec{0}$ - нулевой вектор. Тогда мы можем умножить обе стороны на $Q^{T}$: $Q^{T}Q\vec{v} = Q^{T}\vec{0}$, то есть $I\vec{v} = \vec{0}$, так как $Q^{T}Q=I$. Следовательно, $\vec{v} = \vec{0}$, что означает, что ядро ортогональной матрицы состоит только из нулевого вектора. Таким образом, ортогональная матрица инъективна (взаимно-однозначное соответствие) и является линейным оператором на всем пространстве.

Для доказательства данного утверждения необходимо воспользоваться определением изометрического оператора и сингулярным разложением матрицы оператора.

Определение изометрического оператора. Линейный оператор $A$ называется изометрическим, если выполняется условие $\Vert Ax \Vert = \Vert x \Vert$ для всех $x \in V$, где $\Vert x \Vert$ - норма вектора $x$.

Сингулярное разложение. Любую матрицу $A$ размера $m \times n$ можно представить в виде произведения трех матриц: $A = U \Sigma V^T$, где $U$ и $V$ - ортогональные матрицы размера $m \times m$ и $n \times n$ соответственно, а $\Sigma$ - диагональная матрица размера $n \times m$, элементы главной диагонали которой являются сингулярными числами матрицы $A$.

Доказательство:

Пусть $A$ - изометрический оператор и $A = U \Sigma V^T$ - его сингулярное разложение. Мы можем проверить, что все сингулярные числа матрицы $A$ равны 1, используя определение изометрического оператора и свойства ортогональных матриц.

Для любого вектора $x \in V$ мы можем записать $x$ в виде линейной комбинации столбцов матрицы $V$: $x = \sum_{i=1}^n v_i u_i$, где $u_i$ - столбцы матрицы $U$, а $v_i$ - элементы вектора $V^T x$. Тогда:

$\Vert A x \Vert^2 = \Vert U \Sigma V^T x \Vert^2 = \Vert \Sigma V^T x \Vert^2 = \sum_{i=1}^n \sigma_i^2 (V^T x)_i^2 = \sum_{i=1}^n (V^T x)_i^2 = \Vert x \Vert^2$

 
где $\sigma_i$ - $i$-ое сингулярное число матрицы $A$.

Таким образом, мы получаем, что для любого вектора $x \in V$ выполняется условие $\Vert A x \Vert^2 = \Vert x \Vert^2$, откуда следует, что $\Vert A x \Vert = \Vert x \Vert$ (в силу неотрицательности нормы вектора), что означает, что линейный оператор $A$ является изометрическим.

С учетом этого мы можем заключить, что все сингулярные числа матрицы $A$ равны единице, так как:

$\Vert A x \Vert = \Vert x \Vert$ для любого $x \in V$.
Рассмотрим сингулярное разложение $A=U\Sigma V^T$. Тогда $\sigma_i^2$ равны квадратам собственных значений $A^TA$. Так как $\Vert A x \Vert = \Vert x \Vert$, собственные значения $A^TA$ равны 1 (или можно записать, что $\Vert A^TAx \Vert = \Vert x \Vert$ для любого $x \in V$). Следовательно, $\sigma_i^2 = 1$ для всех $i$.
Таким образом, доказано, что если линейный оператор является изометрическим, то все сингулярные числа его матрицы равны 1.




чтобы показать, что $\sigma_1^2(v_1^Tx)^2 \leq |x|^2/n$, мы воспользовались тем фактом, что $|Ax| = |\Sigma V^Tx|$. Для этого мы заметили, что слагаемые в выражении $|\Sigma V^Tx|$ (которые представляют собой произведения сингулярных чисел на координаты вектора $V^Tx$) являются неотрицательными. Следовательно, наибольшим значением выражения $\sigma_1^2(v_1^Tx)^2$ может быть само значение $|\Sigma V^Tx|^2$, которое является равным $|x|^2$.
\end{document}
