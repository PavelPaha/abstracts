\documentclass[a4paper]{article}
\usepackage[utf8]{inputenc}
\usepackage[T2A]{fontenc}
\usepackage[12pt]{extsizes}
\usepackage[normalem]{ulem}
\usepackage{calligra}
\usepackage{pdfpages}
\usepackage{mathrsfs}
\usepackage{mathalfa}


\usepackage[english,russian]{babel}
\usepackage[left=10mm, top=10mm, right=10mm, bottom=20mm, nohead, nofoot]{geometry}
\usepackage{amsmath,amsfonts,amssymb} % математический пакет
\headsep=10mm

\usepackage[most]{tcolorbox} % для управления цветом
% НАСТРОЙКИ
%теорема
\definecolor{theorem-color}{gray}{0.90} % уровень прозрачности (1 - максимум)
\newtcolorbox{htheorem}{colback=theorem-color,grow to right by=-4mm,grow to left by=-4mm,
    boxrule=0pt,boxsep=0pt,breakable} % настройки области с изменённым фоном

%определение
\definecolor{def-color}{gray}{0.98}
\newtcolorbox{definit}{colback=def-color,grow to right by=-4mm,grow to left by=-4mm,
    boxrule=0pt,boxsep=0pt,breakable} % настройки области с изменённым фоном

%доказательсвто теоремы
\definecolor{proof-color}{gray}{0.95} % уровень прозрачности (1 - максимум)
\newtcolorbox{hproof}{colback=proof-color,grow to right by=-1mm,grow to left by=-1mm,
    boxrule=0pt,boxsep=0pt,breakable} % настройки области с изменённым фоном

%замечания, следствия
\definecolor{consectary-color}{gray}{0.95} % уровень прозрачности (1 - максимум)
\newtcolorbox{cns}{colback=consectary-color,grow to right by=-4mm,grow to left by=-4mm,
    boxrule=0pt,boxsep=0pt,breakable} % настройки области с изменённым фоном

\everymath{\displaystyle}


\usepackage{fancybox,fancyhdr}
\pagestyle{fancy}
\fancyhf{}
\fancyhead[R]{ФТ-104}
\fancyfoot[R]{\thepage}
\fancyhead[L]{экзамен алгем 2 семестр}

\usepackage{hyperref}
\hypersetup{colorlinks=true, allcolors=[RGB]{010 090 200}} % цвет ссылок 
\newcommand{\lr}[1]{\left({#1}\right)} % команда для скобок

\title{Конспектик к экзамену по алгему}
\author{Васильев Павел}
%\linespread{1}
\usepackage{amsmath}

\usepackage{graphicx}
\usepackage{ifpdf}
\ifpdf
\DeclareGraphicsRule{*}{mps}{*}{}
\fi
\usepackage{graphicx}
\usepackage{color}
\graphicspath{ {images/} }

%\renewcommand{\familydefault}{\sfdefault}

\begin{document}

\section*{HOW TO заботать экзамен по алгебре (2 семестр)}

\subsection*{Вопросы}

\begin{enumerate}
\item Перестановки, подстановки, четность, нечетность. Свойства. 
\item Определение определителя квадратной матрицы. Простейшие свойства определителя. 
\item Свойства определителя квадратной матрицы. Разложение определителя квадратной матрицы по произвольной строке и произвольному столбцу.
\item Полураспавшиеся и распавшиеся матрицы. Определитель полураспавшейся и квазидиагональной матриц.
\item Определитель произведения квадратных матриц.
\item Обратное отображение и обратимые матрицы. Матрица, обратная к данной. Критерий обратимости в терминах её определителя.
\item Крамеровы системы линейных уравнений. Формулы Крамера.
\item Построение кольца многочленов. Простейшие свойства многочленов.
\item Деление многочленов с остатком.
\item Делимость многочленов. Свойства отношения делимости. Отношение ассоциированности.
\item Наибольший общий делитель многочленов. Теорема существования. Ассоциированность НОД. 
\item Выражение НОД через исходные многочлены.
\item Взаимно простые многочлены и их свойства.
\item Неприводимые многочлены и их свойства. Теорема о разложении в произведение неприводимых многочленов. Каноническое разложение.
\item Производная многочлена и ее свойства. Кратные множители многочлена. Алгоритм выделения кратных множителей.
\item Значение многочлена. Корни многочлена. Теорема Безу. Равенство многочленов, совпадающих как функции.
\item Интерполяционный многочлен Лагранжа.
\item Разложение многочленов над полем действительных чисел.
\item Многочлены над полем рациональных чисел и кольцом целых чисел. Примитивные многочлены и их свойства.
\item Неприводимые многочлены над полем рациональных чисел и кольцом целых чисел. Критерий Эйзенштейна.
\item Сопряженное отображение. Свойства сопряженного отображения. Единственность сопряженного отображения. 
\item Дважды сопряженное отображение. Существование сопряженного отображения.
\item Изометрические отображения и их свойства.
\item Собственные числа и собственные векторы линейного преобразования. Характеристический многочлен линейного преобразования. Условия существование собственных векторов линейного преобразования.
\item Самосопряженные линейные преобразования и их свойства. Строение матрицы самосопряженного линейного преобразования.
\item Сингулярное представление линейного отображения.
\item Псевдообратный оператор.
\item Билинейные и квадратичные функции. Билинейные и квадратичные формы. Матрица билинейной формы. Конгруэнтные формы и матрицы. 
\item Квадратичные функции и формы. Связь с симметричными билинейными функциями и формами. Конгруэнтность квадратичных функций и форм.
\item Канонический и нормальный виды квадратичной формы. Приведение формы к каноническому виду. Единственность нормального вида над полем комплексных чисел. 
\item Закон инерции вещественных квадратичных форм. 
\item Знакоопределенные квадратичные формы. Критерий Сильвестра. 
\item Приведение вещественной квадратичной формы к главным осям. 
\item Эллипс.
\item Гипербола.
\item Директориальное свойство эллипса.
\item Директориальное свойство гиперболы.
\item Парабола.
\item Приведение кривых второго порядка к каноническому виду.
\item Цилиндрические поверхности
\item Поверхности вращения.
\item Эллипсоид.
\item Однополостный гиперболоид. Асимптотический конус.
\item Двуполостный гиперболоид. Асимптотический конус.
\item Эллиптический параболоид.
\item Гиперболический параболоид.
\item Приведение поверхностей второго порядка к каноническому виду.
\end{enumerate}


\subsection*{Перестановки, подстановки, четность, нечетность. Свойства.}

 Перестановка - любая последовательность длины $n$ $(i_1, ..., i_n)$, в которой каждое число от 1 до $n$ ввходит 1 раз.

 Число инверсий - количество пар $(i,j)$, $i < j$ и номер $j$ меньше $i$.

 Перестановка \textit{чётная}, если в ней чётное число инверсий и нечётная иначе.

\begin{htheorem}\textbf{Теорема}.

Пусть $g$ - перестановка. При перестановке любой пары элементов чётность перестановки поменяется.
\end{htheorem}


 \textit{Подстановка} на $\{ 1, ..., n\}$ - биекция на $\{ 1, ..., n\}$

Запись: 

\[ g = \begin{pmatrix}
1 & 2 & ... & n \\ i_1 & i_2 & ... & i_n
\end{pmatrix}
\]

Подстановка состоит из двух перестановок.

Подстановка чётная, если число инверсий в ней чётное.

\begin{htheorem}\textbf{Теорема}.


\end{htheorem}
\begin{enumerate}
\item Любая подстановка может быть представлена в каноническом виде
\item Чётность не зависит от упорядочения верхнего ряда
\end{enumerate}

\begin{htheorem}\textbf{Предложение}.

Обратная подстановка $g^{-1} = \begin{pmatrix}
i_1 & i_2 & ... & i_n \\ 1 & 2 & ... & n
\end{pmatrix}$ имеет такую же чётность как исходная.

\end{htheorem}

\subsection*{Определение определителя квадратной матрицы. Простейшие свойства определителя.}
$S_n$ - множество подстановок на $\{ 1, ..., n\}$.
\[ |S_n| = n! \]

\textit{Определителем} матрицы $A$ называется число 
\[ |A| = \det A = \sum_{g \in S_n} (-1)^g a_{1 g(1)} a_{2 g(2)} ... a_{n g(n)} \]


\begin{htheorem}\textbf{Теорема}.

\[ |A| = |A^T| \]
\end{htheorem}

\begin{htheorem}\textbf{Теорема}.

Все свойства определителя, справедливые для строк, также справедливы и для столбцов.


\subsubsection*{Свойства определителя}
\begin{itemize}
\item $|A| = |A^T|$
\item $\begin{vmatrix}
a_{11} & ... & a_{1n} \\ ... & ... & ... \\ ta_{k1} & ... & ta_{kn} \\ ... & ... & ... \\ a_{n1} & ... & a_{nn}
\end{vmatrix} = t \begin{vmatrix}
a_{11} & ... & a_{1n} \\ ... & ... & ... \\ a_{k1} & ... & a_{kn} \\ ... & ... & ... \\ a_{n1} & ... & a_{nn}
\end{vmatrix}$
\end{itemize}
\end{htheorem}


\end{document}
