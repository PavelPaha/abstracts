\documentclass[a4paper]{article}
\usepackage[utf8]{inputenc}
\usepackage[T2A]{fontenc}
\usepackage[12pt]{extsizes}
\usepackage[normalem]{ulem}
\usepackage{calligra}
\usepackage{pdfpages}
\usepackage{mathrsfs}
\usepackage{mathalfa}


\usepackage[english,russian]{babel}
\usepackage[left=10mm, top=10mm, right=10mm, bottom=20mm, nohead, nofoot]{geometry}
\usepackage{amsmath,amsfonts,amssymb} % математический пакет
\headsep=10mm

\usepackage[most]{tcolorbox} % для управления цветом
% НАСТРОЙКИ
%теорема
\definecolor{theorem-color}{gray}{0.90} % уровень прозрачности (1 - максимум)
\newtcolorbox{htheorem}{colback=theorem-color,grow to right by=-4mm,grow to left by=-4mm,
    boxrule=0pt,boxsep=0pt,breakable} % настройки области с изменённым фоном

%определение
\definecolor{def-color}{gray}{0.98}
\newtcolorbox{definit}{colback=def-color,grow to right by=-4mm,grow to left by=-4mm,
    boxrule=0pt,boxsep=0pt,breakable} % настройки области с изменённым фоном

%доказательсвто теоремы
\definecolor{proof-color}{gray}{0.95} % уровень прозрачности (1 - максимум)
\newtcolorbox{hproof}{colback=proof-color,grow to right by=-1mm,grow to left by=-1mm,
    boxrule=0pt,boxsep=0pt,breakable} % настройки области с изменённым фоном

%замечания, следствия
\definecolor{consectary-color}{gray}{0.95} % уровень прозрачности (1 - максимум)
\newtcolorbox{cns}{colback=consectary-color,grow to right by=-4mm,grow to left by=-4mm,
    boxrule=0pt,boxsep=0pt,breakable} % настройки области с изменённым фоном

\everymath{\displaystyle}


\usepackage{fancybox,fancyhdr}
\pagestyle{fancy}
\fancyhf{}
\fancyhead[R]{ФТ-104}
\fancyfoot[R]{\thepage}
\fancyhead[L]{экзамен алгем 2 семестр}

\usepackage{hyperref}
\hypersetup{colorlinks=true, allcolors=[RGB]{010 090 200}} % цвет ссылок 
\newcommand{\lr}[1]{\left({#1}\right)} % команда для скобок

\title{Конспектик к экзамену по алгему}
\author{Васильев Павел}
%\linespread{1}
\usepackage{amsmath}

\usepackage{graphicx}
\usepackage{ifpdf}
\ifpdf
\DeclareGraphicsRule{*}{mps}{*}{}
\fi
\usepackage{graphicx}
\usepackage{color}
\graphicspath{ {images/} }

%\renewcommand{\familydefault}{\sfdefault}

\begin{document}

\section*{HOW TO заботать экзамен по алгебре (2 семестр)}

\subsection*{Вопросы}

\begin{enumerate}
\item Минорный ранг матрицы.
\item Крамеровы системы линейных уравнений. Формулы Крамера.
\end{enumerate}


\subsection*{Минорный ранг матрицы}

\textbf{Минором порядка $m$} матрицы А называется определитель квардратной подматрицы порядка $m$ матрицы $A$.

\textbf{Рангом} матрицы A \textbf{по минорам} называется число 0, если $A$ - нулевая матрицы и навиысший порядок отличных от нуля миноров матрицы A, если 
A - ненулевая матрица.

\subsection*{Крамеровы системы линейных уравнений. Формулы Крамера.}

\begin{htheorem}\textbf{Теорема}.\textbf{Правило Крамера}.

Пусть матричное уравнение 

\[ Ax = B \]

описывает систему из $n$ линейных уравнений с $n$ неизвестными.

Если $|A| \neq 0$, то система совместная и имеет единственное решение, описываемое формулой 

\[ x_i = \frac{D_i}{D} \]

где $D = |A|, D_i$ - определитель, полученный из определителя $D$ заменой $i$-ого столбца столбцом свободных членов матрицы $B$:

\[
D_i = \begin{pmatrix}
a_{11} & ... & a_{1, i-1} & b_1 & a_{1, i+1} & ... & a_{1n} \\
... & ... & ... & ... & ... & ... & ... \\
a_{n1} & ... & a_{n, i-1} & b_n & a_{n, i+1} & ... & a_{nn} \\
\end{pmatrix}
\]
\end{htheorem}

\begin{hproof}\textbf{Доказательство.}

Так как $|A| \neq 0$, то существует и притом единственная $A^{-1}$. Умножим обе части равенства $Ax = B$ на $A^{-1}$ слева.

\[ X = A^{-1}B \]

Так как обратная матрица единственна, то $X$ единственный.

Покажем, что из $Ax = B$ следует $x_i = \frac{D_i}{D}$.

\[ A^{-1} = \frac{1}{|A|} adj A = \frac{1}{|A|} |A^*|^T \]

\[ x_i = (A^{-1}B)_i = \frac{1}{D} (A_{1i}, A_{2i}, ..., A_{ni}) \cdot \begin{pmatrix}
b_1 \\ b_2 \\ ... \\ b_n
\end{pmatrix} = \frac{1}{D} \sum_{k=1}^n A_{ki} b_k \]

Заметим, что $\sum_{k=1}^n A_{ki} b_k$ - это и есть определитель той матрицы, у которой мы заменили $i$-ый столбец столбцом свободных членов матрицы $B$, то есть $\sum_{k=1}^n A_{ki} b_k = D_i$

\[ x_i = \frac{D_i}{D} \]

Теперь покажем что из того, что $x_i = \frac{1}{D} \sum_{k=1}^n A_{ki} b_k$ (*) следует, что $Ax=B$.

Домножим равенство (*) на $Da_{ji}$

\[ D a_{ji} x_i = \sum_{k=1}^n A_{ki} a_{ji} b_k \]

И просуммируем по $i$:

\[ D \sum_{i=1}^n a_{ji} x_i = \sum_{i=1}^n  \sum_{k=1}^n A_{ki} a_{ji} b_k  = \sum_{i=1}^n A_{ki} a_{ji}  \sum_{k=1}^n b_k \] (че за манёвр в последнем равенстве?)

\[ \sum_{i=1}^n A_{ki} a_{ji} = \delta_{kj} |A| = \delta_{kj} D \]

\[ \delta_{kj} = (int)(k==j) \]

\[D \sum_{i=1}^n a_{ji} x_i = D \sum_{k=1}^n b_k \delta_{kj} = Db_j \Rightarrow \sum_{i=1}^n a_{ji}x_i = b_j \Rightarrow Ax=B \]

\href{https://portal.tpu.ru/SHARED/k/KONVAL/Sites/Russian_sites/2/18.htm}{Источник}
\end{hproof}

\end{document}
