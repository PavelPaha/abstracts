\documentclass[a4paper]{article}
\usepackage[utf8]{inputenc}
\usepackage[T2A]{fontenc}
\usepackage[12pt]{extsizes}
\usepackage[normalem]{ulem}
\usepackage{calligra}
\usepackage{pdfpages}
\usepackage{mathrsfs}
\usepackage{mathalfa}


\usepackage[english,russian]{babel}
\usepackage[left=10mm, top=10mm, right=10mm, bottom=20mm, nohead, nofoot]{geometry}
\usepackage{amsmath,amsfonts,amssymb} % математический пакет
\headsep=10mm

\usepackage[most]{tcolorbox} % для управления цветом
% НАСТРОЙКИ
%теорема
\definecolor{theorem-color}{gray}{0.90} % уровень прозрачности (1 - максимум)
\newtcolorbox{htheorem}{colback=theorem-color,grow to right by=-4mm,grow to left by=-4mm,
    boxrule=0pt,boxsep=0pt,breakable} % настройки области с изменённым фоном

%определение
\definecolor{def-color}{gray}{0.98}
\newtcolorbox{definit}{colback=def-color,grow to right by=-4mm,grow to left by=-4mm,
    boxrule=0pt,boxsep=0pt,breakable} % настройки области с изменённым фоном

%доказательсвто теоремы
\definecolor{proof-color}{gray}{0.95} % уровень прозрачности (1 - максимум)
\newtcolorbox{hproof}{colback=proof-color,grow to right by=-1mm,grow to left by=-1mm,
    boxrule=0pt,boxsep=0pt,breakable} % настройки области с изменённым фоном

%замечания, следствия
\definecolor{consectary-color}{gray}{0.95} % уровень прозрачности (1 - максимум)
\newtcolorbox{cns}{colback=consectary-color,grow to right by=-4mm,grow to left by=-4mm,
    boxrule=0pt,boxsep=0pt,breakable} % настройки области с изменённым фоном

\everymath{\displaystyle}


\usepackage{fancybox,fancyhdr}
\pagestyle{fancy}
\fancyhf{}
\fancyhead[R]{ФТ-104}
\fancyfoot[R]{\thepage}
\fancyhead[L]{экзамен алгем 2 семестр}

\usepackage{hyperref}
\hypersetup{colorlinks=true, allcolors=[RGB]{010 090 200}} % цвет ссылок 
\newcommand{\lr}[1]{\left({#1}\right)} % команда для скобок

\title{Конспектик к экзамену по алгему}
\author{Васильев Павел}
%\linespread{1}
\usepackage{amsmath}

\usepackage{graphicx}
\usepackage{ifpdf}
\ifpdf
\DeclareGraphicsRule{*}{mps}{*}{}
\fi
\usepackage{graphicx}
\usepackage{color}
\graphicspath{ {images/} }

%\renewcommand{\familydefault}{\sfdefault}

\begin{document}

\section*{HOW TO заботать экзамен по алгебре (2 семестр)}

\subsection*{Вопросы}

\begin{enumerate}
\item Перестановки, подстановки, четность, нечетность. Свойства. 
\item Определение определителя квадратной матрицы. Простейшие свойства определителя. 
\item Полураспавшиеся и распавшиеся матрицы. Определитель полураспавшейся и квазидиагональной матриц.
\item Определитель произведения квадратных матриц.
\item Обратное отображение и обратимые матрицы. Матрица, обратная к данной. Критерий обратимости в терминах её определителя.
\item Крамеровы системы линейных уравнений. Формулы Крамера.
\item Построение кольца многочленов. Простейшие свойства многочленов.
\item Деление многочленов с остатком.
\item Делимость многочленов. Свойства отношения делимости. Отношение ассоциированности.
\item Наибольший общий делитель многочленов. Теорема существования. Ассоциированность НОД. 
\item Выражение НОД через исходные многочлены.
\item Взаимно простые многочлены и их свойства.
\item Неприводимые многочлены и их свойства. Теорема о разложении в произведение неприводимых многочленов. Каноническое разложение.
\item Производная многочлена и ее свойства. Кратные множители многочлена. Алгоритм выделения кратных множителей.
\item Значение многочлена. Корни многочлена. Теорема Безу. Равенство многочленов, совпадающих как функции.
\item Интерполяционный многочлен Лагранжа.
\item Разложение многочленов над полем действительных чисел.
\item Многочлены над полем рациональных чисел и кольцом целых чисел. Примитивные многочлены и их свойства.
\item Неприводимые многочлены над полем рациональных чисел и кольцом целых чисел. Критерий Эйзенштейна.
\item Сопряженное отображение. Свойства сопряженного отображения. Единственность сопряженного отображения. 
\item Дважды сопряженное отображение. Существование сопряженного отображения.
\item Изометрические отображения и их свойства.
\item Собственные числа и собственные векторы линейного преобразования. Характеристический многочлен линейного преобразования. Условия существование собственных векторов линейного преобразования.
\item Самосопряженные линейные преобразования и их свойства. Строение матрицы самосопряженного линейного преобразования.
\item Сингулярное представление линейного отображения.
\item Псевдообратный оператор.
\item Билинейные и квадратичные функции. Билинейные и квадратичные формы. Матрица билинейной формы. Конгруэнтные формы и матрицы. 
\item Квадратичные функции и формы. Связь с симметричными билинейными функциями и формами. Конгруэнтность квадратичных функций и форм.
\item Канонический и нормальный виды квадратичной формы. Приведение формы к каноническому виду. Единственность нормального вида над полем комплексных чисел. 
\item Закон инерции вещественных квадратичных форм. 
\item Знакоопределенные квадратичные формы. Критерий Сильвестра. 
\item Приведение вещественной квадратичной формы к главным осям. 
\item Эллипс.
\item Гипербола.
\item Директориальное свойство эллипса.
\item Директориальное свойство гиперболы.
\item Парабола.
\item Приведение кривых второго порядка к каноническому виду.
\item Цилиндрические поверхности
\item Поверхности вращения.
\item Эллипсоид.
\item Однополостный гиперболоид. Асимптотический конус.
\item Двуполостный гиперболоид. Асимптотический конус.
\item Эллиптический параболоид.
\item Гиперболический параболоид.
\item Приведение поверхностей второго порядка к каноническому виду.
\end{enumerate}


\subsection*{Перестановки, подстановки, четность, нечетность. Свойства.}

 Перестановка - любая последовательность длины $n$ $(i_1, ..., i_n)$, в которой каждое число от 1 до $n$ ввходит 1 раз.

 Число инверсий - количество пар $(i,j)$, $i < j$ и номер $j$ меньше $i$.

 Перестановка \textit{чётная}, если в ней чётное число инверсий и нечётная иначе.

\begin{htheorem}\textbf{Теорема}.

Пусть $g$ - перестановка. При перестановке любой пары элементов чётность перестановки поменяется.
\end{htheorem}


 \textit{Подстановка} на $\{ 1, ..., n\}$ - биекция на $\{ 1, ..., n\}$

Запись: 

\[ g = \begin{pmatrix}
1 & 2 & ... & n \\ i_1 & i_2 & ... & i_n
\end{pmatrix}
\]

Подстановка состоит из двух перестановок.

Подстановка чётная, если число инверсий в ней чётное.

\begin{htheorem}\textbf{Теорема}.


\end{htheorem}
\begin{enumerate}
\item Любая подстановка может быть представлена в каноническом виде
\item Чётность не зависит от упорядочения верхнего ряда
\end{enumerate}

\begin{htheorem}\textbf{Предложение}.

Обратная подстановка $g^{-1} = \begin{pmatrix}
i_1 & i_2 & ... & i_n \\ 1 & 2 & ... & n
\end{pmatrix}$ имеет такую же чётность как исходная.

\end{htheorem}

\subsection*{Определение определителя квадратной матрицы. Простейшие свойства определителя.}
$S_n$ - множество подстановок на $\{ 1, ..., n\}$.
\[ |S_n| = n! \]

\textit{Определителем} матрицы $A$ называется число 
\[ |A| = \det A = \sum_{g \in S_n} (-1)^g a_{1 g(1)} a_{2 g(2)} ... a_{n g(n)} \]


\begin{htheorem}\textbf{Теорема}.

\[ |A| = |A^T| \]
\end{htheorem}

\begin{htheorem}\textbf{Теорема}.

Все свойства определителя, справедливые для строк, также справедливы и для столбцов.


\subsubsection*{Свойства определителя}
\begin{itemize}
\item $|A| = |A^T|$
\item $\begin{vmatrix}
a_{11} & ... & a_{1n} \\ ... & ... & ... \\ ta_{k1} & ... & ta_{kn} \\ ... & ... & ... \\ a_{n1} & ... & a_{nn}
\end{vmatrix} = t \begin{vmatrix}
a_{11} & ... & a_{1n} \\ ... & ... & ... \\ a_{k1} & ... & a_{kn} \\ ... & ... & ... \\ a_{n1} & ... & a_{nn}
\end{vmatrix}$

Следствие: $|tA| = t^n|A|$

\item Если определитель содержит нулевую строку, то он равен 0
\item Если в определителе поменять местами 2 строки, то определитель поменяет знак
\item Если в определителе есть однаковые строки, то он равен 0
\item Если в определителе есть пропорциональные строки, то он равен 0
\item Разложение в сумму:

$\begin{vmatrix}
a_{11} & ... & b_{1k}+c_{1k} & ... & a_{1n} \\ ... & ... & ... & ... \\
a_{m1} & ... & b_{mk}+c_{mk} & ... & a_{mn} \\  ... & ... & ... & ... \\
a_{n1} & ... & b_{nk}+c_{nk} & ... & a_{nn}
\end{vmatrix} = \begin{vmatrix}
a_{11} & ... & b_{1k} & ... & a_{1n} \\ ... & ... & ... & ... \\
a_{m1} & ... & b_{mk} & ... & a_{mn} \\  ... & ... & ... & ... \\
a_{n1} & ... & b_{nk}& ... & a_{nn}
\end{vmatrix} + \begin{vmatrix}
a_{11} & ... & c_{1k} & ... & a_{1n} \\ ... & ... & ... & ... \\
a_{m1} & ... & c_{mk} & ... & a_{mn} \\  ... & ... & ... & ... \\
a_{n1} & ... & c_{nk} & ... & a_{nn}
\end{vmatrix} $



\item Если к одной строке прибавить $t \cdot$(другая строка), то определитель не поменяется
\item Разложение по строке.\textit{Алгебраическое дополнение} элемента $a_{ij}$ - это $A_{ij} = (-1)^{i+j} |M_{ij}|$.

$|A| = a_{k1}A_{k1} + ... + a_{kn}A_{kn}$.

\end{itemize}
\end{htheorem}

\subsection*{Полураспавшиеся и распавшиеся матрицы. Определитель полураспавшейся и квазидиагональной матриц.}

Матрица вида $\begin{vmatrix}
A & N \\ O & B
\end{vmatrix}$, где $A, B$ - матрицы, $O$ - нулевая матрица, называется \textbf{полураспавшейся}.

\begin{htheorem}\textbf{Теорема}.

$\begin{vmatrix}
A & N \\ O & B
\end{vmatrix} = |A||B|$
\end{htheorem}

\subsection*{Определитель произведения квадратных матриц.}

\begin{htheorem}\textbf{Теорема}.

Если $A,B$ - квадратные матрицы, то $|AB| = |A||B|$
\end{htheorem}

\subsection*{Обратное отображение и обратимые матрицы. Матрица, обратная к данной. Критерий обратимости в терминах её определителя.}

Пусть $A$ - произвольная матрица. $B$ - обратная к ней, если $AB = BA = E$.
Если матрица обратима, то она квадратная.

\begin{htheorem}\textbf{Теорема}.

Квадратная матрица $A$ обратима $\Leftrightarrow |A| \neq 0$. Если $|A| \neq 0$, то $A^{-1} = \frac{1}{|A|} (A^*)^T$
\end{htheorem}

\subsubsection*{Свойства обратных матриц}
\begin{itemize}
\item $(A^{-1})^{-1} = A$
\item $(A^{-1})^T = (A^T)^{-1}$
\item $(AB)^{-1} = B^{-1} A^{-1}$
\item $|A^{-1}| = |A|^{-1} = \frac{1}{|A|}$
\end{itemize}

\subsection*{Крамеровы системы линейных уравнений. Формулы Крамера.}

СЛУ называется \textit{крамеровской}, если в ней число уравнений равно числу неизвестных.

Пусть $\Delta_i$ - определитель матрицы, полученной заменой $i$-ого столбца основной матрицы на столбец свободных членов этой системы.

\begin{htheorem}\textbf{Теорема}.

Если $\Delta \neq 0$, то система имеет единственное решение и $\Delta x_i = \frac{\Delta_i}{\Delta}$
\end{htheorem}

\subsection*{Построение кольца многочленов. Простейшие свойства многочленов.}

\textit{Многочленом от одной переменной над кольцом $K$} называетсся выражение  

\[ f = f(x) = a_0 + a_1x + ... + a_nx^n = \sum_{i=0}^n a_ix^i, a_i \in K \]

Если $a_0 \neq 0$ - то $a_n$ - старший коэффициент, $n$ - степень многочлена.

\begin{htheorem}\textbf{Теорема}.
\begin{enumerate}
\item $K$ - кольцо $\Rightarrow K[x]$ - кольцо
\item Если $K$ - коммутативное кольцо, то $K[x]$ - коммутативное кольцо
\item Если $K$ содержит единицу, то $K[x]$ содержит единицу
\item Если $K$ не имеет делителей нуляи, то $K[x]$ не имеет делителей нуля
\end{enumerate}

\end{htheorem}

\subsection*{Деление многочленов с остатком.}

\begin{htheorem}\textbf{Теорема}.

Пусть $F$ - поле и $f, g \in F[x], g \neq 0$. Тогда $\exists q, r \in F[x]:$
\[ f = qg + r \]

$\deg r < \deg g$

$q$ - частное, $r$ - остаток.
\end{htheorem}

\subsection*{Делимость многочленов. Свойства отношения делимости. Отношение ассоциированности.}



\subsubsection*{Свойства отношения делимости}
$f, g, h$ - ненулевые многочлены.
\begin{itemize}
\item Рефлексивность
\item Транзитивность
\item Антисимметричности нет из-за наличия ассоциированных многочленов
\end{itemize}

Многочлены $f$ и $g$ \textit{ассоциированные}, если существует ненулевой $\gamma \in F$ такой, что $f = \gamma g$.

Многочлены $f$ и $g$ ассоциированны $\Leftrightarrow f|g$ и $g|f$.

Отношение ассоциированности является отношением эквивалентности на множестве $F[x]$.

\subsubsection*{Ещё свойства}
\begin{itemize}
\item $f|g \Rightarrow \forall h: f|(gh)$
\item $f|g_1 \text{и} f|g_2 \Rightarrow f|(g_1 + g_2)$
\item $f|g_1, ..., f|g_n \Rightarrow \forall h_1, ..., h_n : f|(h_1g_1 + ... + h_ng_n)$
\end{itemize}

\subsection*{Наибольший общий делитель многочленов. Теорема существования. Ассоциированность НОД.}

Пусть $F$ - поле и $f, g \in F[x]$

$h(x) \in F[x]$ - НОД($f,g$), если $h | f$ и $h|g$ и $\forall p \in F[x]: (p | f$ и $ p | g \Rightarrow p | h)$

\begin{htheorem}\textbf{Теорема о НОДе}.

Для любых ненулевых $f$ и $g$ над полем $F$ существует НОД и для некоторых $u, v \in F[x]$

\[ \text{НОД}(f, g) = uf + vg\]
\end{htheorem}

\subsection*{Взаимно простые многочлены и их свойства.}

$f$ и $g$ взаимно простые, если $\text{НОД}(f,g) = 1$

\begin{htheorem}\textbf{Преложение}.

Пусть $f, g, h$ - многочлены над полем $F$.

\begin{itemize}
\item $\text{НОД}(f,g) = 1, f|h$ и  $g|h \Rightarrow (fg)|h$
\item $\text{НОД}(f,g) = 1$ и $f|(gh) \Rightarrow f|h$
\end{itemize}

\end{htheorem}

\subsection*{Неприводимые многочлены и их свойства. Теорема о разложении в произведение неприводимых многочленов. Каноническое разложение.}

Многочлен $f \in F[x]$ \textit{неприводимый} над полем $F$, если $\deg f \geq 1$ и $\forall g, h \in F[x]: f = gh \Rightarrow \deg g = f \text{или} \deg h = \deg f$

\begin{htheorem}\textbf{Теорема}.

В $F[x]$ каждый многочлен степени $n \geq 1$ однозначно представим как произведение неприводимых многочленов.
\end{htheorem}

\begin{htheorem}\textbf{Предложение о неприводимых многочленах}.

Если $g$ - неприводимый над $F$ и $g$ делит произведение некоторых многочленов $h_1, ..., h_m$, то $g$ делит один из $h_i$.
\end{htheorem}

\subsection{Производная многочлена и ее свойства. Кратные множители многочлена. Алгоритм выделения кратных множителей.}

\textit{Производной} многочлена $f(x) = a_nx^n + ... + a_1x + a_0$ называется многочлен $f'(x) = na_nx^{n-1} + ... + 2a_2x + a_1$.

\subsubsection*{Свойства производной}
\begin{itemize}
\item $(f+g)' = f'+g'$
\item $(fg)' = f'g+fg'$
\item $\forall c \in F: (cf)' = cf'$
\item $(f^m)' = ,f^{m-1}$
\end{itemize}

\begin{htheorem}\textbf{Теорема}.

Пусть число $c$ - корень многочлена $f(x)$ кратности $k \geq 1$. Тогда $c$ является корнем многочлена $f'(x)$ кратности $k-1$.
\end{htheorem}


\begin{htheorem}\textbf{Теорема}.

Число $c$ - корень $f(x)$ кратности $k \Leftrightarrow$ $c$ - корень $f(x)$ и $c$ корень $f'(x)$ кратности $k-1$.
\end{htheorem}

\begin{htheorem}\textbf{Теорема}.

Если $p$ неприводимый многочлен и $p' \neq 0$, то $\text{НОД}(p, p') = 1$
\end{htheorem}

\begin{htheorem}\textbf{Замечание}.

Многочлен $\frac{f}{\text{НОД}(f, f')}$ имеет те же корни, что и $f$, но не имеет кратных корней.
\end{htheorem}

\subsection*{Значение многочлена. Корни многочлена. Теорема Безу. Равенство многочленов, совпадающих как функции.}

\begin{htheorem}\textbf{Теорема Безу}.

Пусть $f(x)$ многчлен над полем $F$, $\alpha \in F$. Остаток  от деления $f(x)$ на $x-\alpha$ равен $f(\alpha)$.
\end{htheorem}

\begin{htheorem}\textbf{Замечание}.

$\alpha$ - корень многочлена $f(x)$ $\Leftrightarrow (x-\alpha) | f(x)$
\end{htheorem}

\begin{htheorem}\textbf{Основная теорема алгебры комплексных чисел}.

Любой многочлен положительной степени над $\mathbb{C}$ имеет хотя бы 1 комплексный корень.
\end{htheorem}

\begin{htheorem}\textbf{Теорема}.

Неприводимыми над $\mathbb{C}$ являются линейные двучлены и только они
\end{htheorem}

\begin{htheorem}\textbf{Следствие}.

Любой многочлен степени $n > 0$ над полем $\mathbb{C}$ однозначно представим как произведение $n$ множителей.
\end{htheorem}

\begin{htheorem}\textbf{Лемма о корнях и комплексной сопряжённости}.

Если $f$ - многочлен над $\mathbb{R}$, $\gamma \in \mathbb{C}$ - его корень, то и $\overline{\gamma}$ - тоже корень.

\end{htheorem}


\begin{htheorem}\textbf{Теорема}.

Неприводимы над $\mathbb{R}$ линейные двучлены и квадратные трёхчлены с отрицательным дискриминантом и только они.
\end{htheorem}

\begin{htheorem}\textbf{Следствие}.

Любой многочлен степени $n>0$ над $\mathbb{R}[x]$ однозначно представим как $k \leq \lfloor n/2 \rfloor$ квадратных трёхчленов с отрицательным дискриминантом и $n-2k$ лиейных двучленов.
\end{htheorem}

\subsection*{Интерполяционный многочлен Лагранжа.}

\begin{htheorem}\textbf{Теорема}.

Многочлен $f(x)$ степени $n$ однозначно определяется своими значениями в $(n+1)$ попарно различных точках.

\end{htheorem}

\[
f(x) = f(x_0) \frac{(x-x_1)(x-x_2)...(x-x_n)}{(x_0-x_1)(x_0-x_2)...(x_0-x_n)} + f(x_1) \frac{(x-x_0)(x-x_2)...(x-x_n)}{(x_1-x_0)(x_1-x_2)...(x_1-x_n)} + ...
\]

\subsection*{Многочлены над полем рациональных чисел и кольцом целых чисел. Примитивные многочлены и их свойства.}

\begin{htheorem}\textbf{Теорема}.

Пусть $f \in \mathbb{Z}[x]$. Многочлен разложим над $\mathbb{Z}[x] \Leftrightarrow$ он разложим над $\mathbb{Q}[x]$.
\end{htheorem}


Многочлен называется \textit{примитивным}, если НОД его коэффициентов равен 1.

\subsection*{Неприводимые многочлены над полем рациональных чисел и кольцом целых чисел. Критерий Эйзенштейна.}

\begin{htheorem}\textbf{Критерий Эйзенштейна (не критерий)}.

Пусть $f(x) = a_nx^n + ... + a_0$ - многочлен с целыми коэффициентами. Если существует такое простое число $p$, что старший коэффициент $a_n$ не делится на $p$, все остальные $a_{n-1}, ... , a_0$ делятся на $p$ и $a_0$ не делится на $p^2$, то $f$ неприводим над $\mathbb{Q}$
\end{htheorem}

\subsection*{Сопряженное отображение. Свойства сопряженного отображения. Единственность сопряженного отображения. }

Отображение $\mathcal{B}: V \rightarrow U$ называется \textit{сопряжённым} к $\mathcal{A}$, если

\[ \forall x \in U \forall y \in V: (\mathcal{A}x, y) = (x, \mathcal{A} y)\]

\begin{htheorem}\textbf{Предложение}.

Если для $\mathcal{A}: U \rightarrow U$ существует сопряжённое отображение, то оно определяется однозначно и является линейным отображением из $V$ в $U$.

\end{htheorem}

\begin{htheorem}\textbf{Лемма}.

Пусть $W$ - евклидово или унитарное пространство. Если $y,z \in W, (x,y) = (x,z) \forall x \in W$, то $y=z$.
\end{htheorem}

\subsubsection*{Свойства сопряжённыых отображений:}
\begin{itemize}
\item $(\mathcal{A} + \mathcal{B})^* = \mathcal{A}^*+\mathcal{B}^*$
\item $(\lambda \mathcal{A})^* = \overline{\lambda} \mathcal{A}^*$
\item $(\mathcal{A}^*)^* = \mathcal{A}$
\end{itemize}


\begin{htheorem}\textbf{Следствие}.

Если $\mathcal{A} : U \rightarrow V$ евклидовых или унитарных пространств $U,V$ обладает сопряжённым отображением, то $\mathcal{A}$ - линейное.

\end{htheorem}

\begin{htheorem}\textbf{Теорема}.

Пусть для отображений $\mathcal{A}: U \rightarrow V$ и $\mathcal{B}: V \rightarrow W$ евклидовых или униитарных пространств $U, V, W$ существуют сопряжённые отображения. Тогда $(\mathcal{B} \mathcal{A})^* = \mathcal{A}^* \mathcal{B}^*$
\end{htheorem}

\begin{htheorem}\textbf{Теорема}.

Пусть $\mathcal{A} : U \rightarrow B$ - линейное отображение конечномерного евклидова или унитарного пространства $U, V$. Тогда существует единственное $\mathcal{A}^*: V \rightarrow U$, которое является линейным.
\end{htheorem}

\subsection*{Изометрические отображения и их свойства.}

Оператор $\mathcal{A} : V \rightarrow V$ в евклидовом (эрмитовом) пространстве, удовлетворяющихх одному из эквивалентных условий (ниже), является ортогональным (унитарным). Иногда такие операторы называют \textit{изометрическими}.

\begin{htheorem}\textbf{Предложение}.

Следующие условия для оператора $\mathcal{A}: V \rightarrow V$ в евклидовом или эрмитовом пространстве эквивалентны:

\begin{itemize}
\item $|\mathcal{A} v| = |v| \quad \forall v \in V$
\item $(\mathcal{A} u, \mathcal{A} v) = (u,v) \quad \forall u, v \in V$
\item оператор $\mathcal{A}$ переводит ортонормированные базисы в ортонормированные, то есть если $e_1, ..., e_n$ - ортонормированный базис, то $\mathcal{A} e_1, ..., \mathcal{A} e_n$ также ортонормированный.
\item Матрица $A$ оператора $\mathcal{A}$ в ортонормированном базисе ортогональная (унитарная), то есть $A^TA=E$ (соответственно $\overline{A}^TA=E$)
\item $\mathcal{A}^* \mathcal{A} =  id$, то есть сопряжённый оператор к $\mathcal{A}$ является его обратным.
\end{itemize}
\end{htheorem}


\subsection*{Собственные числа и собственные векторы линейного преобразования. Характеристический многочлен линейного преобразования. Условия существование собственных векторов линейного преобразования.}

Пусть $V$ - векторное пространство над полем $F$, $\mathcal{A}$ - линейный оператор на $V$. Вектор $x \in V$ называется \textit{собственным вектором} оператора $\mathcal{A}$, если $x \neq 0$ и $\exists \lambda \in F:$

\[ \mathcal{A} x = \lambda x \]

Собственнными векторами оператора $\mathcal{A}$ являются вектора, координатные столбцы которых являются ненулевым решением системы $(A-\lambda E)x=0$ (*) и только они.

\textit{Собственными значениями} оператора $\mathcal{A}$ являются те значения $\lambda$, при которых (*) имеет ненулевые решения, и только они.

Если $\dim V = n$, то в системе $(A-\lambda E)x = 0$ есть $n$ уравнений и $n$ неизвестных. Такая система имеет ненулевые решения, если и только если ранг матрицы $A-\lambda E$ строго меньше $n$, то есть только если $|A-\lambda E| = 0$.

Если $A = (a_{ij})$, то $|A-\lambda E| = \begin{vmatrix}
a_{11}-\lambda & a_{12} & ... & a_{1n}\\
a_{21} & a_{22}-\lambda & ... & ... \\
... & ... & ... & ... \\
a_{n1} & ... & ... & a_{nn} - \lambda
\end{vmatrix}$ - многочлен $n$-ой степени от $\lambda$. Он называется \textit{характеристическим многочленом матрицы А}.

\begin{htheorem}\textbf{Замечание}.

Характеристические многочлены подобных матриц равны.
\end{htheorem}

\begin{htheorem}\textbf{Теорема}.

Собственные вектора, принадлежащие попарно различным собственным значениям, линейно независимы.
\end{htheorem}

\begin{htheorem}\textbf{Следствие}.

Если у $\mathcal{A}$ на $n$-мерном векторном пространстве $V$ имеется $n$ различных собственных значений, от в $V$ существует базис из собственных векторов оператора $\mathcal{A}$.
\end{htheorem}

\subsection*{Самосопряженные линейные преобразования и их свойства. Строение матрицы самосопряженного линейного преобразования.}

Линейный оператор $\mathcal{A}$ на пространстве $V$ со скалярным произведением над полем $F \in \{ \mathbb{R}, \mathbb{C} \}$ называется \textit{самосопряжённым}, если он равен своему сопряжённому, то есть если

\[ ( \mathcal{A} x, y ) = (x, \mathcal{A} y) \]


\begin{htheorem}\textbf{Замечание}.

Собственные значения самосопряжённого оператора действительны.
\end{htheorem}

\begin{htheorem}\textbf{Теорема}.

Линейный оператора $\mathcal{A}$ на пространстве $V$ со скалярным произведением самосопряжённый $\Leftrightarrow$ в $V$ есть ортонормированный базис, в котором матрица оператора $\mathcal{A}$ диагональна и действительна.
\end{htheorem}

\begin{htheorem}\textbf{Следствие}.

Если все собственные значения нормального оператора действительны, то оператор самосопряжён. (нормальный --- $\mathcal{A}^* \mathcal{A} = \mathcal{A} \mathcal{A}^*$)
\end{htheorem}

\begin{htheorem}\textbf{Следствие об эрмитовых матрицах}.

Квадратная матрица А над полем $\mathbb{C}$ эрмитова $\Leftrightarrow$ существует унитарная матрица $U$ и диагональная матрица $D$, что $S = U^* A U$.
\end{htheorem}

\begin{htheorem}\textbf{Следствие о симметрических матрицах}.

Квадратная матрица А над полем $\mathbb{R}$ симметрична $\Leftrightarrow$ существует ортогональная матрица $U$ и диагональная матрица $D$, что $D = U^T A U$.
\end{htheorem}



\end{document}
