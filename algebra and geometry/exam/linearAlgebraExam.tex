\documentclass[a4paper]{article}
\usepackage[utf8]{inputenc}
\usepackage[T2A]{fontenc}
\usepackage[12pt]{extsizes}
\usepackage[normalem]{ulem}
\usepackage{calligra}
\usepackage{pdfpages}
\usepackage{mathrsfs}
\usepackage{mathalfa}


\usepackage[english,russian]{babel}
\usepackage[left=10mm, top=10mm, right=10mm, bottom=20mm, nohead, nofoot]{geometry}
\usepackage{amsmath,amsfonts,amssymb} % математический пакет
\headsep=10mm

\usepackage[most]{tcolorbox} % для управления цветом
% НАСТРОЙКИ
%теорема
\definecolor{theorem-color}{gray}{0.90} % уровень прозрачности (1 - максимум)
\newtcolorbox{htheorem}{colback=theorem-color,grow to right by=-4mm,grow to left by=-4mm,
    boxrule=0pt,boxsep=0pt,breakable} % настройки области с изменённым фоном

%определение
\definecolor{def-color}{gray}{0.98}
\newtcolorbox{definit}{colback=def-color,grow to right by=-4mm,grow to left by=-4mm,
    boxrule=0pt,boxsep=0pt,breakable} % настройки области с изменённым фоном

%доказательсвто теоремы
\definecolor{proof-color}{gray}{0.95} % уровень прозрачности (1 - максимум)
\newtcolorbox{hproof}{colback=proof-color,grow to right by=-1mm,grow to left by=-1mm,
    boxrule=0pt,boxsep=0pt,breakable} % настройки области с изменённым фоном

%замечания, следствия
\definecolor{consectary-color}{gray}{0.95} % уровень прозрачности (1 - максимум)
\newtcolorbox{cns}{colback=consectary-color,grow to right by=-4mm,grow to left by=-4mm,
    boxrule=0pt,boxsep=0pt,breakable} % настройки области с изменённым фоном

\everymath{\displaystyle}


\usepackage{fancybox,fancyhdr}
\pagestyle{fancy}
\fancyhf{}
\fancyhead[R]{ФТ-104}
\fancyfoot[R]{\thepage}
\fancyhead[L]{экзамен алгем 2 семестр}

\usepackage{hyperref}
\hypersetup{colorlinks=true, allcolors=[RGB]{010 090 200}} % цвет ссылок 
\newcommand{\lr}[1]{\left({#1}\right)} % команда для скобок

\title{Конспектик к коллоквиуму по алгему}
\author{Васильев Павел}
%\linespread{1}
\usepackage{amsmath}

\usepackage{graphicx}
\usepackage{ifpdf}
\ifpdf
\DeclareGraphicsRule{*}{mps}{*}{}
\fi
\usepackage{graphicx}
\usepackage{color}
\graphicspath{ {images/} }

%\renewcommand{\familydefault}{\sfdefault}

\begin{document}

\section*{HOW TO заботать экзамен по алгебре (2 семестр)}

Должен признаться, что не планирую делать полный конспект к экзамену по линейной алгебре, так как это слишком долгий процесс.

\subsection*{Список билетов}
\begin{enumerate}
\item Многочлены
\begin{enumerate}
\item Основные понятия теории делимости. Отношение ассоциированности
\item Деление многочленов с остатком
\item Теорема о наибольшем общем делителе. Алгоритм Евклида
\item Существование и однозначность разложения на неприводимые многочлены в кольце многочленов над полем
\item Поле частных области. Рациональные дроби.
\item Кольцо многочленов над областью с однозначным разложением. Лемма Гаусса и ее следствия
\item Однозначность разложения на неприводимые многочлены в кольце многочленов над областью с однозначным разложением
\item Теорема Безу. Корни многочлена.
\item Классификация неприводимых многочленов над полями комплексных и действительных чисел
\item Неприводимые многочлены с целыми коэффициентами. Критерий Эйзенштейна. Алгоритм Кронекера
\item Неприводимые многочлены над полями вычетов
\item Отделение кратных множителей
\item Кратные корни. Число корней многочлена n-й степени
\item Поле разложения многочлена. Конечные поля
\item Симметрические многочлены. Формулы Виета. Основная теорема о симметрических многочленах
\item Лемма о модуле старшего члена. Основная теорема алгебры комплексных чисел.
\end{enumerate}

\item Линейные операторы
\begin{enumerate}
\item Изменение матрицы при замене базиса
\item Собственные числа и собственные значения линейного оператора. Линейные операторы простой структуры
\item Линейные функционалы. Теорема о строении линейного функционала на унитарном (евклидовом) пространстве.
\item Сопряженный оператор. Линейность сопряженного оператора. Свойства операции сопряжения. Матрица сопряженного оператора
\item Теорема Фредгольма. Альтернатива Фредгольма.
\item Нормальный оператор. Теорема о строении нормального оператора.
\item Унитарные и ортогональные операторы.
\item Самосопряженные операторы.
\item Неотрицательные самосопряженные операторы. Квадратные корни из неотрицательных самосопряженных операторов.
\item Полярное разложение оператора на унитарном (евклидовом) пространстве
\item Сингулярные числа и их применения. Теорема Эккарта-Янга
\item Псевдообратный оператор. Нормальное псевдорешение несовместной системы линейных уравнений.
\end{enumerate}

\item Жорданова теория
\begin{enumerate}
\item Разложение Фиттинга. Корневое разложение. Теорема о корневом разложении.
\item Теорема о минимальном многочлене. Теорема Гамильтона-Кэли
\item Жорданов базис нильпотентного оператора
\item Теорема Жордана
\end{enumerate}

\item Квадратичные формы
\begin{enumerate}
\item Метод Лагранжа
\item Закон инерции действительных квадратичных форм
\item Критерий Сильвестра
\end{enumerate}

\item Квадрики на плоскости и в пространстве

\begin{enumerate}
\item Эллипс, гипербола, парабола
\item Упрощение уравнения 2-го порядка от двух переменных. Классификация плоских квадрик
\item Эллипсоиды, гиперболоиды, параболоиды, конусы, цилиндры
\item Упрощение уравнения 2-го порядка от трех переменных. Классификация пространственных квадрик
\end{enumerate}

\end{enumerate}


\section*{Линейные операторы}

\subsection*{Изменение матрицы при замене базиса}

Пусть $V$ - конечномерное векторное пространство над полем $F$, а $P = \{ p_1, p_2, ..., p_n \}$ и $Q = \{ q_1, q_2, ..., q_n \}$ - два базиса этого пространства. \textbf{Матрицей перехода от базиса $P$ к базису $Q$} нызывается $n \times n$ матрицы, $i$-ый столбец которой (для каждого $i=1,...,n$) есть координатный столбец вектора $q_i$ в базисе $P$.

Обозначается как $T_{P \rightarrow Q}$.

\begin{htheorem}\textbf{Предложение}.

Пусть $P$ и $Q$ - два базиса пространства $V$.
Тогда для любого $x \in V$

\[
[x]_P = T_{P \rightarrow Q} [x]_Q
\]

\[
[x]_P = T_{P \rightarrow Q} T_{Q \rightarrow P} [x]_P
\]
\end{htheorem}

\begin{htheorem}\textbf{Предложение}.

Пусть $P$ и $Q$ - два базиса пространства $V$. Матрица $T_{P \rightarrow Q}$ обратима и обратной к ней является матрицаа обратного перехода $T_{Q \rightarrow P}$.
\end{htheorem}

\begin{htheorem}\textbf{Теорема (о замене матрицы)}.

Пусть $V$ и $W$ - конечномерные векторные пространства над полем $F$, $P_1, P_2$ - базисы пространства $V, Q_1, Q_2$ - базисы пространства $W$, а $\mathcal{A}: V \rightarrow W$ - линейный оператор. Тогда 

\[
A_{P_2, Q_2} = T_{Q_2 \rightarrow Q_1} A_{P_1,Q_1} T_{P_1 \rightarrow P_2}
\]

Важный частный случай $W=V$. Тогда $Q_1 = P_1, Q_2=P_2$.
\end{htheorem}

\textbf{Определение.} Квадратные матрицы $A$ и $B$ над некоторым полем $F$ называются подобными над $F$, если существует невырожденная квадратная матрица над $F$ такая, что $B = T^{-1}AT$.

Таким образом, все матрицы одного и тогоже линейного оператора $\mathcal{A}: V \rightarrow V$ подобны между собой.



\end{document}
