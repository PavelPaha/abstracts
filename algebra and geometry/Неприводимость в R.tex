\documentclass[a4paper]{article}
\usepackage[utf8]{inputenc}
\usepackage[T2A]{fontenc}
\usepackage[12pt]{extsizes}

\usepackage[english,russian]{babel}
\usepackage[left=10mm, top=10mm, right=10mm, bottom=20mm, nohead, nofoot]{geometry}
\usepackage{amsmath,amsfonts,amssymb} % математический пакет
\headsep=10mm

\usepackage[most]{tcolorbox} % для управления цветом
% НАСТРОЙКИ
%теорема
\definecolor{theorem-color}{gray}{0.90} % уровень прозрачности (1 - максимум)
\newtcolorbox{htheorem}{colback=theorem-color,grow to right by=-4mm,grow to left by=-4mm,
    boxrule=0pt,boxsep=0pt,breakable} % настройки области с изменённым фоном

%определение
\definecolor{def-color}{gray}{0.98}
\newtcolorbox{definit}{colback=def-color,grow to right by=-4mm,grow to left by=-4mm,
    boxrule=0pt,boxsep=0pt,breakable} % настройки области с изменённым фоном

%доказательсвто теоремы
\definecolor{proof-color}{gray}{0.95} % уровень прозрачности (1 - максимум)
\newtcolorbox{hproof}{colback=proof-color,grow to right by=-1mm,grow to left by=-1mm,
    boxrule=0pt,boxsep=0pt,breakable} % настройки области с изменённым фоном

%замечания, следствия
\definecolor{consectary-color}{gray}{0.95} % уровень прозрачности (1 - максимум)
\newtcolorbox{cns}{colback=consectary-color,grow to right by=-4mm,grow to left by=-4mm,
    boxrule=0pt,boxsep=0pt,breakable} % настройки области с изменённым фоном



\usepackage{fancybox,fancyhdr}
\pagestyle{fancy}
\fancyhf{}
\fancyhead[R]{27.02.2023}
\fancyfoot[R]{\thepage}
\fancyhead[L]{алгем}

\usepackage{hyperref}
\hypersetup{colorlinks=true, allcolors=[RGB]{010 090 200}} % цвет ссылок 
\newcommand{\lr}[1]{\left({#1}\right)} % команда для скобок

\author{Васильев Павел}
%\linespread{1}
\usepackage{amsmath}

\usepackage{graphicx}
\usepackage{ifpdf}
\ifpdf
\DeclareGraphicsRule{*}{mps}{*}{}
\fi
\usepackage{graphicx}
\usepackage{color}
\graphicspath{ {images/} }

\begin{document}
\section*{Неприводимость над $\mathbb{R}$}

\begin{htheorem}\textbf{Лемма}.
Пусть $f(x) \in \mathbb{R}[x]$. Если число $z \in \mathbb{C}$ является корнем $f(x)$, то и сопряжённое ему число $\overline{z}$ тоже является корнем $f(x)$.
\end{htheorem}

\begin{hproof}\textbf{Доказательство.}
$f(x) = a_nx^n + ... + a_0, a_i \in \mathbb{R}$

$z$ - корень $\Rightarrow f(z) = a_nz^n + ... + a_1z+a_0 = 0$.

Возьмём комплексное сопряжённое от обеих частей. Получим $ \overline{f(z)} = \overline{a_nz^n + ... + a_1z+a_0} = \overline{0}$

$\overline{a_n}( \overline{z^n})+( \overline{a_{n-1}})(\overline{z^{n-1}}) + ... + \overline{a_1} \overline{z} + \overline{a_0} = 0$

$a_i \in \mathbb{R}$

$a_n( \overline{z^n}) +a_{n-1}( \overline{z^{n-1}}) + ... + a_1 \overline{z} + a_0 = 0 \Rightarrow f( \overline{z}) = 0$ чтд.

\end{hproof}

\begin{htheorem}\textbf{Теорема}.
Над $\mathbb{R}$ неразложимимы являются многочлены только первой и второй степени (с отрицательным дискриминантом)

\end{htheorem}
\begin{hproof}\textbf{Доказательство.}

$\mathbb{R} \subseteq \mathbb{C} \Rightarrow f(x) \in \mathbb{R}[x]$ имеет в точности $n$ корней над полем $\mathbb{C}$.

Множество корней можно разбить на два типа:
\begin{enumerate}
\item $a_1, ..., a_k$ - вещественные корни
\item $z_1, \overline{z_1}, z_2, \overline{z_2}, ..., z_m$ - комплексные корни, у каждого из которых есть сопряженная пара (смотри предыдущую лемму)
\end{enumerate}

$n$ - степень $f(x) \Rightarrow k+2m = n$

$f(x) = (x-a_1)...(x-a_k)(x-z_1)(x-\overline{z_1})...(x-z_m)(x- \overline{z_m})$

Перемножим пары скобок, которые содержат сопряжённые числа.

$(x-z_i)(x-\overline{z_i}) = x^2 - x(z_i + \overline{z_i}) + z_i \overline{z_i} = x^2 - 2c_i + (x_i^2 + d_i^2), (z_i = c_i + id_i, z_i + \overline{z_i} = 2c_i, z_i \overline{z_i} = c_i^2 + d_i^2$

Для каждого $i = 1, ..., m$ получаем многочлен второй степени $f_i(x) = x^2 - 2c_i + (x_i^2 + d_i^2)$
\end{hproof}

\begin{htheorem}\textbf{Следствие}.
Любой многочлен над $\mathbb{R}$, имеющий нечётную степень, имеет вещественный корень.
\end{htheorem}

\subsection*{Разложение многочленов над $\mathbb{Q}$ и $\mathbb{Z}$}



\begin{htheorem}\textbf{Теорема}.
Пусть $f(x) \in \mathbb{Z}[x]$. Многочлен разложим над $\mathbb{Z}[x] \Leftrightarrow$ он разложим над $\mathbb{Q}[x]$.
\end{htheorem}

\begin{hproof}\textbf{Доказательство.}
Пусть $f(x) \in \mathbb{Z}[x]$.

Пусть $f(x)$ разложим над $\mathbb{Q}[x]$.

$f(x) = a_nx^n + ... + a_1x + a_0, a_i \in \mathbb{Z}, i = 0, ..., n$

$f(x) = g(x)h(x), g(x), h(x) \in \mathbb{Q}[x]$

Рассмотрим $g(x)h(x)$.

$\displaystyle g(x) = \frac{c_1}{b_1}g_1(x)$

$\displaystyle h(x) = \frac{c_2}{b_2}h_1(x)$

$b_1$ - общий знаменатель, $c_1$ - общий множитель в числителе.



\begin{htheorem}
\textbf{Определение}. Многочлен называется \textbf{примитивным}, если НОД его коэффициентов равен 1.
\end{htheorem}
$g_1(x)$ и $h_1(x) \in \mathbb{Z}[x]$ (примитивные)

$\displaystyle f(x) = a_nx^n + ... + a_1x + a_0 = g(x)h(x) = \frac{c_1c_2}{b_1b_2}(g_1(x)h_1(x))$

$g_1(x)h_1(x)$ - примитивный многочлен с целыми коэффициентами.

Если $\displaystyle \frac{c_1c_2}{b_1b_2} = \frac{p}{q}$, то $\displaystyle \frac{p}{q} f_1(x)$ - многочлен, в котором есть рациональная дробь (поскольку он примитивный, то НОД коэффициентов равен 1 и при $q \neq 1$ остаётся коэффициент, не являющийся целым числом)
\end{hproof}


\begin{htheorem}\textbf{Лемма Гаусса}.
Произведение примитивных многочленов является примитивным многочленом.
\end{htheorem}

\begin{hproof}\textbf{Доказательство.}
Пусть $g(x) = b_kx^k + ... + b_1x+b_0 (b_i \in \mathbb{Z}), h(x) = c_mx^m + ... + c_1x + c_0 (c_i \in \mathbb{Z}) \in \mathbb{Z}[x]$ являются примитивными.

$f(x) = a_nx^n + ... + a_1x + a_0 = (b_kx^k + ... + b_1x+b_0)(c_mx^m + ... + c_1x + c_0)$

$a_n = c_mb_{n-m}$

$a_{n-1} = c_{m-1}b_{n-m-1} + c_{m-1}b_{n-m}$

$...$

$a_i = c_0b_i + c_1b_{i-1} + ... + c_ib_{k-i}$

Пусть $f(x)$ непримитивный $\Rightarrow \exists d \neq 1$  такое, что $d$ делит любой коэффициент $f(x)$ (будем считать, что $d$ - простое).

Возьмём наименьший индекс $i_0$ такой, что $c_{i_0}$ не делится на $d$ (если все коэффициенты $h(x)$ делятся на $d$, то $h(x)$ непримитивный)

Возьмём наименьший индекс $j_0$ такой, что $b_{j_0}$ не делится на $d$ (если все коэффициенты $h(x)$ делятся на $d$, то $h(x)$ непримитивный)


Рассмотрим коэффициент $a_{i_0 + j_0}$ при степени $x^{i_0+j_0}$:

$a_{i_0+j_0} = c_0b_{i_0+j_0} + c_1b_{i_0+j_0-1} + ... + c_{i_0}b_{j_0} + c_{i_0+1}b_{j_0-1} + ... + c_{i_0+j_0}b_0$

Все члены до и после $c_{i_0}b_{j_0}$ делятся на $d$, а $c_{i_0}b_{j_0}$ на $d$ не делится. Пришли к противоречию. чтд
\end{hproof}

\begin{htheorem}\textbf{Теорема (Критерий Эйзенштейна}.
Пусть $f(x) \in \mathbb{Z}[x], f(x) = a_nx^n + ... + ax_1 + a_0$. Если существует простое число $p$ такое, что \begin{enumerate}
\item $p$ не делит $a_n$
\item $p$ делит все остальные $a_i (i =0, ..., n-1)$
\item $p^2$ не делит $a_0$
\end{enumerate}

Тогда многочлен $f(x)$ неприводим над $\mathbb{Q}$
\end{htheorem}

\begin{hproof}\textbf{Доказательство.}
$f(x) \in \mathbb{Z}[x]$ и пусть выполняется условия критерия Эйзенштейна, то есть существует $p$, для которого выполняются условия 1) - 3) и при этом $f(x) = g(x)h(x), (f(x), g(x) \in \mathbb{Z}[x])$

$g(x) = b_kx^k + ... + b_1x+b_0 (b_i \in \mathbb{Z}), h(x) = c_mx^m + ... + c_1x + c_0 (c_i \in \mathbb{Z}) \in \mathbb{Z}[x]$

$a_0 = b_0c_0$

$p^2$ не делит $a_0 \Rightarrow$ либо $p|b_0$ b $p$ не делит $c_0$ либо наоборот.

Пусть $p|c_0$ и $p$ не делит $b_0$ (второй случай рассматривается аналогично).

$a_1 = b_1c_0 + c_1b_0$. Отсюда получаем, что так как $p|a_1$, то $p|c_1$.

$a_2 = b_2c_0 + c_1b_1 + c_2b_0$. Отсюда получаем, что $p|c_2$

...

$a_m = b_mc_0 + ... + b_0c_m \Rightarrow p|c_m$

Берём старший коэффициент
$a_n = b_kc_m$

$p|c_m$ а значит $p|a_n$. Противоречие.
\end{hproof}

\begin{htheorem}\textbf{Теорема (о виде рациональных корней многочлена над $\mathbb{Z}$}.
Если $\frac{p}{q}$ является корнем многочлена $f(x) = a)nx^n + ... + a_1x + a_0$, то $q|a_n$ и $p|a_0$.
\end{htheorem}

\begin{hproof}\textbf{Доказательство.}
Пусть $\frac{p}{q}$ является корнем, $p$ и $q$ взаимно просты.

Просто подставляем: $f(\frac{p}{q}) = a_n(\frac{p}{q})^n + ... + a_q(\frac{p}{q}) + a_0 = 0$

$a_np^n + a_{n-1}p^{n-1}q + ... + a_1pq^{n-1} + a_0q^n = 0$

$a_np^n = - a_{n-1}p^{n-1}q - ... - a_1pq^{n-1} - a_0q^n$

$q|(a_np^n) \Rightarrow q|a_n$

$a_np^n + a_{n-1}p^{n-1}q + ... + a_1pq^{n-1} = -a_0q^n \Rightarrow p|a_0$

Противоречие. Чтд.

\end{hproof}


\end{document}
