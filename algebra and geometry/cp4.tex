\documentclass[a4paper]{article}
\usepackage[utf8]{inputenc}
\usepackage[T2A]{fontenc}
\usepackage[12pt]{extsizes}
\usepackage[normalem]{ulem}
\usepackage{calligra}
\usepackage{pdfpages}
\usepackage{mathrsfs}
\usepackage{mathalfa}


\usepackage[english,russian]{babel}
\usepackage[left=10mm, top=10mm, right=10mm, bottom=20mm, nohead, nofoot]{geometry}
\usepackage{amsmath,amsfonts,amssymb} % математический пакет
\headsep=10mm

\usepackage[most]{tcolorbox} % для управления цветом
% НАСТРОЙКИ
%теорема
\definecolor{theorem-color}{gray}{0.90} % уровень прозрачности (1 - максимум)
\newtcolorbox{htheorem}{colback=theorem-color,grow to right by=-4mm,grow to left by=-4mm,
    boxrule=0pt,boxsep=0pt,breakable} % настройки области с изменённым фоном

%определение
\definecolor{def-color}{gray}{0.98}
\newtcolorbox{definit}{colback=def-color,grow to right by=-4mm,grow to left by=-4mm,
    boxrule=0pt,boxsep=0pt,breakable} % настройки области с изменённым фоном

%доказательсвто теоремы
\definecolor{proof-color}{gray}{0.95} % уровень прозрачности (1 - максимум)
\newtcolorbox{hproof}{colback=proof-color,grow to right by=-1mm,grow to left by=-1mm,
    boxrule=0pt,boxsep=0pt,breakable} % настройки области с изменённым фоном

%замечания, следствия
\definecolor{consectary-color}{gray}{0.95} % уровень прозрачности (1 - максимум)
\newtcolorbox{cns}{colback=consectary-color,grow to right by=-4mm,grow to left by=-4mm,
    boxrule=0pt,boxsep=0pt,breakable} % настройки области с изменённым фоном

\everymath{\displaystyle}


\usepackage{fancybox,fancyhdr}
\pagestyle{fancy}
\fancyhf{}
\fancyfoot[R]{\thepage}
\fancyhead[L]{компьютерная практика 4}

\usepackage{hyperref}
\hypersetup{colorlinks=true, allcolors=[RGB]{010 090 200}} % цвет ссылок 
\newcommand{\lr}[1]{\left({#1}\right)} % команда для скобок

\author{Васильев Павел}
%\linespread{1}
\usepackage{amsmath}

\usepackage{graphicx}
\usepackage{ifpdf}
\ifpdf
\DeclareGraphicsRule{*}{mps}{*}{}
\fi
\usepackage{graphicx}
\usepackage{color}
\graphicspath{ {images/} }

%\renewcommand{\familydefault}{\sfdefault}

\begin{document}

\section*{Компьютерная практика 4, Васильев Павел ФТ-104-1}

\subsection*{1 задание}

Для начала найду линейное многообразие $\vec{c} + M$ размерности не больше $k = 2$, для которого сумма квадратов расстояний $\sum_{i=1}^n d(\vec{x_i}, c + M)^2$ минимальна. Есть теорема, которая говорит следующее:

\textit{Пусть для $\vec{x_1}, ..., \vec{x_n}$ в пространстве со скалярным произведением имеем $\vec{c} + M$ - это линейное многобразие размерности не больше $k$ такое, что для любого многообразия $\vec{c}' + M'$ размерности не больше $k$ выполняется
\[
\sum_{i=1}^n d(\vec{x_i}, c + M)^2 \leq \sum_{i=1}^n d(\vec{x_i}, c' + M')^2
\] Тогда выполняется:}

\[
\vec{c} + M = \left( \frac{1}{n} \sum_{i=1}^n \vec{x_i} \right) + M
\]

То есть $\vec{c} = \frac{1}{n} \left( \sum_{i=1}^n \vec{x_i} \right)$

Для поиска наилучшего приближения $\vec{x_1}, ..., \vec{x_n}$ многообразием достаточно найти для векторов $\vec{x_1} - \vec{c}, ..., \vec{x_n} - \vec{c}$ приближение подпространством $M$ и тогда искомое многообразие имеет вид $\vec{c} + M$. А подпространство $M$ я получу через сингулярное разложение и использую следующую теорему:

\textit{
Пусть координаты векторов $\vec{x_1}, \vec{x_2}, ..., \vec{x_n}$ в некотором ортонормированном базисе записаны в строках $n \times m$ матрицы $B$ с сингулярным разложением $B = RAS^*$ и $\vec{s_1}, \vec{s_2}, ..., \vec{s_m}$ - это векторы, чьи координаты $[\vec{s_1}], [\vec{s_2}], ...,[\vec{s_m}]$ в том же базисе записаны в столбцах матрицы $S$ (так что $||B[\vec{s_1}]|| \geq ... \geq ||B[\vec{s_m}||$). Тогда для любого подпространства $M$ размерности не больше $k$ имеем}

\[\sum_{i=1}^n d(\vec{x_i}, \langle \vec{s_1}, \vec{s_2}, ..., \vec{s_k} \rangle ) ^2 \leq \sum_{i=1}^n d( \vec{x_i}, M)^2\]


Таким образом, для получения ответа мы нормируем исходные вектора и из каждого вектора вычтем $q = \frac{1}{n} \left( \sum_{i=1}^n \vec{x_i} \right)$. Затем выполним сингулярное разложение и возьмём первые два вектора из столбцов матрицы $S$ (так как мы просто в теорему подставим $k=2$.

А так как функция должна вернуть два вектора: точка и нормаль плоскости, то вернём $q$ и векторное произведение первых двух первых векторов из столбцов матрицы $S$. 

\subsection*{2 задание}

Заметим, что, максимизируя $\sum_{q \in Q} d (q, \langle p_1, ..., p_k \rangle ^ \perp)^2$, мы минимизируем $\sum_{q \in Q} d (q, \langle p_1, ..., p_k \rangle)^2$.

Мы можем использовать теоремы, изложенные выше (на пространстве матриц в формулировке задания определено скалярное произведение, а пространство матриц $m \times n$ со  скалярным произведением: $(p, q) = \sum_{i=1}^{m}\sum_{j=1}^{n} p_{ij}q_{ij}$ (где $p, q \in \mathbb{R}^{m \times n}$) изоморфно пространству строк длины $n \cdot m$).

Тогда построим матрицу, составленную из матриц (которые являются аргументом нашей функции), преобразованных в строки, и для неё найдём сингулярное разложение. Затем возьмём первые $k$ векторов из матрицы $S$ и обратно "превратим" их в матрицы.

\end{document}
