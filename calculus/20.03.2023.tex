\documentclass[a4paper]{article}
\usepackage[utf8]{inputenc}
\usepackage[T2A]{fontenc}
\usepackage[12pt]{extsizes}

\usepackage[english,russian]{babel}
\usepackage[left=10mm, top=10mm, right=10mm, bottom=20mm, nohead, nofoot]{geometry}
\usepackage{amsmath,amsfonts,amssymb} % математический пакет
\headsep=10mm

\usepackage[most]{tcolorbox} % для управления цветом
% НАСТРОЙКИ
%теорема
\definecolor{theorem-color}{gray}{0.90} % уровень прозрачности (1 - максимум)
\newtcolorbox{htheorem}{colback=theorem-color,grow to right by=-4mm,grow to left by=-4mm,
    boxrule=0pt,boxsep=0pt,breakable} % настройки области с изменённым фоном

%определение
\definecolor{def-color}{gray}{0.98}
\newtcolorbox{definit}{colback=def-color,grow to right by=-4mm,grow to left by=-4mm,
    boxrule=0pt,boxsep=0pt,breakable} % настройки области с изменённым фоном

%доказательсвто теоремы
\definecolor{proof-color}{gray}{0.95} % уровень прозрачности (1 - максимум)
\newtcolorbox{hproof}{colback=proof-color,grow to right by=-1mm,grow to left by=-1mm,
    boxrule=0pt,boxsep=0pt,breakable} % настройки области с изменённым фоном

%замечания, следствия
\definecolor{consectary-color}{gray}{0.95} % уровень прозрачности (1 - максимум)
\newtcolorbox{cns}{colback=consectary-color,grow to right by=-4mm,grow to left by=-4mm,
    boxrule=0pt,boxsep=0pt,breakable} % настройки области с изменённым фоном



\usepackage{fancybox,fancyhdr}
\pagestyle{fancy}
\fancyhf{}
\fancyhead[R]{ФТ-104}
\fancyfoot[R]{\thepage}
\fancyhead[L]{Матанализ}

\usepackage{hyperref}
\hypersetup{colorlinks=true, allcolors=[RGB]{010 090 200}} % цвет ссылок 
\newcommand{\lr}[1]{\left({#1}\right)} % команда для скобок

\title{Конспектик к экзамену по алгему}
\author{Васильев Павел}
%\linespread{1}
\usepackage{amsmath}

\usepackage{graphicx}
\usepackage{ifpdf}
\ifpdf
\DeclareGraphicsRule{*}{mps}{*}{}
\fi
\usepackage{graphicx}
\usepackage{color}
\graphicspath{ {images/} }

\begin{document}
\section*{20.03.2023 Интеграл Римана.}

$a = x_0 < x_1 < ... ,x_{n-1} < x_n = b$

$\lambda_\tau = max(x_j-x_{j-1})$

$\xi_j \in [x_{j-1}, x_j]$

$S(f, \tau, \xi) = \sum_{j=1}^w f(\xi_j) \delta x_j$

$\exists I" \forall \epsilon \quad \exists \delta(\epsilon)>0: \quad \forall \tau \forall \xi \quad (\lambda_\tau \Rightarrow |S(f, \tau \xi) - I| < \epsilon)$

$I = \int_a^b f(x)dx$

Верхняя и нижняя сумма Дарбу:

$\displaystyle \overline{S_\tau} = \sum_{j=1}^n M_j \Delta x_j (M_j = [x_{j-1}, x_j])$

$\displaystyle \underline{S_\tau} = \sum_{j=1}^n M_j \Delta x_j (M_j = [x_{j-1}, x_j])$


\begin{htheorem}\textbf{Теорема}. $\forall \tau_1, \tau_2 \underline{S_{\tau_1}} \leq \overline{S_{\tau_2}}$
\end{htheorem}

\begin{htheorem}\textbf{Следствие}.
$\underline{S_{\tau}} \leq I_* \leq I^* \leq \overline{S_\tau}$
\end{htheorem}

\begin{htheorem}\textbf{Теорема}. Пустть $f$ определена на $[a,b]$, $f$ интегрируема на $[a,b] \Leftrightarrow$ $\exists \delta(\epsilon) > 0 \quad \forall \tau \quad (\lambda_\tau < \delta \Rightarrow \overline{S_\tau} - \underline{S_\tau} < \epsilon$
\end{htheorem}

\begin{hproof}\textbf{Доказательство.}
\begin{itemize}
\item $\Rightarrow$.  Пусть $f$ интегрируема на $[a,b]$. По определению интегрируемости, $\displaystyle \exists \delta: \quad	\forall \tau \forall \xi \quad (\lambda_\tau < \delta  \Rightarrow |S(f,  \tau, \xi) - I| < \frac{\epsilon}{3}$

$\displaystyle I - \frac{\epsilon}{3} < S(f, \tau, \xi) < I + \frac{\epsilon}{3}$

Мы хотим получить $\displaystyle I - \frac{\epsilon}{3} < \underline{S_\tau} \leq \overline{S_\tau} < I + \frac{\epsilon}{3}$

Рассмотрим левую часть неравенства $\displaystyle I - \frac{\epsilon}{3} < S(f, \tau, \xi) < I + \frac{\epsilon}{3}$
 $\displaystyle I - \frac{\epsilon}{3} < S(f, \tau, \xi)$. Возьмём инфимум по $\xi \Rightarrow I - \frac{\epsilon}{3} \leq inf S(f, \tau, \xi) = \underline{S_\tau}$.

$\underline{S_\tau} = \sum_{j=1, x \in [x_{j-1}, x_j]}^w inf f(x) \Delta x_j$

$S(f, \tau, \xi) < I + \frac{\epsilon}{3}$

$\overline{S_\tau} \leq I+\frac{\epsilon}{3} \Rightarrow I - \frac{\epsilon}{3} \leq \underline{S_\tau} \leq \overline{S_\tau} \leq I + \frac{\epsilon}{3}$.

\item $\Leftarrow$. $\underline{S_\tau} \leq I_* \leq I^* \leq \overline{S_\tau}$ (следствие из теоремы)ю

Знаем, что $\forall \epsilon \quad \exists \delta: \forall \tau (\lambda_\tau < \delta \Rightarrow \overline{S_\tau} - \underline{S_\tau} < \epsilon$

$\underline{S_\tau} \leq S_\tau \leq \overline{S_\tau}$

$|S_\tau - I_*| < \epsilon \Rightarrow I_*$ по определению интеграла является интегралом Римана.
\end{itemize}

\end{hproof}



\begin{htheorem}\textbf{Следствие}. $f$ интегрируема по Римана на $\displaystyle [a,b] \Rightarrow I_* = I^* = \int_a^b f(x) dx$
\end{htheorem}

\begin{htheorem}\textbf{Теорема (без доказательства}.
$f$ интегрируема на $[a,b] \Leftrightarrow I_* = I^*$, и при этом всегда $\displaystyle \int_a^b f(x) dx = I_* = I^*$


\end{htheorem}

\subsection*{Некоторые свойства интеграла}

\begin{enumerate}
\item Аддитивность интеграла: 
$\displaystyle \int_a^b f(x)dx = \int_a^bf + \int_c^b f, (c \in [a,b])$.

\textbf{Доказательство.} Возьмём произвольное разбиение $\tau$ на $[a,b]$

$c \in [x_{j-1}, x_j]$

Рассмотрим вспомогательное разбиение отрезка $\tau'$ на $[a,c]$ и $\tau''$ на $[c,b]$:
\begin{itemize}
\item $\tau': a<x_1<...<x_{j-1}<x$
\item $\tau'': x<x_j<...<b$
\end{itemize}

\begin{enumerate}
\item Пусть $f$ интегрируема на $[a,b]$. Покажем, что $\exists \int_a^c$ и $\exists \int_a^b$.

$f$ интегрируема на $[a,b] \Rightarrow \overline{S_\tau} - \underline{S_\tau} < \epsilon$ для $\tau: \lambda_\tau < \delta(\epsilon)$.

$\overline{S_\tau} \geq \overline{S_{\tau'}} + \overline{S_{\tau''}}$ - очевидно, так как в одной из сумм $sup_[x_j, x_{j+1}] f(x)$ может стать меньше, а больше стать не может.

Также $\underline{S_\tau} \leq \underline{S_{\tau'}} + \underline{S_{\tau''}}$ - тоже очевидно.

Итого $\epsilon > \overline{S_\tau} - \underline{S_\tau} \geq \overline{S_{\tau'}} + \overline{S_{\tau''}} - (\underline{S_{\tau'}} + \underline{S_{\tau''}}) = (\overline{S_{\tau'}} + \underline{S_{\tau'}}) + (\overline{S_{\tau''}} - \underline{S_{\tau''}}) \leq \epsilon$.

Значит $ \overline{S_{\tau'}} + \underline{S_{\tau'}} < \epsilon$ и $\overline{S_{\tau''}} - \underline{S_{\tau''}} < \epsilon$.

\item Пусть $f$ интегрируема на $[a,c]$ и $[c,b]$. Так что есть $\int_a^b f$.

$\overline{S_\tau} = \sum_{k \neq j} + M_j \Delta x_j, \underline{S_\tau} = \sum_{k \neq j} + m_j \Delta x_j$

Ужмём $|f(x)| \leq B, x \in [a,b]$.

Мы хотии получить такую оценку: $\overline{S_\tau} \leq \overline{S_{\tau'}} + \overline{S_{\tau''}} + \text{еще что-то}$.


$\displaystyle \overline{S_\tau} - ( \overline{S_{\tau'}} + \overline{S_{\tau''}}) \leq \text{еще что-то}$

$\displaystyle \overline{S_\tau} - ( \overline{S_{\tau'}} + \overline{S_{\tau''}}) = \{  \sum_{k \neq j} \text{сокращается} \} = M_j \Delta x_j - (sup_{[x_{j-1}, c]} f(x) (c-x_{j-1}) + sup_{[c, x_j]} f(x) (x_j - c)) \leq B \Delta x_j + B(c-x_{j-1} + x_j - c) = 2B \Delta x_j \quad \quad (c-x_{j-1} + x_j - c = \Delta x_j)$

$[x_{j-1}, c] \geq -B$

$[c, x_j] \leq -B$


$\displaystyle \overline{S_\tau} - ( \overline{S_{\tau'}} + \overline{S_{\tau''}}) \leq 2B \lambda_{\tau}$

А для инфимумов получается $\displaystyle \underline{S_\tau} \geq ( \underline{S_{\tau'}} + \underline{S_{\tau''}}) - 2B \lambda_{\tau}$

$\displaystyle \overline{S_\tau} \leq \overline{S_{\tau'}} + \overline{S_{\tau''}} + 2B \lambda_{\tau}$

$\displaystyle \underline{S_\tau} \geq \underline{S_{\tau'}} + \underline{S_{\tau''}} - 2B \lambda_{\tau}$

$\displaystyle \overline{S_\tau} -\underline{S_\tau} \leq \overline{S_{\tau'}} + \overline{S_{\tau''}} + 2 B \lambda_\tau - \underline{S_{\tau'}} - \underline{S_{\tau''}} + 2B \lambda_{\tau} = (\overline{S_{\tau'}} - \underline{S_{\tau'}}) + (\overline{S_{\tau''}} - \underline{S_{\tau''}}) + 4B \lambda_\tau < \epsilon$


$\displaystyle \overline{S_{\tau'}} - \underline{S_{\tau'}} < \frac{\epsilon}{3}$

$\displaystyle \overline{S_{\tau''}} - \underline{S_{\tau''}} < \frac{\epsilon}{3}$

Потому что берем минимальную дельту.

\item Докажем уже наконец-то, что $\displaystyle \int_a^b f = \int_a^c f + \int_c^b f $ (я уже устал техать)

$\displaystyle \left| \int_a^b - \left( \int_a^b + \int_a^b \right) \right| < \epsilon$

$\displaystyle \left| \int_a^b - \left( \int_a^b + \int_a^b \right) + \left( S_{\tau'} + S_{\tau''} \right) - (S_{\tau'} + S_{\tau''}) \right| < \epsilon$

$\displaystyle \left| \int_a^b - \left( \int_a^b + \int_a^b \right) + \left( S_{\tau'} + S_{\tau''} \right) - (S_{\tau'} + S_{\tau''}) \right| \leq \left| \int_a^b -  S_{\tau'} + S_{\tau''} \right| + \left| \int_a^b - S_{\tau'} \right| + \left| \int_c^b - S_{\tau''} \right| <\epsilon$

$\left| \int_a^b -  S_{\tau'} + S_{\tau''} \right| < \frac{\epsilon}{3}$

$ \left| \int_a^b - S_{\tau'} \right| < \frac{\epsilon}{3}$

$\left| \int_c^b - S_{\tau''} \right| < \frac{\epsilon}{3}$
\end{enumerate}


\end{enumerate}


\begin{htheorem}\textbf{Утверждение}. Есть интегрируемая функция. Если мы поменяем значения функции в конечном числе точек, то площадь остаентся той же.

\end{htheorem}

\begin{hproof}\textbf{Доказательство.}
Докажем для одной точки $x_0$.

Давайте поменяем значение $f(x_0)$ на $c$.
Рассмотрим функуцию $g(x) = \begin{cases} 0, x \neq x_0 \\ f(x_0)+c, x=x_0 \end{cases}$

Покажем, что $\displaystyle \int_a*B g(x)dx = 0$.

Рассмотрим прооизвольное разбиение.

$\displaystyle \forall \tau \quad  |S_\tau| \leq |C - f(x_0)| \cdot \lambda_\tau \rightarrow 0 \Rightarrow \int_a*B g(x)dx = 0$

$\displaystyle \int_a^b (f(x) + g(x)) = \int_a^b f + \int_a^b g =  \int_a^b f + 0 =  \int_a^b f$

\end{hproof}




\end{document}
