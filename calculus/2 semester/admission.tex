\documentclass[a4paper]{article}
\usepackage[utf8]{inputenc}
\usepackage[T2A]{fontenc}
\usepackage[12pt]{extsizes}
\everymath{\displaystyle}

\usepackage[english,russian]{babel}
\usepackage[left=10mm, top=10mm, right=10mm, bottom=20mm, nohead, nofoot]{geometry}
\usepackage{amsmath,amsfonts,amssymb} % математический пакет
\headsep=10mm

\usepackage[most]{tcolorbox} % для управления цветом
% НАСТРОЙКИ
%теорема
\definecolor{theorem-color}{gray}{0.90} % уровень прозрачности (1 - максимум)
\newtcolorbox{htheorem}{colback=theorem-color,grow to right by=-4mm,grow to left by=-4mm,
    boxrule=0pt,boxsep=0pt,breakable} % настройки области с изменённым фоном

%определение
\definecolor{def-color}{gray}{0.98}
\newtcolorbox{definit}{colback=def-color,grow to right by=-4mm,grow to left by=-4mm,
    boxrule=0pt,boxsep=0pt,breakable} % настройки области с изменённым фоном

%доказательсвто теоремы
\definecolor{proof-color}{gray}{0.95} % уровень прозрачности (1 - максимум)
\newtcolorbox{hproof}{colback=proof-color,grow to right by=-1mm,grow to left by=-1mm,
    boxrule=0pt,boxsep=0pt,breakable} % настройки области с изменённым фоном

%замечания, следствия
\definecolor{consectary-color}{gray}{0.95} % уровень прозрачности (1 - максимум)
\newtcolorbox{cns}{colback=consectary-color,grow to right by=-4mm,grow to left by=-4mm,
    boxrule=0pt,boxsep=0pt,breakable} % настройки области с изменённым фоном



\usepackage{fancybox,fancyhdr}
\pagestyle{fancy}

\usepackage{hyperref}
\hypersetup{colorlinks=true, allcolors=[RGB]{010 090 200}} % цвет ссылок 
\newcommand{\lr}[1]{\left({#1}\right)} % команда для скобок

\author{Васильев Павел}
%\linespread{1}
\usepackage{amsmath}

\usepackage{graphicx}
\usepackage{ifpdf}
\ifpdf
\DeclareGraphicsRule{*}{mps}{*}{}
\fi
\usepackage{graphicx}
\usepackage{color}
\graphicspath{ {images/} }

%\renewcommand{\familydefault}{\sfdefault}

\begin{document}

\section*{HOW TO заботать зачёт по матанализу 2}

\section*{Интегралы}

\begin{hproof}
Пусть $f(x)$ собственно интегрируема на любом $[a,b] (b>a)$. Тогда
\[
\int_a^{+\infty} f(x)dx = \lim_{b \rightarrow {+\infty}} \int_a^b f(x)dx
\]

\end{hproof}

\begin{hproof}Если $f(x)$ не ограничена в окрестности точки $b$ и собственно интегрируема на $[a,b-\varepsilon] (\varepsilon > 0)$, то 
\[
\int_a^b f(x)dx = \lim_{\varepsilon \rightarrow 0} \int_a^{b-\varepsilon} f(x)dx
\]

\end{hproof}


\begin{hproof}
$\int_a^{+\infty} f(x)dx$ сходится, если $\int_a^{+\infty} |f(x)|dx$ сходится.
\end{hproof}

\subsection*{Признаки сравнения}

\begin{hproof}
\textbf{1 признак сравнения}

Если:
\begin{enumerate}
\item $|f(x)| \leq F(x)$
\item $\int_a^{+\infty} F(x)dx$ сходится
\end{enumerate}

Тогда $\int_a^{+\infty} f(x)dx$ сходится абсолютно.
\end{hproof}

\begin{hproof}
\textbf{2 признак сравнения}

Пусть $\psi(x) > 0, \phi(x) = O^*(\psi(x))$ при $x \rightarrow +\infty$. 

\[\phi(x) = O^*(\psi(x)) \Leftrightarrow \lim_{x \rightarrow {+\infty}} \frac{\phi(x)}{\psi(x)} = c \neq 0\]
Тогда $\int_a^{+\infty} \psi(x) dx$ и $\int_a^{+\infty} \phi(x)dx$ сходятся и расходятся одновременно.

\end{hproof}

\begin{hproof}
\textbf{3 признак сравнения}

\begin{itemize}
\item Если $f(x) = O^*\left( \frac{1}{x^p} \right)$ при $x \rightarrow +\infty$ то

\begin{enumerate}
\item $p>1 \Rightarrow $ сходится; 
\item $p \leq 1 \Rightarrow $ расходится.
\end{enumerate}

\item Если $f(x) = O^*\left( \frac{1}{(x-b)^p} \right)$ при $x \rightarrow b+0$, то $\int_b^a f(x)dx$:

\begin{enumerate}
\item при $p \geq 1$ расходится;
\item при $p < 1$ сходится.
\end{enumerate}

\end{itemize}
\end{hproof}

\section*{Ряды}

\subsection*{Признаки сходимости}

\begin{hproof}\textbf{Д'Аламбера}

Если с некоторого момента $\frac{a_{n+1}}{a_n} \leq q < 1$, то сходится.
Если $\frac{a_{n+1}}{a_n} \geq q > 1$, то расходится.
При $q = 1$ непонятно.

\end{hproof}

\begin{hproof}\textbf{Коши}

$\lim_{n \rightarrow \infty} \sqrt[n]{a^n} = q$.

\begin{itemize}
\item $q<1 \Rightarrow$ сходится.
\item $q>1 \Rightarrow$ расходится.
\end{itemize}
\end{hproof}


\begin{hproof}\textbf{Признак сравнения для знакопостоянных рядов}

$ \forall n > N: b_n \geq a_n$ и $\sum_{n=1}^\infty b_n$ сходится $\Rightarrow \sum_{n=1}^\infty a_n$ сходится.

$\sum_{n=1}^\infty \frac{1}{n^k}$:
\begin{enumerate}
\item сходится при $k>1$
\item расходится при $k \leq 1$
\end{enumerate}
\end{hproof}

\begin{hproof}\textbf{Признак Лейбница для знакочередующихся рядов}

Если ряд знакочередующийся и $\lim_{n \rightarrow + \infty} |a_n| = 0$ и $|a_n|$ убывают монотонно, то ряд $\sum_{n=1}^{\infty} a_n$ сходится.
\end{hproof}

\subsection*{Признаки равномерной сходимости}

\begin{hproof}\textbf{Признак Вейерштрасса}


Если существует $a_n$ --- числовая последовательность такая, что $|f_n(x)| \leq a_n \quad \forall n \in \mathbb{N} \forall x \in X$ и $\sum_{n=1}^\infty$ сходится. Тогда $\sum_{n=1}^\infty f_n(x)dx$ равномерно сходится на $X$.
\end{hproof}

\begin{hproof}\textbf{Признак Дирихле}

Если $\sum_{n=1}^\infty a_n(x)$ ограничена в совокупности на $X$ и $b_n(x)$ монотонна $\forall x \in X$ и равномерно сходится к 0, то $\sum_{n=1}^\infty a_n(x)b_n(x)$ равномерно сходится на $X$.


\end{hproof}

\begin{hproof}\textbf{Признак Абеля}

Пусть $\sum_{n=1}^\infty a_n(x)$ равномерно сходится на $X$ и $|b_n(x)|$ ограничена в совокупности и $b_n(x)$ монотонна по $n$. 

Тогда $\sum_{n=1}^\infty a_n(x)b_n(x)$ сходится равномерно на $X$.
\end{hproof}

\subsection*{Функциональные последовательности}

\begin{hproof}
Пусть  $f(x) = \lim_{n \rightarrow + \infty} f_n(x)$.
Последовательность $\{ f_n(x) \}$ равномерно сходится на множестве $E$ 

к $f(x) \Leftrightarrow \lim_{n \rightarrow +\infty} \sup_{x \in E} |f_n(x) - f(x)| = 0$

\[ x'(t) = 4(- \sin t + \sin t + t \cos t) = 4 t \cos t\]

\[ y'(t) = 4(\cos t - \cos t + t \sin t) = 4 t \sin t\]

\[ |L| = \int_0^{2 \pi} \sqrt{(4t)^2 (\sin^2 t + \cos^2 t)} dt = \int _0^{2 \pi} 4t dt = 2t^2 \bigg|_0^{2 \pi} = 8 \pi^2 \]

\textit{Ответ $8 \pi^2$}
\end{hproof}

\end{document}
