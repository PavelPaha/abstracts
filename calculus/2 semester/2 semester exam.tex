\documentclass[a4paper]{article}
\usepackage[utf8]{inputenc}
\usepackage[T2A]{fontenc}
\usepackage[12pt]{extsizes}

\usepackage[english,russian]{babel}
\usepackage[left=10mm, top=10mm, right=10mm, bottom=20mm, nohead, nofoot]{geometry}
\usepackage{amsmath,amsfonts,amssymb} % математический пакет
\headsep=10mm

\usepackage[most]{tcolorbox} % для управления цветом
% НАСТРОЙКИ
%теорема
\definecolor{theorem-color}{gray}{0.90} % уровень прозрачности (1 - максимум)
\newtcolorbox{htheorem}{colback=theorem-color,grow to right by=-4mm,grow to left by=-4mm,
    boxrule=0pt,boxsep=0pt,breakable} % настройки области с изменённым фоном

%определение
\definecolor{def-color}{gray}{0.98}
\newtcolorbox{definit}{colback=def-color,grow to right by=-4mm,grow to left by=-4mm,
    boxrule=0pt,boxsep=0pt,breakable} % настройки области с изменённым фоном

%доказательсвто теоремы
\definecolor{proof-color}{gray}{0.95} % уровень прозрачности (1 - максимум)
\newtcolorbox{hproof}{colback=proof-color,grow to right by=-1mm,grow to left by=-1mm,
    boxrule=0pt,boxsep=0pt,breakable} % настройки области с изменённым фоном

%замечания, следствия
\definecolor{consectary-color}{gray}{0.95} % уровень прозрачности (1 - максимум)
\newtcolorbox{cns}{colback=consectary-color,grow to right by=-4mm,grow to left by=-4mm,
    boxrule=0pt,boxsep=0pt,breakable} % настройки области с изменённым фоном



\usepackage{fancybox,fancyhdr}
\pagestyle{fancy}

\usepackage{hyperref}
\hypersetup{colorlinks=true, allcolors=[RGB]{010 090 200}} % цвет ссылок 
\newcommand{\lr}[1]{\left({#1}\right)} % команда для скобок

\author{Васильев Павел}
%\linespread{1}
\usepackage{amsmath}

\usepackage{graphicx}
\usepackage{ifpdf}
\ifpdf
\DeclareGraphicsRule{*}{mps}{*}{}
\fi
\usepackage{graphicx}
\usepackage{color}
\graphicspath{ {images/} }

%\renewcommand{\familydefault}{\sfdefault}

\begin{document}

\section*{HOW TO заботать экзамен по матанализу 2}

\textbf{Список билетов}
\begin{enumerate}
\item Критерий Коши для несобственных интегралов
\item Признаки сравнения для несобственных интегралов
\item Признаки Дирихле и Абеля сходимости несобственных интегралов
\item Преобразование Абеля. Признаки Дирихле и Абеля сходимости знакопеременных рядов
\item Перестановка членов в абсолютно сходящихся рядах
\item Теорема Римана
\item Критерий Коши равномерной сходимости функциональной последовательности и функционального ряда
\item Признак Вейерштрасса равномерной сходимости функционального ряда
\item Признаки Абеля и Дирихле равномерной сходимости функционального ряда 
\item Теорема о предельном переходе в функциональных последовательностях и функциональных рядах 
\item Теорема о почленном интегрировании функциональной последовательности и функционального ряда
\item Теорема о почленном дифференцировании функциональной последовательности и функционального ряда
\item Первая и вторая теоремы Абеля для степенных рядов и следствия
\item Теорема Коши-Адамара
\item Бесконечная дифференцируемость степенного ряда
\item Теорема Вейерштрасса о приближении непрерывной на отрезке функции многочленом

\end{enumerate} 

\end{document}
