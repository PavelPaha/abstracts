\documentclass[a4paper]{article}
\usepackage[utf8]{inputenc}
\usepackage[T2A]{fontenc}
\usepackage[12pt]{extsizes}
\usepackage[normalem]{ulem}

\usepackage[english,russian]{babel}
\usepackage[left=10mm, top=10mm, right=10mm, bottom=20mm, nohead, nofoot]{geometry}
\usepackage{amsmath,amsfonts,amssymb} % математический пакет
\headsep=10mm

\usepackage[most]{tcolorbox} % для управления цветом
% НАСТРОЙКИ
%теорема
\definecolor{theorem-color}{gray}{0.90} % уровень прозрачности (1 - максимум)
\newtcolorbox{htheorem}{colback=theorem-color,grow to right by=-4mm,grow to left by=-4mm,
    boxrule=0pt,boxsep=0pt,breakable} % настройки области с изменённым фоном

%определение
\definecolor{def-color}{gray}{0.98}
\newtcolorbox{definit}{colback=def-color,grow to right by=-4mm,grow to left by=-4mm,
    boxrule=0pt,boxsep=0pt,breakable} % настройки области с изменённым фоном

%доказательсвто теоремы
\definecolor{proof-color}{gray}{0.95} % уровень прозрачности (1 - максимум)
\newtcolorbox{hproof}{colback=proof-color,grow to right by=-1mm,grow to left by=-1mm,
    boxrule=0pt,boxsep=0pt,breakable} % настройки области с изменённым фоном

%замечания, следствия
\definecolor{consectary-color}{gray}{0.95} % уровень прозрачности (1 - максимум)
\newtcolorbox{cns}{colback=consectary-color,grow to right by=-4mm,grow to left by=-4mm,
    boxrule=0pt,boxsep=0pt,breakable} % настройки области с изменённым фоном

\everymath{\displaystyle}


\usepackage{fancybox,fancyhdr}
\pagestyle{fancy}
\fancyhf{}
\fancyhead[R]{ФТ-104}
\fancyfoot[R]{\thepage}
\fancyhead[L]{колок матан 2 семестр}

\usepackage{hyperref}
\hypersetup{colorlinks=true, allcolors=[RGB]{010 090 200}} % цвет ссылок 
\newcommand{\lr}[1]{\left({#1}\right)} % команда для скобок

\title{Конспектик к коллоквиуму по матанализу}
\author{Васильев Павел}
%\linespread{1}
\usepackage{amsmath}

\usepackage{graphicx}
\usepackage{ifpdf}
\ifpdf
\DeclareGraphicsRule{*}{mps}{*}{}
\fi
\usepackage{graphicx}
\usepackage{color}
\graphicspath{ {images/} }

%\renewcommand{\familydefault}{\sfdefault}

\begin{document}

\section*{HOW TO заботать коллоквиум по матанализу (2 семестр)}


\subsection*{Список билетов}
\begin{enumerate}
\item \sout{Понятие определённого интеграла}
\item \sout{Интегрируемость суммы функций}
\item \sout{Ограниченность интегрируемой функции}
\item \sout{Пример ограниченной неинтегрируемой функции}
\item Суммы Дарбу и их свойства. Критерий интегрируемости
\item \sout{Аддитивность интеграла по множеству}
\item \sout{Интегрируемость непрерывной функции}
\item \sout{Интегрируемость монотонной функции}
\item \sout{Неизменность интеграла при изменении функции в конечном числе точек}
\item Интегрируемость композиции непрерывной и интегрируемой функций
\item Интегрируемость произведения функций
\item \sout{Интеграл с переменным верхним пределом: непрерывность и дифференцируемость}
\item \sout{Формула Ньютона-Лейбница}
\item \sout{Пример неинтегрируемой функции с первообразной}
\item \sout{Пример интегрируемой функции без первообразной}
\item \sout{Интегрирование по частям}
\item \sout{Замена переменной}
\item \sout{Первая теорема о среднем}
\item Вычисление площадей
\item Вычисление длины дуги
\item Приближённое вычисление интеграла: методы прямоугольников, трапеций, Симпсона 
\end{enumerate}

\newpage

\begin{definit}
\subsection*{Понятие определённого интеграла}

\textit{Разбиение} отрезка $[a, b]$ - $\{ a = x_0, x_1, ..., x_n = b\}$, $x_i < x_{i+1}$

\textit{Мелкость} разбиения: $$\lambda = \max_{1 \leq k \leq n} \Delta x_k = \max (x_k - x_{k-1})$$

\textit{Интегральная сумма}: $S(f, \tau, \xi) = \sum_{k=1}^n f(\xi_k) \Delta x_k = S_\tau$

\textbf{Определение} $f$, определённая на $[a,b]$, интегрируема по Риману на $[a,b]$, если

$$\exists I \in \mathbb{R}: \quad \forall \varepsilon > 0 \exists \delta (\varepsilon) \quad \forall \tau  \forall \xi \quad (\lambda_t < \delta \Rightarrow |S(f, \tau, \xi) - I| < \varepsilon)$$

Обозначаем $I = \int_a^b f(x)dx$


\end{definit}

\begin{definit}
\subsection*{Интегрируемость суммы функций}
\begin{htheorem}\textbf{Теорема}.
Пусть $f$ и $g$ интегрируемы на $[a,b]$ и $\alpha, \beta \in \mathbb{R}$. Тогда $\alpha f + \beta g$ интегрируема на $[a,b]$ и $$\int_a^b(\alpha f + \beta g)dx = \alpha \int_a^b fdx + \beta \int_a^b gdx$$
\end{htheorem}

\begin{hproof}\textbf{Доказательство.}
$$S(\alpha f + \beta g, \tau, \xi) = \sum_{k=1}^n (\alpha f(\xi_k) + \beta g(\xi_k)) \Delta x_k = \newline \alpha \sum_{k=1}^n f(\xi_k) \Delta x_k + \beta \sum_{k=1}^n g(\xi_k) \Delta x_k = $$ 

$$= \alpha S(f, \tau, \xi) + \beta S(g, \tau, \xi)$$


$$ \left| S(\alpha f + \beta g, \tau, \xi) - \left( \alpha \int_a^b fdx + \beta \int_a^b gdx \right) \right| \leq |\alpha| \left| S(f, \tau, \xi) - \int_a^b f \right| + |\beta| \left| S(g, \tau, \xi) - \int_a^b g  \right|$$

В определении интегрируемости $f$ и $g$ берём не $\varepsilon$, а $\frac{\varepsilon}{|\alpha|+|\beta|}$

Тогда $$ \left| S(f, \tau, \xi) - \int_a^b f \right| \leq \frac{\varepsilon}{|\alpha|+|\beta|}\quad \text{и} \quad \left| S(g, \tau, \xi) - \int_a^b g  \right| \leq \frac{\varepsilon}{|\alpha|+|\beta|} \Rightarrow$$

$$
\Rightarrow |\alpha| \left| S(f, \tau, \xi) - \int_a^b f \right| + |\beta| \left| S(g, \tau, \xi) - \int_a^b g  \right| < \varepsilon$$

\end{hproof}
\end{definit}

\begin{definit}
\subsection*{Ограниченность интегрируемой функции}

\begin{htheorem}\textbf{Теорема}.
Если $f$ интегрируема на $[a,b]$, то она ограничена на $[a,b]$.

\end{htheorem}

\begin{hproof}\textbf{Доказательство.}

Проведём от противного: пусть $f$ не ограничена, но интегрируема.

Тогда $I - \varepsilon < \sum_{k=1}^n f(\xi_k) \Delta x_k < I + \varepsilon$ для какого-то разбиения $\tau$ при заданном $\varepsilon > 0$ и любом выборе $\xi_k \in [x_{k-1}, x_k]$.

Так как $f$ не ограничена, то найдётся такой отрезок $[x_{k-1}, x_k]$, на котором $f$ не ограничена $\Rightarrow \sum_{k=1}^n f(\xi_k) \Delta x_k $ можно сделать сколь угодно большим по модулю. Противоречие.
\end{hproof}

\end{definit}


\begin{definit}
\subsection*{Пример ограниченной неинтегрируемой функции}

Например, функция Дирихле $D(x)$:

\[
	D(x) = \begin{cases}
		1, x \in \mathbb{Q} \\
		0, x \notin \mathbb{Q}
	\end{cases}
\]

\[
	\forall x \in [a,b] \quad D(x) \quad \text{не интегрируема на} \quad  [a, b], \quad \text{так как}
\]

\begin{enumerate}
\item $\{ \xi_k \} \subset \mathbb{R} \backslash \mathbb{Q}, f(\xi_k) = 0 \Rightarrow \forall \tau S_\tau = 0$
\item $\{ \xi_k \} \subset \mathbb{Q}, f(\xi_k) = 1 \Rightarrow \forall \tau S_\tau = \sum \Delta x_k = b - a$
\end{enumerate}
\end{definit}



\begin{definit}
\subsection*{Аддитивность интеграла по множеству}


Пусть $c \in (a, b)$ и функция $f(x)$ определена и интегрируема на отрезке $[a, b]$. Тогда интеграл функции $f(x)$ на отрезке $[a, b]$ равен сумме интегралов функции $f(x)$ на отрезках $[a, c]$ и $[c, b]$:

\[
\int_{a}^{b} f(x) \, dx = \int_{a}^{c} f(x) \, dx + \int_{c}^{b} f(x) \, dx
\]

\textbf{Доказательство:}

Поскольку $f(x)$ интегрируема на отрезке $[a, b]$, для любого $\varepsilon > 0$ существует разбиение $\tau = \{a, x_1, \dots, x_{n-1}, b\}$ отрезка $[a, b]$ такое, что верхняя сумма Дарбу $\overline{S_\tau}$ и нижняя сумма Дарбу $\underline{S_\tau}$ удовлетворяют условию:

\[
\overline{S_\tau} - \underline{S_\tau} < \varepsilon
\]

Выберем такое разбиение, которое включает точку $c$. Теперь разбиваем $\tau$ на два подмножества $\tau'$ и $\tau''$, соответствующие отрезкам $[a, c]$ и $[c, b]$, так что $\tau = \{x_i \in \tau \mid x_i \leq c\}$ и $\tau'' = \{x_i \in \tau \mid x_i \geq c\}$.

Тогда верхние и нижние суммы Дарбу для $f(x)$ на отрезках $[a, c]$ и $[c, b]$ будут равны $\overline{S_{\tau'}})$ и $\underline{S_{\tau'}}$, а также $\overline{S_{\tau''}})$ и $\underline{S_{\tau'}})$ соответственно.

Поскольку разбиение $\tau$ является объединением $\tau'$ и $\tau''$, имеем:

\[
\overline{S_{\tau}} = \overline{S_{\tau'}} + \overline{S_{\tau''}}
\]
\[
\underline{S_{\tau}} = \underline{S_{\tau'}} + \underline{S_{\tau''}}
\]

Из этого следует, что:

\[
(\overline{S_{\tau'}} + \overline{S_{\tau''}} - (\underline{S_{\tau'}} + \underline{S_{\tau''}}) < \varepsilon
\]

\[
(\overline{S_{\tau'}} - \underline{S_{\tau'}}) + (\overline{S_{\tau''}} - \underline{S_{\tau''}}) < \varepsilon
\]

\[
\overline{S_{\tau'}} - \underline{S_{\tau'}} < \varepsilon
\]

\[
\overline{S_{\tau''}} - \underline{S_{\tau''}} < \varepsilon
\]



Докажем уже наконец-то, что $\displaystyle \int_a^b f = \int_a^c f + \int_c^b f $

$\displaystyle \left| \int_a^b - \left( \int_a^c + \int_c^b \right) \right| < \varepsilon$

$\displaystyle \left| \int_a^b - \left( \int_a^c + \int_c^b \right) + \left( S_{\tau'} + S_{\tau''} \right) - (S_{\tau'} + S_{\tau''}) \right| < \varepsilon$

$\displaystyle \left| \int_a^b - \left( \int_a^c + \int_c^b \right) + \left( S_{\tau'} + S_{\tau''} \right) - (S_{\tau'} + S_{\tau''}) \right| \leq \left| \int_a^b -  (S_{\tau'} + S_{\tau''}) \right| + \left| \int_a^c - S_{\tau'} \right| + \left| \int_c^b - S_{\tau''} \right| <\varepsilon$

$\left| \int_a^b - ( S_{\tau'} + S_{\tau''}) \right| < \frac{\varepsilon}{3}$

$ \left| \int_a^c - S_{\tau'} \right| < \frac{\varepsilon}{3}$

$\left| \int_c^b - S_{\tau''} \right| < \frac{\varepsilon}{3}$

\end{definit}

\begin{definit}
\subsection*{Интегрируемость непрерывной функции}

\begin{htheorem}\textbf{Теорема}.

Пусть $f$ непрерывна на $[a,b]$, тогда $f$ интегрируема на $[a,b]$.
\end{htheorem}

\begin{hproof}\textbf{Доказательство.}

$f$ непрерывна на $[a,b]$, значит она равномерно нерпрерывна на $[a,b]$ (теорема Кантора):

\[
\forall \varepsilon > 0 \quad \exists \delta ( \varepsilon ) \quad \forall x', x'' \in [a,b] \quad |x'-x''| < \delta \Rightarrow |f(x') - f(x'')| < \varepsilon
\]

Теперь поймём, что 

\[
\forall \varepsilon' > 0 \quad \exists \delta'(\varepsilon) > 0 \quad \forall \tau \quad (\lambda_\tau < \delta' \Rightarrow \overline{S_\tau} - \underline{S_\tau} < \varepsilon')
\]

Рассмотрим разбиение $\tau$, у которого $\lambda_\tau < \delta$ ($\delta$ берём из определения равномерной непрерывности).

\[
\overline{S_\tau} - \underline{S_\tau} = \sum_{k=1}^n (M_k-m_k) \Delta x_k
\]

\[
\forall \xi_j', \xi_j'' \in [x_j, x_{j+1}] \quad |\xi_j' - \xi_j''| < \delta, \quad \text{так как} \quad \lambda_\tau < \delta
\]

И по равномерной непрерывности получаем

\[
|f(\xi_j)' - f(\xi_j'')| < \varepsilon
\]

\[
|f(\xi_j)' - f(\xi_j'')| < \varepsilon \Leftrightarrow M_j - m_j < \varepsilon
\]

Докажем это утверждение:
\begin{enumerate}
\item $\Leftarrow$:

\[
M_j - m_j < \varepsilon
\]

\[
\begin{cases}
f(\xi_j') \leq M_j \\
f(\xi_j'') \geq m_j \\
\end{cases}
\Rightarrow f(\xi_j') - f(\xi_j'') \leq M_j-m_j < \varepsilon
\]

\item $\Rightarrow$:

Знаем $|f(\xi_j)' - f(\xi_j'')| < \varepsilon$.

Возьмём $\sup$ по $\xi_j'$:

\[
	\sup_{\xi_j'}(f(\xi_j') - f(\xi_j'')) = M_j - f(\xi_j'') \leq \varepsilon
\]

А затем возьмём $\inf$ по $\xi_j''$ и получим:

\[
M_j - m_j < \varepsilon
\]

\end{enumerate}

Мы доказали 

\[
|f(\xi_j)' - f(\xi_j'')| < \varepsilon \Leftrightarrow M_j - m_j < \varepsilon
\]

\[
\overline{S_\tau} - \underline{S_\tau} = \sum_{k=1}^n (M_k-m_k) \Delta x_k < \varepsilon \sum_{k=1}^n \Delta x_k = \varepsilon (b-a)
\]
\end{hproof}


\end{definit}


\begin{definit}
\subsection*{Интегрируемость монотонной функции}

\begin{htheorem}\textbf{Теорема}.

Пусть $f$ монотонна на $[a,b]$. Тогда $f$ интегрируема $[a,b]$.
\end{htheorem}

\begin{hproof}\textbf{Доказательство.}

Б.о.о. скажем, что $f$ монотонно возрастает.

Определение интегрируемости:

\[
\forall \varepsilon > 0 \quad \exists \delta(\varepsilon) > 0 \quad \forall \tau \quad (\lambda_\tau < \delta \Rightarrow \overline{S_\tau} - \underline{S_\tau} < \varepsilon)
\]

\[
\overline{S_\tau} - \underline{S_\tau} = \sum_{k=1}^n (M_k-m_k) \Delta x_k = \sum_{k=1}^n (f(x_k) - f(x_{k-1})) \Delta x_k \leq \delta \sum_{k=1}^n (f(x_k) - f(x_{k-1})
\]

\[
\sum_{k=1}^n (f(x_k) - f(x_{k-1})) = \delta (f(b)-f(a))
\]

Ну и чтобы $\delta (f(b)-f(a))$ было меньше $\varepsilon$, возьмём $\delta < \frac{\varepsilon}{f(b)-f(a)}$.
\end{hproof}
\end{definit}

\begin{definit}
\subsection*{Неизменность интеграла при изменении функции в конечном числе точек}
\begin{htheorem}\textbf{Теорема}.

Пусть $f$ интегрируема $[a,b]$. Тогда если поменяем $f$ в конечном числе точек, то площадь останется неизменной.
\end{htheorem}


\begin{hproof}\textbf{Доказательство.}

Пусть $\breve{f}$ это $f$, у которой поменяли $f(x_0)$ на $c$

\[
g(x) = \begin{cases} 0, x \neq x_0 \\ f(x_0) - c, x=x_o  \end{cases}
\]

\[
\int_a^b g(x)dx = 0
\]

\[
\forall \tau |s_\tau| \leq |x - f(x_0)| \cdot \lambda_\tau \rightarrow 0 \Rightarrow \int_a^b g(x)dx = 0
\]

\[
\int_a^b f = \int_a^b (\breve{f} + g) = \int_a^b \breve{f} + 0 =  \int_a^b \breve{f}
\]
\end{hproof}
\end{definit}

\begin{definit}
\subsection*{Интеграл с переменным верхним пределом: непрерывность и дифференцируемость}

\textbf{Определение.} Пусть $f$ интегрируема на $[a,b]$ ($\Rightarrow \forall x \in (a,b) \quad \exists \int_a^x f(t)dt$.

$\Phi(x) = \int_a^x f(t)dt$ - интеграл с переменным верхним пределом (договоримся, что $\int_a^a = 0$).

\begin{htheorem}\textbf{Теорема}.

Пусть $f$ ограничена на $[a,b]$. Тогда $\Phi$ непрерывна и $\exists C: \quad |\Phi(x) - \Phi(y)| \leq С|x-y| \quad \forall x, y \in (a,b)$ (\textit{липшицевость}).
\end{htheorem}

\begin{hproof}\textbf{Доказательство.}
Из липшицевости следует непрерывность по определению непрерывности ($\delta = \frac{\varepsilon}{C}$) (типа множество липшицевостных функций является подмножеством непрерывных функций)


Предположим $x<y$:

\[
	|\Phi(x) - \Phi(y)| = \left| \int_a^x f(t)dt - \int_a^y dt \right| = \left| \int_x^y f(t)dt \right| \leq \begin{bmatrix}
	\left| \sum f(\xi_k) \Delta x_k \right| \leq \sum \left| f(\xi_k) \right| \Delta x_k \\
	\left| \sum f(\xi_k) \Delta x_k \right| \rightarrow |\int f| \\
	 \sum \left| f(\xi_k) \right| \Delta x_k \rightarrow |\int f|
	\end{bmatrix} \leq
\]
 
\[ \leq
 \begin{bmatrix}
 |f| \leq B
 \end{bmatrix} \leq B \cdot \left| \int_x^y dt \right| \leq B(y-x)
\]
\end{hproof}


\end{definit}

\begin{definit}
\subsection*{Формула Ньютона-Лейбница}

\begin{htheorem}\textbf{Теорема}.

Пусть $f$ интегрируема на $[a,b]$ и имеет на $[a,b]$ первообразную $F$. Тогда $\int_a^b f(x)dx = F(b) - F(a)$
\end{htheorem}

\begin{hproof}\textbf{Доказательство.}

Рассмотрим равномерное разбиение $[a,b]$ (на $n$ равныйх частей). Тогда $\frac{b-a}{n}$ - длина отрезка разбиения.

\[
F(b) - F(a) = \sum_{k=1}^n (F(x_k) - F(x_{k-1})) = \begin{bmatrix}
\text{теорема Лагранжа} \\
\exists \xi_k \in [x_k, x_{k-1}]: \\
F(x_k)-F(x_{k-1} = F'(\xi_k)(x_k-x_{k-1}) = \\ = f(\xi_k) \Delta x_k
\end{bmatrix} = 
\]

\[
 = \sum_{k=1}^n f(\xi_k) \Delta x_k = \sum_{k=1}^n f(\xi_k) \frac{b-a}{n}
\]
\[
\lim_{n \rightarrow \infty} (F(b)-F(a)) = \lim_{n \rightarrow \infty} \sum_{k=1}^n f(\xi_k) \frac{b-a}{n} = \int_a^b f(x)dx
\]
\end{hproof}

\end{definit}

\begin{definit}
\subsection*{Пример неинтегрируемой функции с первообразной}

\[
F(x) = x^2 \sin \left( \frac{1}{x^2} \right), x \in (0,1]
\]

\[
f = F'(x) = 2x \sin \left( \frac{1}{x^2} \right) - \frac{2}{x} \cos \left( \frac{1}{x^2} \right)
\]

$f$ неинтегрируема на $(0,1]$, потому что не ограничена на этом множестве.
\end{definit}


\begin{definit}
\subsection*{Пример интегрируемой функции без первообразной}

\[
f(x) = sgn x, x \in [-1,1]
\]

\[
\int_{-1}^1 sgn x = 0
\]

\[
F(x) = \begin{cases}
x+C_1, x>0\\
-x+C_2, x<0\\
\end{cases}
\]

Добьёмся того, чтобы в нуле первообразная была непрерывна. Тогда $C_1 = C_2 = C$. Тогда первообразная представляет собой функцию $|x|+c$, которая, конечно, не дифферинцируема в нуле.
\end{definit}


\begin{definit}
\subsection*{Интегрирование по частям}

\begin{htheorem}\textbf{Теорема}.

Пусть $u$ и $v$ непрерывны и кусочно-непрерывно дифференцируемы на $[a,b]$. Тогда

\[
\int_a^b udv = uv |_a^b - \int_a^b vdu
\]

\[
uv |_a^b = u(b)v(b) - u(a)v(a)
\]
\end{htheorem}

\begin{hproof}\textbf{Доказательство.}

По условиям теоремы, оба интеграла существуют как интегралы от кусочно непрерывных функций.

\[
(uv)' = u'v + uv'
\]

За исключением конечного числа точек, так как кусочная дифференцируемость.

\[
\int_a^b (uv)' = \int_a^b (u'v) + \int_a^b (uv')
\]
\end{hproof}
\end{definit}

\begin{definit}
\subsection*{Замена переменной}

\begin{htheorem}\textbf{Теорема}.

Пусть $f$ непрерывна на $[x_1, x_2]$, а $g$ непрерывно дифференцируема на $[t_1, t_2]$ и $g(t_1) = x_1, g(t_2) = x_2$ и $g(t) \in [x_1, x_2], t \in [t_1, t_2]$. Тогда

\[
\int_{x_1}^{x_2} f(x)dx = \int_{t_1}^{t_2} f(g(t)) d'(t)dt
\]
\end{htheorem}

\begin{hproof}\textbf{Доказательство.}

$f$ непрерывна. А если $f$ непрерывна, то существует первообразная $F$.

По теореме (?) $\Phi = \int_{x_1}^x f(t)dt$ - дифференцируема и $\Phi'(x) = f(x)$. Тогда $\Phi(x)$ - первообразная $\Rightarrow \int_{x_1}^{x_2} f(x)dx = F(x_2) - F(x_1)$.

Рассмотрим $F(g(t)), t \in [t_1, t_2]$.

\[
F'(g(t)) = F'(g(t))g'(t) = f(g(t))g'(t)
\]

$F(g(t))$ - первообразная для $f(g(t))g'(t) \Rightarrow \int_{t_1}^{t_2} f(g(t))g'(t)dt = F(g(t_2)) - F(g(t_1)) = F(x_2) - F(x_1)$

Итого $F(x_2) - F(x_1) = F(x_2) - F(x_1)$. Доказано.
\end{hproof}

\begin{htheorem}\textbf{Теорема}.
Пусть $f$ непрерывна на $[a,b]$. Тогда $\forall x_0 \in (a,b) \quad \Phi'(x_0) = f(x_0)$

\end{htheorem}

\begin{hproof}\textbf{Доказательство.}

\[
\Phi'(x_0) = \lim_{h \rightarrow 0} \frac{\Phi(x_0 + h) - \Phi (x_0)}{h}
\]

\[
\frac{\Phi(x_0 + h) - \Phi (x_0)}{h} = \frac{\int_{x_0}^{x_0+h} f(t)dt}{h} = \frac{\int_{x_0}^{x_0+h} f(t)dt - f(x_0)h}{h} + f(x_0) = 
\]

\[
 = \int_{x_0}^{x_0+h} \frac{(f(t_-f(x_0))dt}{h} + f(x_0)
\]

\[
\left| \frac{\Phi(x_0+h)-\Phi(x_0)}{h} - f(x_0) \right| = \left| \int_{x_0}^{x_0+h} \frac{(f(t_-f(x_0)dt}{h} \right| \leq 
\]

\[
 \leq \begin{bmatrix}
 \text{О, а ведь} \quad f \quad \text{непрерывна}: \forall \varepsilon > 0 \quad \exists \delta: \quad 0 < |t-x_0| < \delta \Rightarrow |f(t)-f(x_0)| < \varepsilon \\
 \text{Берём любой} \quad \varepsilon \quad \text{и по нему находим} \quad \delta \quad \text{и берём} \quad h < \delta 
 \end{bmatrix} \leq 
\]

\[
\leq \frac{1}{h} \int_{x_0}^{x_0+h} |f(t) - f(x_0)| dt \leq \frac{1}{h} \int_{x_0}^{x_0+h} \varepsilon dt = \varepsilon
\]
\end{hproof}
\end{definit}

\begin{definit}
\subsection*{Первая теорема о среднем}

\begin{htheorem}\textbf{Теорема}.

Пусть $f$ интегрируема на $[a,b]$, $\Phi$ - весовая функция ($\geq 0$ и интегрируемая) \newline и $m \leq f \leq M$ на $[a,b]$. Тогда существует $\mu \in [m, M]: \quad \mu \int_a^b \Phi = \int_a^b \Phi f$

\textbf{Замечание.}

В частности, если $f$ непрерывная, то она достигает $min = m$ и $max = M$ на $[a,b]$ и по теореме Коши о промежуточных значениях

\[
\forall \mu \in [m,M] \quad \exists x_0 \in [a,b]: \quad \mu = f(x_0)
\]

\textbf{Замечание.}

Важно, чтобы $\Phi$ была знакопостоянной. Контрпример - $f = x, \Phi = sgn x$ на $[-1,1]$.

$\int_{-1}^1 x sgn x fx = 1$

$\mu \int_{-1}^1 sgn x dx = 0 \quad \forall \mu$
\end{htheorem}

\begin{hproof}\textbf{Доказательство.}

Рассмотрим 2 случая:
\begin{enumerate}
\item $\int_a^b \Phi = 0 \Rightarrow  m \int_a^b \Phi \leq \int_a^b f \Phi \leq M \int_a^b \Phi$
\[
\begin{cases}
m \int_a^b \Phi = 0 \\ M \int_a^b \Phi = 0
\end{cases} \Rightarrow \int_a^b f \Phi = 0
\]

\item $\int_a^b \Phi \neq 0$

Поделим неравенство из предыдущего пункта на $\int_a^b \Phi$:
\[
\Rightarrow  m \leq \frac{\int_a^b f \Phi}{\int_a^b \Phi} \leq M
\]


\end{enumerate}
\end{hproof}

\end{definit}

\begin{definit}
\subsection*{Вторая теорема о среднем}
Пусть на $[a,b]$ функция $f$ монотонна (б.о.о. убывает) и $\Phi$ интегрируема на $[a,b]$. Тогда

\[
\exists \xi \in [a,b]: \quad \int_a^b \Phi f = f(a) \int_a^{\xi} \Phi(x)dx + f(b) \int_{\xi}^b\Phi(x)dx
\]

Эта теорема без доказательства (а почему не знаю).
\end{definit}

\begin{definit}

\subsection*{Вычисление длины дуги}

\textbf{Определение.} \textit{Кривой} называется непрерывное отображение отрезка на плоскость.

\textbf{Определение.}  Кривая $L$ называется \textit{спрямляемой}, если множество длин вписанных в неё ломаных $l$ ограничено сверху.
\end{definit}







\end{document}
