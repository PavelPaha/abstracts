\documentclass[a4paper]{article}
\usepackage[utf8]{inputenc}
\usepackage[T2A]{fontenc}
\usepackage[12pt]{extsizes}
\usepackage[normalem]{ulem}

\usepackage[english,russian]{babel}
\usepackage[left=10mm, top=10mm, right=10mm, bottom=20mm, nohead, nofoot]{geometry}
\usepackage{amsmath,amsfonts,amssymb} % математический пакет
\headsep=10mm

\usepackage[most]{tcolorbox} % для управления цветом
% НАСТРОЙКИ
%теорема
\definecolor{theorem-color}{gray}{0.90} % уровень прозрачности (1 - максимум)
\newtcolorbox{htheorem}{colback=theorem-color,grow to right by=-4mm,grow to left by=-4mm,
    boxrule=0pt,boxsep=0pt,breakable} % настройки области с изменённым фоном

%определение
\definecolor{def-color}{gray}{0.98}
\newtcolorbox{definit}{colback=def-color,grow to right by=-4mm,grow to left by=-4mm,
    boxrule=0pt,boxsep=0pt,breakable} % настройки области с изменённым фоном

%доказательсвто теоремы
\definecolor{proof-color}{gray}{0.95} % уровень прозрачности (1 - максимум)
\newtcolorbox{hproof}{colback=proof-color,grow to right by=-1mm,grow to left by=-1mm,
    boxrule=0pt,boxsep=0pt,breakable} % настройки области с изменённым фоном

%замечания, следствия
\definecolor{consectary-color}{gray}{0.95} % уровень прозрачности (1 - максимум)
\newtcolorbox{cns}{colback=consectary-color,grow to right by=-4mm,grow to left by=-4mm,
    boxrule=0pt,boxsep=0pt,breakable} % настройки области с изменённым фоном

\everymath{\displaystyle}


\usepackage{fancybox,fancyhdr}
\pagestyle{fancy}
\fancyhf{}
\fancyhead[R]{ФТ-104}
\fancyfoot[R]{\thepage}
\fancyhead[L]{колок матан 2 семестр}

\usepackage{hyperref}
\hypersetup{colorlinks=true, allcolors=[RGB]{010 090 200}} % цвет ссылок 
\newcommand{\lr}[1]{\left({#1}\right)} % команда для скобок

\title{Конспектик к коллоквиуму по матанализу}
\author{Васильев Павел}
%\linespread{1}
\usepackage{amsmath}

\usepackage{graphicx}
\usepackage{ifpdf}
\ifpdf
\DeclareGraphicsRule{*}{mps}{*}{}
\fi
\usepackage{graphicx}
\usepackage{color}
\graphicspath{ {images/} }

%\renewcommand{\familydefault}{\sfdefault}

\begin{document}

\section*{HOW TO заботать коллоквиум по матанализу (2 семестр)}


\subsection*{Список билетов}
\begin{enumerate}
\item \sout{Понятие определённого интеграла}
\item \sout{Интегрируемость суммы функций}
\item \sout{Ограниченность интегрируемой функции}
\item Пример ограниченной неинтегрируемой функции
\item Суммы Дарбу и их свойства. Критерий интегрируемости
\item Аддитивность интеграла по множеству
\item Интегрируемость непрерывной функции
\item Интегрируемость монотонной функции
\item Неизменность интеграла при изменении функции в конечном числе точек
\item Интегрируемость композиции непрерывной и интегрируемой функций
\item Интегрируемость произведения функций
\item Интеграл с переменным верхним пределом: непрерывность и дифференцируемость
\item Формула Ньютона-Лейбница
\item Пример неинтегрируемой функции с первообразной
\item Пример интегрируемой функции без первообразной
\item Интегрирование по частям
\item Замена переменной
\item Первая теорема о среднем
\item Вычисление площадей
\item Вычисление длины дуги
\item Приближённое вычисление интеграла: методы прямоугольников, трапеций, Симпсона 
\end{enumerate}

\begin{definit}
\subsection*{Понятие определённого интеграла}

\textit{Разбиение} отрезка $[a, b]$ - $\{ a = x_0, x_1, ..., x_n = b\}$, $x_i < x_{i+1}$

\textit{Мелкость} разбиения: $$\lambda = \max_{1 \leq k \leq n} \Delta x_k = \max (x_k - x_{k-1})$$

\textit{Интегральная сумма}: $S(f, \tau, \xi) = \sum_{k=1}^n f(\xi_k) \Delta x_k = S_\tau$

\textbf{Определение} $f$, определённая на $[a,b]$, интегрируема по Риману на $[a,b]$, если

$$\exists I \in \mathbb{R}: \quad \forall \epsilon > 0 \exists \delta (\epsilon) \quad \forall \tau  \forall \xi \quad (\lambda_t < \delta \Rightarrow |S(f, \tau, \xi) - I| < \epsilon)$$

Обозначаем $I = \int_a^b f(x)dx$


\end{definit}

\begin{definit}
\subsection*{Интегрируемость суммы функций}
\begin{htheorem}\textbf{Теорема}.
Пусть $f$ и $g$ интегрируемы на $[a,b]$ и $\alpha, \beta \in \mathbb{R}$. Тогда $\alpha f + \beta g$ интегрирума на $[a,b]$ и $$\int_a^b(\alpha f + \beta g)dx = \alpha \int_a^b fdx + \beta \int_a^b gdx$$
\end{htheorem}

\begin{hproof}\textbf{Доказательство.}
$$S(\alpha f + \beta g, \tau, \xi) = \sum_{k=1}^n (\alpha f(\xi_k) + \beta g(\xi_k)) \Delta x_k = \newline \alpha \sum_{k=1}^n f(\xi_k) \Delta x_k + \beta \sum_{k=1}^n g(\xi_k) \Delta x_k = $$ 

$$= \alpha S(f, \tau, \xi) + \beta S(g, \tau, \xi)$$


$$ \left| S(\alpha f + \beta g, \tau, \xi) - \left( \alpha \int_a^b fdx + \beta \int_a^b gdx \right) \right| \leq |\alpha| \left| S(f, \tau, \xi) - \int_a^b f \right| + |\beta| \left| S(g, \tau, \xi) - \int_a^b g  \right|$$

В определении интегрируемости $f$ и $g$ берём не $\epsilon$, а $\frac{\epsilon}{|\alpha|+|\beta|}$

Тогда $$ \left| S(f, \tau, \xi) - \int_a^b f \right| \leq \frac{\epsilon}{|\alpha|+|\beta|}\quad \text{и} \quad \left| S(g, \tau, \xi) - \int_a^b g  \right| \leq \frac{\epsilon}{|\alpha|+|\beta|} \Rightarrow$$

$$
\Rightarrow |\alpha| \left| S(f, \tau, \xi) - \int_a^b f \right| + |\beta| \left| S(g, \tau, \xi) - \int_a^b g  \right| < \epsilon$$

\end{hproof}
\end{definit}

\begin{definit}
\subsection*{Ограниченность интегрируемой функции}

\begin{htheorem}\textbf{Теорема}.
Если $f$ интегрируема на $[a,b]$, то она ограничена на $[a,b]$.

\end{htheorem}

\begin{hproof}\textbf{Доказательство.}

Проведём от противного: пусть $f$ не ограничена, но интегрируема.

Тогда $I - \epsilon < \sum_{k=1}^n f(\xi_k) \Delta x_k < I + \epsilon$ для какого-то разбиения $\tau$ при заданном $\epsilon > 0$ и любом выборе $\xi_k \in [x_{k-1}, x_k]$.

Так как $f$ не ограничена, то найдётся такой отрезок $[x_{k-1}, x_k]$, на котором $f$ не ограничена $\Rightarrow \sum_{k=1}^n f(\xi_k) \Delta x_k $ можно сделать сколь угодно большим по модулю. Противоречие.
\end{hproof}

\end{definit}

\end{document}
