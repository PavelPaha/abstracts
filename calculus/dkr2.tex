\documentclass[a4paper]{article}
\usepackage[utf8]{inputenc}
\usepackage[T2A]{fontenc}
\usepackage[12pt]{extsizes}

\usepackage[english,russian]{babel}
\usepackage[left=10mm, top=10mm, right=10mm, bottom=20mm, nohead, nofoot]{geometry}
\usepackage{amsmath,amsfonts,amssymb} % математический пакет
\headsep=10mm

\usepackage[most]{tcolorbox} % для управления цветом
% НАСТРОЙКИ
%теорема
\definecolor{theorem-color}{gray}{0.90} % уровень прозрачности (1 - максимум)
\newtcolorbox{htheorem}{colback=theorem-color,grow to right by=-4mm,grow to left by=-4mm,
    boxrule=0pt,boxsep=0pt,breakable} % настройки области с изменённым фоном

%определение
\definecolor{def-color}{gray}{0.98}
\newtcolorbox{definit}{colback=def-color,grow to right by=-4mm,grow to left by=-4mm,
    boxrule=0pt,boxsep=0pt,breakable} % настройки области с изменённым фоном

%доказательсвто теоремы
\definecolor{proof-color}{gray}{0.95} % уровень прозрачности (1 - максимум)
\newtcolorbox{hproof}{colback=proof-color,grow to right by=-1mm,grow to left by=-1mm,
    boxrule=0pt,boxsep=0pt,breakable} % настройки области с изменённым фоном

%замечания, следствия
\definecolor{consectary-color}{gray}{0.95} % уровень прозрачности (1 - максимум)
\newtcolorbox{cns}{colback=consectary-color,grow to right by=-4mm,grow to left by=-4mm,
    boxrule=0pt,boxsep=0pt,breakable} % настройки области с изменённым фоном



\usepackage{fancybox,fancyhdr}
\pagestyle{fancy}

\usepackage{hyperref}
\hypersetup{colorlinks=true, allcolors=[RGB]{010 090 200}} % цвет ссылок 
\newcommand{\lr}[1]{\left({#1}\right)} % команда для скобок

\author{Васильев Павел}
%\linespread{1}
\usepackage{amsmath}

\usepackage{graphicx}
\usepackage{ifpdf}
\ifpdf
\DeclareGraphicsRule{*}{mps}{*}{}
\fi
\usepackage{graphicx}
\usepackage{color}
\graphicspath{ {images/} }

%\renewcommand{\familydefault}{\sfdefault}

\begin{document}

\section*{Домашняя контрольная работа по матанализу 2}

\subsection*{1.$\displaystyle \lim_{n \rightarrow \infty} \sum_{k=1}^n \frac{\sqrt{k}}{n \sqrt{n} + k \sqrt{k}}$}

$\displaystyle \sum_{k=1}^n \frac{\sqrt{k}}{n \sqrt{n} + k \sqrt{k}} = \frac{1}{n} \sum_{k=1}^n \frac{n \sqrt{k}}{n \sqrt{n} + k \sqrt{k}} = \frac{\sqrt{k}}{\sqrt{n} + \frac{k \sqrt{k}}{n}} = \frac{1}{\frac{\sqrt{n}}{\sqrt{k}} + \frac{k}{n}} = \frac{1}{n} \sum_{k=1}^n \frac{1}{ \left( \frac{k}{n} \right)^{-\frac{1}{2}} + \frac{k}{n}} = \frac{1}{n} \sum_{k=1}^n f \left( \frac{k}{n} \right)$

\[ \displaystyle f(x) = \frac{1}{x^{-\frac{1}{2}} + x} = \frac{ \sqrt{x} }{1 + x \sqrt{x}}
\]

\[
\displaystyle \lim_{n \rightarrow \infty} \sum_{k=1}^n f \left(\frac{k}{n} \right) = \int_0^1 f(x) dx = \int_0^1 \frac{ \sqrt{x} }{1 + x \sqrt{x}} dx
\]

\[
\int \frac{ \sqrt{x} }{1 + x \sqrt{x}} dx = \begin{bmatrix}
t = \sqrt{x} \\
 dt = \frac{dx}{2t}
\end{bmatrix}
= \int \frac{2t^2}{1+t^3} dt = 2 \left( \int \frac{1}{3(t+1)} dt + \int \frac{2t-1}{3(1-t+t^2)} dt \right) =
\]

\[
= 2 \left( \frac{1}{3} \ln |t+1| +\int \frac{2t-1}{3(1-t+t^2)} dt \right) = \begin{bmatrix}
u = 1-t+t^2 \\ du = (2t-1)dt
\end{bmatrix} = 2 \left( \frac{1}{3} \ln |t+1| + \frac{1}{3} \ln|u| \right) =
\]

\[
= \frac{2}{3} \left(\ln |t+1| + \ln|1-t+t^2| \right) = 
\]

\[
= \frac{2}{3} \left( \ln |\sqrt{x}+1| + \ln|1-\sqrt{x}+x| \right)
\]

\[
\int_0^1 f(x)dx = \frac{2}{3} \left( \ln |\sqrt{x}+1| + \ln|1-\sqrt{x}+x| \right) \bigg|_0^1 = \frac{2}{3} \left( \ln 2 + \ln 1 - \ln 1 - \ln 1 \right) = \frac{2 \ln 2}{3}
\]

\textit{Ответ: $\frac{2 \ln 2}{3}$}

\subsection*{2. $y = \sqrt{4-x^2}, y=0, 0 \leq x \leq \frac{\pi}{2}$}

\[ \int \sqrt{4-x^2} dx  = \begin{bmatrix}
x = 2\sin t \\ dx = 2 \cos t dt \\ t = \arcsin(\frac{x}{2})
\end{bmatrix} = 2 \int \cos^2 t dt = 4 \int \frac{1+\cos 2t}{2} dt = \sin 2t + 2t
\] 

\[\int_0^\frac{\pi}{2} \sqrt{4-x^2} dx = (\sin 2t + 2t) = 
\] 

\[ =
\left( \sin \left( 2 \arcsin \left( \frac{x}{2}\right) \right) + 2 \arcsin \left( \frac{x}{2} \right) = 2 \sin \left( \arcsin \left( \frac{x}{2} \right) \right) \cos \left( \arcsin \left( \frac{x}{2}\right) \right) + 2 \arcsin \left( \frac{x}{2} \right) \right) \bigg|_0^\frac{\pi}{2} = 
\]

\[
= \left( 2 \frac{x}{2} \sqrt{1-\frac{x^2}{4}} + 2 \arcsin \left( \frac{x}{2} \right) \right) \bigg|_0^\frac{\pi}{2}
= \left( \frac{x}{2} \sqrt{4-x^2} + 2 \arcsin \left( \frac{x}{2} \right) \right) \bigg|_0^\frac{\pi}{2}
=
\]

\[
=
2 \arcsin \left( \frac{\pi}{4} \right) + \frac{\pi}{4} \sqrt{4 - \frac{\pi^2}{4}}
\]

\textit{Ответ: $2 \arcsin \left( \frac{\pi}{4} \right) + \frac{\pi}{4} \sqrt{4 - \frac{\pi^2}{4}}$}
\end{document}
