\documentclass[a4paper]{article}
\usepackage[utf8]{inputenc}
\usepackage[T2A]{fontenc}
\usepackage[12pt]{extsizes}
\everymath{\displaystyle}

\usepackage[english,russian]{babel}
\usepackage[left=10mm, top=10mm, right=10mm, bottom=20mm, nohead, nofoot]{geometry}
\usepackage{amsmath,amsfonts,amssymb} % математический пакет
\headsep=10mm

\usepackage[most]{tcolorbox} % для управления цветом
% НАСТРОЙКИ
%теорема
\definecolor{theorem-color}{gray}{0.90} % уровень прозрачности (1 - максимум)
\newtcolorbox{htheorem}{colback=theorem-color,grow to right by=-4mm,grow to left by=-4mm,
    boxrule=0pt,boxsep=0pt,breakable} % настройки области с изменённым фоном

%определение
\definecolor{def-color}{gray}{0.98}
\newtcolorbox{definit}{colback=def-color,grow to right by=-4mm,grow to left by=-4mm,
    boxrule=0pt,boxsep=0pt,breakable} % настройки области с изменённым фоном

%доказательсвто теоремы
\definecolor{proof-color}{gray}{0.95} % уровень прозрачности (1 - максимум)
\newtcolorbox{hproof}{colback=proof-color,grow to right by=-1mm,grow to left by=-1mm,
    boxrule=0pt,boxsep=0pt,breakable} % настройки области с изменённым фоном

%замечания, следствия
\definecolor{consectary-color}{gray}{0.95} % уровень прозрачности (1 - максимум)
\newtcolorbox{cns}{colback=consectary-color,grow to right by=-4mm,grow to left by=-4mm,
    boxrule=0pt,boxsep=0pt,breakable} % настройки области с изменённым фоном



\usepackage{fancybox,fancyhdr}
\pagestyle{fancy}

\usepackage{hyperref}
\hypersetup{colorlinks=true, allcolors=[RGB]{010 090 200}} % цвет ссылок 
\newcommand{\lr}[1]{\left({#1}\right)} % команда для скобок

\author{Васильев Павел}
%\linespread{1}
\usepackage{amsmath}

\usepackage{graphicx}
\usepackage{ifpdf}
\ifpdf
\DeclareGraphicsRule{*}{mps}{*}{}
\fi
\usepackage{graphicx}
\usepackage{color}
\graphicspath{ {images/} }


\renewcommand{\headrulewidth}{0pt}

%\renewcommand{\familydefault}{\sfdefault}

\begin{document}

\section*{ШАД 27 мая 2017}

\subsection*{Задание 1.}

За время обучения в ШАД Михаил 20 раз решал задачи классификации. В каждой задаче он использовал ансамбль из пяти различных классификаторов, причем никакую пару классификаторов он не применял более одного раза. Каково минимально возможное число известных Михаилу классификаторов?

\textbf{Ответ: 21}

\subsubsection*{Решение}


Раз в каждой взятой пятёрке пары различные, то в каждой взятой пятёрке $\frac{5 \cdot 4}{2}$ различных пар, а всего $20 \cdot \frac{5 \cdot 4}{2} = 200$ различных пар.

Пусть ответ равен $k$. Тогда у нас должно быть не меньше 200 различных пар среди этих $k$ классификаторов. А максимальное количество различных пар среди $k$ классификаторов равно $\frac{k \cdot (k-1)}{2}$.

\[ 
\frac{k \cdot (k-1)}{2} \geq 200
\]

$k \geq 21 $

\subsection*{Задание 2.}

Найдите сумму

\[ \sum_{k=0}^\infty (-1)^k \frac{(k+1)^2}{k!} \]


\textbf{Ответ: $-\frac{1}{e}$.}

\subsubsection*{Решение}


Что-то похоже на формулу Тейлора для $e^x$:

\[ e^x = \sum_{k=1}^\infty \frac{x^k}{k!} \]

\begin{equation}
\begin{gathered}
\sum_{k=0}^\infty (-1)^k \frac{(k+1)^2}{k!} = \sum_{k=0}^\infty (-1)^k \frac{k^2}{k!} + \sum_{k=0}^\infty (-1)^k \frac{2k}{k!} + \sum_{k=0}^\infty (-1)^k \frac{1}{k!} \\
\sum_{k=0}^\infty (-1)^k \frac{k^2}{k!} = \sum_{k=1}^\infty (-1)^k \frac{k}{(k-1)!} = \sum_{k=1}^\infty (-1)^k \frac{k-1+1}{(k-1)!} = \sum_{k=2}^\infty (-1)^k \frac{1}{(k-2)!} + \sum_{k=1}^\infty (-1)^k \frac{1}{(k-1)!} = \\
= (-1)^2 \sum_{m=0}^\infty (-1)^m \frac{1}{m!} + (-1)\sum_{m=0}^\infty (-1)^m \frac{1}{m!} = 0 \\
\sum_{k=0}^\infty (-1)^k \frac{2k}{k!} = 2\sum_{k=1}^\infty (-1)^k \frac{1}{(k-1)!} = -2\sum_{m=0}^\infty (-1)^m \frac{1}{m!} = -\frac{2}{e} \\
\sum_{k=0}^\infty (-1)^k \frac{1}{k!} = \frac{1}{e}
\end{gathered}
\end{equation}

Итого

\begin{equation}
\sum_{k=0}^\infty (-1)^k \frac{(k+1)^2}{k!} = - \frac{2}{e} + \frac{1}{e} = -\frac{1}{e}
\end{equation}


\end{document}
