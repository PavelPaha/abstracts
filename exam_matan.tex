\documentclass[a4paper]{article}
\usepackage[utf8]{inputenc}
\usepackage[T2A]{fontenc}
\usepackage[12pt]{extsizes}

\usepackage[english,russian]{babel}
\usepackage[left=10mm, top=10mm, right=10mm, bottom=20mm, nohead, nofoot]{geometry}
\usepackage{amsmath,amsfonts,amssymb} % математический пакет
\headsep=10mm

\usepackage[most]{tcolorbox} % для управления цветом
% НАСТРОЙКИ
%теорема
\definecolor{theorem-color}{gray}{0.90} % уровень прозрачности (1 - максимум)
\newtcolorbox{htheorem}{colback=theorem-color,grow to right by=-4mm,grow to left by=-4mm,
    boxrule=0pt,boxsep=0pt,breakable} % настройки области с изменённым фоном

%определение
\definecolor{def-color}{gray}{0.98}
\newtcolorbox{definit}{colback=def-color,grow to right by=-4mm,grow to left by=-4mm,
    boxrule=0pt,boxsep=0pt,breakable} % настройки области с изменённым фоном

%доказательсвто теоремы
\definecolor{proof-color}{gray}{0.95} % уровень прозрачности (1 - максимум)
\newtcolorbox{hproof}{colback=proof-color,grow to right by=-1mm,grow to left by=-1mm,
    boxrule=0pt,boxsep=0pt,breakable} % настройки области с изменённым фоном

%замечания, следствия
\definecolor{consectary-color}{gray}{0.95} % уровень прозрачности (1 - максимум)
\newtcolorbox{cns}{colback=consectary-color,grow to right by=-4mm,grow to left by=-4mm,
    boxrule=0pt,boxsep=0pt,breakable} % настройки области с изменённым фоном



\usepackage{fancybox,fancyhdr}
\pagestyle{fancy}
\fancyhf{}
\fancyhead[R]{ФТ-104}
\fancyfoot[R]{\thepage}
\fancyhead[L]{мотан}

\usepackage{hyperref}
\hypersetup{colorlinks=true, allcolors=[RGB]{010 090 200}} % цвет ссылок 
\newcommand{\lr}[1]{\left({#1}\right)} % команда для скобок

\title{Конспектик к экзамену по матану}
\author{Васильев Павел}
%\linespread{1}
\usepackage{amsmath}

\usepackage{graphicx}
\usepackage{ifpdf}
\ifpdf
\DeclareGraphicsRule{*}{mps}{*}{}
\fi
\usepackage{graphicx}
\usepackage{color}
\graphicspath{ {images/} }

%\renewcommand{\familydefault}{\sfdefault}

\begin{document}

\section*{Ботаем экзамен по матанализу}

\subsection*{Билет 1. Определение предела функции в точке по Коши и по Гейне, их равносильность}

\begin{itemize}
\item По Коши: 
	\begin{equation}
		\displaystyle a = \lim_{x \rightarrow x_0} f(x) \Leftrightarrow \forall \epsilon > 0 \quad \exists \delta ( \epsilon) > 0 \quad \forall x \in D(f): 0<|x-x_0|< \delta \Rightarrow |f(x) - a| < \epsilon
	\end{equation}
	
\item По Гейне:
	\begin{equation}
		\displaystyle a = \lim_{x \rightarrow x_0} f(x) \Leftrightarrow \forall \{ x_n \} \subseteq D(f) \backslash{} \{ x_0 \}: x_n \rightarrow x_0 \Rightarrow f(x_n) \rightarrow a
	\end{equation}
\end{itemize}

\begin{htheorem}
\textbf{Теорема. Коши $\Leftrightarrow$ Гейне}
\end{htheorem}

\begin{hproof}
\textbf{Доказательство.}

Пусть $a$ - предел по Гейне, то есть $\displaystyle a = \lim_{x \rightarrow x_0} f(x) \Leftrightarrow \forall \{ x_n \} \subseteq D(f) \backslash{} \{ x_0 \}: x_n \rightarrow x_0 \Rightarrow f(x_n) \rightarrow a$. Хотим $\displaystyle a = \lim_{x \rightarrow x_0} f(x) \Leftrightarrow \forall \epsilon > 0 \quad \exists \delta ( \epsilon) > 0 \quad \forall x \in D(f): 0<|x-x_0|< \delta \Rightarrow |f(x) - a| < \epsilon$.

От противного: пусть $\exists \epsilon > 0 \quad \forall \delta > 0 \quad 0 < |x-x_0| < \delta$ и $|f(x) - a| \geq \epsilon$.

Возьмём $\displaystyle \delta = \frac{1}{2}, \delta = \frac{1}{2}, ..., \delta = \frac{1}{n}, ...$

$\displaystyle |x-x_0| < \frac{1}{n}$ и $|f(x)-a| \geq \epsilon$, то есть $x_n \in U_{\frac{1}{n}}(x_0) \Rightarrow x_n \rightarrow x_0$, но $f(x_n) \nrightarrow a$
\end{hproof}



\end{document}

