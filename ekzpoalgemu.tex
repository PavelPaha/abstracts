\documentclass[a4paper]{article}
\usepackage[utf8]{inputenc}
\usepackage[T2A]{fontenc}
\usepackage[12pt]{extsizes}

\usepackage[english,russian]{babel}
\usepackage[left=10mm, top=10mm, right=10mm, bottom=20mm, nohead, nofoot]{geometry}
\usepackage{amsmath,amsfonts,amssymb} % математический пакет
\headsep=10mm

\usepackage[most]{tcolorbox} % для управления цветом
% НАСТРОЙКИ
%теорема
\definecolor{theorem-color}{gray}{0.90} % уровень прозрачности (1 - максимум)
\newtcolorbox{htheorem}{colback=theorem-color,grow to right by=-4mm,grow to left by=-4mm,
    boxrule=0pt,boxsep=0pt,breakable} % настройки области с изменённым фоном

%определение
\definecolor{def-color}{gray}{0.98}
\newtcolorbox{definit}{colback=def-color,grow to right by=-4mm,grow to left by=-4mm,
    boxrule=0pt,boxsep=0pt,breakable} % настройки области с изменённым фоном

%доказательсвто теоремы
\definecolor{proof-color}{gray}{0.95} % уровень прозрачности (1 - максимум)
\newtcolorbox{hproof}{colback=proof-color,grow to right by=-1mm,grow to left by=-1mm,
    boxrule=0pt,boxsep=0pt,breakable} % настройки области с изменённым фоном

%замечания, следствия
\definecolor{consectary-color}{gray}{0.95} % уровень прозрачности (1 - максимум)
\newtcolorbox{cns}{colback=consectary-color,grow to right by=-4mm,grow to left by=-4mm,
    boxrule=0pt,boxsep=0pt,breakable} % настройки области с изменённым фоном



\usepackage{fancybox,fancyhdr}
\pagestyle{fancy}
\fancyhf{}
\fancyhead[R]{ФТ-104}
\fancyfoot[R]{\thepage}
\fancyhead[L]{алгем}

\usepackage{hyperref}
\hypersetup{colorlinks=true, allcolors=[RGB]{010 090 200}} % цвет ссылок 
\newcommand{\lr}[1]{\left({#1}\right)} % команда для скобок

\title{Конспектик к экзамену по алгему}
\author{Васильев Павел}
%\linespread{1}
\usepackage{amsmath}

\usepackage{graphicx}
\usepackage{ifpdf}
\ifpdf
\DeclareGraphicsRule{*}{mps}{*}{}
\fi
\usepackage{graphicx}
\usepackage{color}
\graphicspath{ {images/} }

%\renewcommand{\familydefault}{\sfdefault}

\begin{document}


    \section*{Ботаем экзамен по алгему}

    \begin{center}
        \begin{Large}
            \fbox{БИЛЕТ 1,2}
        \end{Large}
    \end{center}

    \subsection*{Декартово произведение множеств}

    \begin{definit}
        \textbf{Прямое, или декартово произведение двух множеств} — множество, элементами которого являются все возможные упорядоченные пары элементов исходных множеств
    \end{definit}

    $\{ (x;y) | x \in A, y \in B \}$

    \subsection*{Понятие отношения на множестве
    }

    \begin{definit}
        Пусть $M_1, M_2, …, M_n$ – некоторые множества.
        \textbf{Отношением} на совокупности этих множеств называется любое подмножество
        декартова произведения этих множеств. Если $M_1 = M_2 = … = M_n = M$, то говорят об отношении на множестве M.
    \end{definit}

    \subsection*{Свойства отношений}

    \begin{itemize}
        \item \textit{Рефлексивность}: $\forall a \in X (a R a)$
        \item \textit{Симметричность}: $\forall a, b \in X (a R b \Rightarrow b R a)$
        \item \textit{Антисимметричность}: $\forall a, b \in X (a R b \land b R a \Rightarrow a = b)$
        \item \textit{Транзитивность}: $\forall a, b, c \in X (a R b \land b R c => a R c)$


    \end{itemize}


    \subsection*{Отношение эквивалентности
    }
    Отношение на множестве $M$ называется \textbf{отношением
    эквивалентности}, если оно \textbf{рефлексивно, симметрично} и \textbf{транзитивно}.

    Вот несколько примеров отношений эквивалентности:
    \begin{itemize}
        \item на любом множестве: отношение равенства;
        \item на множестве треугольников: отношение подобия;
        \item на множестве действительных чисел: иметь одинаковую целую часть;
        \item на множестве вершин графа: быть связанными;
        \item на множестве людей: быть одного года рождения.
    \end{itemize}


    \subsection*{Теорема о разбиении}
    \begin{definit}
        \textbf{Разбиением множества M} называется его представление в виде объединения непустых непересекающихся подмножеств.
    \end{definit}

    \begin{htheorem}
        \textbf{Теорема о разбиении множества}
        Каждое отношение эквивалентности
        задаёт разбиение множества, на котором оно определено. Любое разбиение множества задается некоторым отношением эквивалентности.
    \end{htheorem}


    \begin{hproof}
        \textbf{Доказательство}

        Пусть R –
        отношение эквивалентности на множестве $M$. Для каждого элемента a из M построим множество $M_a = \{x | x \in M$ и $x R a\}$. Среди этих множеств могут
        оказаться одинаковые. Соберём совокупность всех не совпадающих между
        собой множеств $M_a$ и покажем, что их объединение образует разбиение множества M.

        Во-первых, заметим, что каждое множество не пусто, поскольку $a \in M_a$ в
        силу рефлексивности отношения $R$.

        Во-вторых, объединение всех выбранных нами множеств совпадает с $M$,
        поскольку каждый элемент из $M$ попадает в подмножество, отмеченное им
        самим в роли индекса.

        В третьих, покажем, что два различных множества $M_a$ и $M_b$ не пересекаются.
        Допустим противное: пусть $c \in M_a \cap M_b$. По построению множеств $M_a$ и $M_b$ это
        означает, что $c R a$ и $c R b$. Ввиду симметричности отношения R имеем $a R c$.
        Выберем теперь произвольный x из $M_a$. Поскольку $x R a$ и $a R c$, транзитивность
        отношения показывает, что $x R c$. Но при этом $c R b$. Применяя ещё раз свойство
        транзитивности, получаем $x R b \Rightarrow x \in M_b$. Так как x произвольный, $M_a \subseteq M_b$. В то же время элементы $A$ и $B$ абсолютно
        равноправны, поэтому $M_b \subseteq M_a$. Значит, $M_a = M_b$, что противоречит тому, что $M_a$ и $M_b$ различны.

        Пусть теперь имеется некоторое разбиение множества M:
        $M = \bigcup \{ M_i | M_i \neq \emptyset$ и $M_i \cap M_j = \emptyset$ при $i \neq j\}$

        Определим на $M$ следующее отношение R: $aRb$, если найдётся множество $M_i$, для которого $a \in M_i$ и $b \in M_i$.

        Покажем, что $R$ – отношение эквивалентности.

        Во-первых, $R$ рефлексивно. По определению объединения множеств каждый
        элемент $a$ из $M$ попадает хотя бы в одно подмножество $M_i$. Это означает, что
        $aRa$.

        Во-вторых, $R$ симметрично. Ясно, что если для пары $(a, b)$ нашлось
        множество $M_i$, для которого $a \in M_i$ и $b \in M_i$, то это же множество годится для
        пары $(b, a)$.

        В-третьих, R транзитивно. Пусть $a, b$ и $c$ – такие элементы, что $aRb$ и $bRc$.
        Значит, найдётся такое множество $M_i$, для которого $a \in M_i$ и $b \in M_i$, и такое
        множество $M_k$, для которого $b \in M_k$ и $c \in M_k$. Мы видим, что b оказался общим
        элементом множеств $M_i$ и $M_k$, а по определению разбиения разные его
        подмножества общих элементов не имеют. Следовательно, $M_i = M_k$, а тогда $aRc$.

        Ясно также, что отношение R задаёт именно то разбиение, на основании
        которого оно было построено.


    \end{hproof}

    \subsection*{Отношение порядка}
    Отношение на множестве $M$ называется \textbf{отношением
    порядка}, если оно \textbf{рефлексивно, транзитивно и антисимметрично}.
    Вот несколько примеров отношений порядка:
    \begin{itemize}
        \item на множестве прямоугольников: содержаться;
        \item на множестве действительных чисел: меньше или равно;
        \item на множестве сотрудников одного учреждения: быть начальником.
    \end{itemize}

    \subsection*{Максимальные и минимальные элементы}
    Элемент $a$ множества $А$, упорядоченного отношением $\unlhd$ называется \textbf{минимальным}, если не существует элемента $b$, не равного $a$ такого, что $a \unlhd b$\newline

    Элемент $a$ множества $А$, упорядоченного отношением $\unlhd$ называется \textbf{максимальным}, если не существует элемента $b$, не равного $a$ такого, что $b \unlhd a$

    \subsection*{Наибольшие и наименьшие элементы
    }
    Элемент $a \in A$ называется \textbf{наименьшим}, если $\forall x \in A (a \unlhd x)$\newline
    Элемент $a \in A$ называется \textbf{наибольшим}, если $\forall x \in A (x \unrhd a)$\newline
    Таким образом, минимальный элемент всегда является наименьшим.

    \newpage \begin{center}
                 \begin{Large}
                     \fbox{БИЛЕТ 3}
                 \end{Large}
    \end{center}

    \subsection*{Отображения множеств
    }

    Отображением
    множества $M_1$ в множество $M_2$ называют бинарное отношение, определённое на
    этих множествах, если первый компонент пары $(a, b) \in M_1 \times M_2$ рассматривается как \textit{аргумент}, а второй – как \textit{значение} для этого аргумента.

    \subsection*{Свойства отображений}

    Отображение $f$ множества $M_1$ в множество $M_2$ называется
    \textbf{всюду определённым}, если $D(f) = M_1$.

    Отображение $f$ множества $M_1$ в множество $M_2$ называется
    \textbf{сюръективным}, если $E(f) = M_2$.

    Отображение $f$ множества $M_1$ в множество $M_2$ называется
    \textbf{однозначным}, если каждый элемент a из $D(f)$ имеет ровно одно значения в множестве $M_2$:
    \begin{equation}
        \forall a \in M_1 \forall b \in M_2 \forall c \in M_2 (b  = f(a) \land c = f(a) \Rightarrow b = c)
    \end{equation}


    Отображение $f$ множества $M_1$ в множество $M_2$ называется
    \textbf{инъективным}, если каждый элемент b из $E(f)$ является значением только одного элемента из $M_1$:
    \begin{equation}
        \forall b \in M_2 \forall a \in M_1 \forall c \in M_1 (b = f(a) \land b = f(c) \Rightarrow a = c)
    \end{equation}

    Отображение $f$ множества $M_1$ в множество $M_2$ называется
    \textbf{биективным}, если оно \textbf{инъективно и сюръективно}.
%\begin{equation}
%\forall x_1 \in M_1 \forall x_2 \in M_2 \quad x_1 \neq x_2 \Rightarrow f(x_1) \neq f(x_2) \land \forall y \in M_2 \exists x \in M_1 \quad f(x) = y
%\end{equation}

    Отображение $f$ множества $M_1$ в множество $M_2$ называется
    \textbf{взаимнооднозначным}, если оно \textbf{инъективно, однозначно и сюръективно}.

    \subsection*{Обратное отображение
    }
    Пусть $f$ : $X \rightarrow Y$ --- биективное отображение. Тогда каждому $y \in F$ соответствует единичный $x$, который обозначается как $f^{-1} (y)$ b такой, что $f(x) = y$. Таким образом определено отображение $f^{-1}$:$F \rightarrow E$, которое называется \textbf{обратным} отображению $f$.

    \subsection*{Композиция отображений и ее свойства}
    \textbf{Композицией отображений} $f$: $X \rightarrow Y$ и $g$:$Y \rightarrow Z$ называется отображение $f \circ g$: $X \rightarrow Z$, обозначающее $f(g(x))$.

    Свойства:
    \begin{itemize}
        \item Композиция двух отображений определена, тогда и только тогда, когда область значений первого отображения совпадает с областью отображения второго.
        \item Отображение тогда и только тогда
        имеет обратное, когда оно взаимно однозначно (биективно).
        \item Из биективности отображения вытекает биективность обратного отображения
    \end{itemize}


    \subsubsection**{Свойства композиции отображений}

    \begin{itemize}
        \item \textit{Ассоциативность}: $f \circ (g \circ h) = (f \circ g) \circ h$
        \item \textit{Некоммутативность}: $fg \neq gf$

    \end{itemize}

    Ещё важные свойства \begin{itemize}
                            
                            \item Отображение, \textbf{обратное всюду определенному} отображению, \textbf{сюръективно};
                            \item Отображение, \textbf{обратное однозначному} отображению, \textbf{инъективно};
                            \item Отображение, \textbf{обратное сюръективному} отображению, \textbf{всюду определенное};
                            \item Отображение, \textbf{обратное инъективному} отображению, \textbf{однозначно}.
                            \item Отображение, \textbf{обратное взаимно однозначному} отображению, \textbf{взаимно однозначно}.
                            \item Композиция \textbf{взаимно однозначных} отображений является \textbf{взаимно однозначным} отображением.
                            \item Композиция всюду определенных (однозначных, сюръективных, инъективных) отображения является вюду определенным отображением.
    \end{itemize}


    \newpage \begin{center}
                 \begin{Large}
                     \fbox{БИЛЕТ 4}
                 \end{Large}
    \end{center}
    \subsection*{Операции на множестве}
    \textbf{Операцией} на множестве $M$ называется всюду
    определённая функция из $M^n$ в $M$. Число $n$ называют \textit{арностью}, или
    \textit{местностью}, данной операции.
    Вот несколько примеров операций:
    \begin{itemize}
        \item на множестве натуральных чисел: сложение двух чисел; это бинарная
        (двуместная) операция;
        \item на множестве целых чисел: нахождение числа, противоположного
        данному; это унарная (одноместная) операция;
        \item на множестве рациональных чисел: нахождение среднего арифметического
        n чисел; это n-арная (n-местная) операция;
        \item на множестве подмножеств данного множества: операция пресечения
        подмножеств; это бинарная операция.
    \end{itemize}

    \subsection*{Свойства операций}
    \begin{itemize}
        \item \textit{Коммутативность} операции $\circ$: $\forall x, y \in M \quad (x \circ y = y \circ x)$
        \item \textit{Ассоциативность} операции $\circ$: $\forall x, y, z \in M \quad (x \circ y) \circ z = x \circ (y \circ z)$
        \item Элемент $e$ из M называется \textit{нейтральным} относительно операции $\circ$: $\forall x \in M \quad x \circ e = e \circ x = x$
        \item Элемент $y$ из M называется \textit{симметричным} относительно операции $\circ$: $x \circ y = y \circ x = e$
    \end{itemize}

    \subsection*{Понятие полугруппы, группы}
    \textbf{Полугруппа} --- множество, на котором определена ассоциативная операция.

    Примеры полугрупп:
    \begin{itemize}
        \item множество натуральных чисел как относительно операции сложения, так и
        операции умножения;
        \item множество отрицательных целых чисел относительно сложения;
        \item булеан множества $M$ относительно операций объединения и пересечения;
        \item множество всюду определённых функций, отображающих множество M в
        себя, относительно операции композиции
    \end{itemize}

    \textbf{Группа} --- множество, на котором определена ассоциативная операция, имеется нейтральный элемент и каждый элемент обладает симметричным.

    Примеры групп:
    \begin{itemize}
        \item множество целых чисел относительно операции сложения;
        \item множество положительных рациональных чисел относительно операции
        умножения;
        \item множество ненулевых действительных чисел относительно операции
        умножения;
        \item множество взаимно-однозначных отображений произвольного множества M на себя.
    \end{itemize}

    \subsection*{Симметрическая группа}
    \textbf{Симметрической группой ($S_n$)} называется множество \textit{подстановок} на множестве (подстановка на множестве $M_n$ --- взаимнооднозначное отображение множества $M_n$ на себя)

    \subsection*{Разрешимость уравнений в группе
    }
    \begin{htheorem}
        \textbf{Теорема.}  В группе  $G$ уравнение $x \circ x = x$ имеет
        единственное решение $x = e$, где $e$ – нейтральный элемент группы.
    \end{htheorem}

    \begin{hproof}
        \textbf{Доказательство.} Так как $e \circ e = e$, $e$ --- решение. Пусть $x_0$ --- какое-то решение системы. Тогда $x_0 = x_0 \circ e = x_0 \circ (x_0 \circ x_0^{-1}) = (x_0 \circ x_0) \circ x_0^{-1} = x_0 \circ x_0^{-1} = e$.
    \end{hproof}

    \begin{htheorem}
        \textbf{Теорема.} Если в полугруппе M с операцией $\circ$ для любых элементов $A$ и $B$
        существуют такие элементы x и y, для которых $a \circ x = b$ и $y \circ a  = b$, то $M$ является группой относительно этой операции.
    \end{htheorem}


    \begin{hproof}
        \textbf{Доказательство.} Сначала покажем, что в полугруппе M есть нейтральный
        элемент. Выберем какой-нибудь элемент $a$ из $M$ и рассмотрим уравнение $a \circ x = a$. Обозначим через $e_1$ какое-либо его решение (нам не дано, что уравнение
        имеет единственное решение!). Покажем, что для любого элемента $c$ из $M$
        выполнено равенство $x \circ e_1 = c$. Для этого рассмотрим уравнение $y \circ a  = c$ и обозначим через $c_1$ какое-нибудь его решение. Напишем цепочку равенств:

        $c \circ e_1 = (c_1 \circ a ) \circ e_1 = c_1 \circ ( a \circ e_1) = c_1 \circ a = c$

        Теперь рассмотрим уравнение $y \circ a = a$ и обозначим через $e_2$ какое-либо его
        решение. Аналогично доказывается, что для любого элемента c из M
        выполнено равенство $e_2 \circ c = c$.

        Наконец, заметим, что $e_2 = e_2 \circ e_1 = e_1$. Следовательно, $e_1 = e_2 = e$ –
        нейтральный элемент полугруппы M.

        Докажем теперь наличие симметричного у любого элемента a из M.
        Рассмотрим уравнения $a \circ x = e$ и $y \circ a = e$. Обозначим через $x_0$ и $y_0$ решения
        этих уравнений. Тогда $x_0 = e \circ x_0 = (y_0 \circ a) \circ x_0 = y_0 \circ (a \circ x_0) = y_0 \circ e = y_0$, т.е.
        элемент $x_0 = y_0$ симметричен элементу a.
        Эта теорема показывает, что желание иметь в данном множестве решения
        для любого линейного уравнения при условии ассоциативности операции
        неизбежно приводит к понятию группы. готово брат.
    \end{hproof}

    \begin{htheorem}
        \textbf{Теорема}. Пусть G – группа относительно операции $\circ$. Тогда для любых
        элементов $a$ и $b$ из $G$ существуют и при том единственные такие элементы $x$ и $y$,
        для которых $a
        \circ x = b$ и $y \circ a = b.$
    \end{htheorem}


    \begin{hproof}
        \textbf{Доказательство.}  Для элемента $a$ существует симметричный $a^{-1}$. Положим
        $x_0 = a^{-1} \circ b.$ Тогда
        $a \circ x_0 = a \circ (a^{-1} \circ b) = (a \circ a^{-1} ) \circ b = e \circ b = b$, то есть построенный нами элемент $x_0$ удовлетворяет требованиям теоремы.

        Покажем теперь, что любой элемент группы G, удовлетворяющий равенству $a \circ x = b$, совпадает с $x_0$. Пусть $x_1$ таков, что $a \circ x_1 = b$. Тогда $a^{-1} \circ (a \circ x_1) = a^{-1} \circ b$. В то же время $a^{-1} \circ (a \circ x_1) = (a^{-1} \circ a) \circ x_1 = e \circ x_1 = e \circ x_1 = x_1 \Rightarrow x_1 a^{-1} \circ b = x_0$

        \textbf{Замечание}. Если операция $\circ$  не коммутативна, то элементы $a^{-1} \circ b$ и $b \circ a^{-1}$ могут и не совпадать.
    \end{hproof}

    Теорема показывает, что в любой группе разрешимы уравнения первой
    степени. Уравнения более высоких степеней, скажем, квадратные, уже могут не
    иметь решений. Например, в группе положительных рациональных чисел
    относительно операции умножения уравнение $x^2 = 2$ решений не имеет.


    \newpage \begin{center}
                 \begin{Large}
                     \fbox{БИЛЕТ 5}
                 \end{Large}
    \end{center}

    \subsection*{Кольца и их свойства}
    \textbf{Кольцо} --- множество M, на котором определены две бинарные операции $\circ$ и $*$, удовлетворяющие следующим условиям:
    \begin{itemize}
        \item M --- группа относительно $\circ$
        \item $*$ дистрибутивна относительно $\circ$
    \end{itemize}

    Примеры:
    \begin{itemize}
        \item множество целых чисел относительно операций сложения и умножения;
        \item множество действительных чисел относительно операций сложения и
        умножения;
        \item множество многочленов с действительными коэффициентами
        относительно операций сложения и умножения;
        \item множество функций из R в R относительно операций сложения и умножения.
    \end{itemize}

    \subsection*{Области целостности и поля}
    \textbf{Область целостности} --- коммутативное ассоциатовное кольцо без \textit{делителей нуля} (Ненулевые элементы $a$ и $b$ кольца K называются \textit{делителями нуля}, если $ab = 0$).

    \textbf{Поле} --- коммутативное ассоциативное кольцо с 1, каждый ненулевой элемент которого обратим.


    \newpage \begin{center}
                 \begin{Large}
                     \fbox{БИЛЕТ 6}
                 \end{Large}
    \end{center}
    \subsection*{Понятие вектора}
    \textbf{Вектор} — это элемент векторного пространства (некоторого множества с двумя операциями на нём, которые подчиняются восьми аксиомам).

    \textbf{Вектором} называется отрезок,
    концы которого упорядочены. Первый из его концов
    называется началом, второй – концом вектора.

    \subsection*{Операции сложения и умножения векторов и их свойства}

    Свойства:
    \begin{itemize}
        \item \textit{Коммутативность}: $\vec{a} + \vec{b} = \vec{b} + \vec{a}$
        \item \textit{Ассоциативность сложения} $(\vec{a} + \vec{b}) + \vec{c} = \vec{a} + (\vec{b}+\vec{c})$
        \item \textit{Нейтральный элемент} относительно сложения $\vec{0}$: $\vec{a} + \vec{0} = \vec{a}$
        \item \textit{Противоположный вектор} для любого ненулевого относительно сложения: $\vec{a}+(-\vec{a}) = \vec{0}$
        \item \textit{Ассоциативность умножения}: $(\lambda \cdot \eta) \cdot \vec{a} = \lambda \cdot ( \eta \cdot \vec{a})$
        \item \textit{Дистрибутивность} умножения относительно сложения: $\lambda \cdot (\vec{a} + \vec{b}) = \lambda \cdot \vec{a} + \lambda \cdot \vec{b}$, $(\lambda + \eta) \cdot \vec{a} = \lambda \cdot \vec{a} + \eta \cdot \vec{a}$
        \item \textit{Нейтральный элемент} относительно умножения: $1 \cdot \vec{a} = \vec{a}$
    \end{itemize}

    \newpage \begin{center}
                 \begin{Large}
                     \fbox{БИЛЕТ 7}
                 \end{Large}
    \end{center}
    \subsection*{Коллинеарность векторов}
    Два вектора $\vec{a}$ и $\vec{b}$ \textbf{коллинеарны}, если $\exists t$: $\vec{a} = t \cdot \vec{b}$.

    Два вектора \textbf{коллинеарны}, если отношения их координат равны.

    Два вектора \textbf{коллинеарны}, если их векторное произведение равно нулевому вектору.


    \newpage \begin{center}
                 \begin{Large}
                     \fbox{БИЛЕТ 8}
                 \end{Large}
    \end{center}
    \subsection*{Базис на плоскости}
    \textbf{Базисом} на плоскости называется любая упорядоченная пара линейно независимых векторов, принадлежащих этой плоскости.


    \begin{htheorem}
        \textbf{Теорема о разложении вектора по базису на плоскости}

        Пусть $(\vec{a},\vec{b})$ – базис некоторой плоскости, а $\vec{x}$ – вектор, лежащий в этой
        плоскости. Тогда существуют, и притом единственные, числа $t_1$ и $t_2$ такие,
        что
        \begin{equation}
            \vec{x} = t_1 \vec{a} + t_2 \vec{b}
        \end{equation}

    \end{htheorem}


    \textbf{Доказательство}
    Отложим вектора $\vec{a}$, $\vec{b}$ и $\vec{x}$ от некоторой точки O нашей
    плоскости и обозначим концы полученных направленных отрезков через A, B и M соответственно.

    \includegraphics[width=10cm]{t1}

    Спроектируем точку $M$ на прямую $OA$ параллельно прямой $OB$ и на прямую $OB$ параллельно прямой $OA$. Обозначим полученные точки через $A'$ и $B'$ соответственно и положим  $\vec{a}' := \overrightarrow{OA}'$ и $\vec{b}' := \overrightarrow{OB}'$. Ясно, что $\vec{a}' \parallel \vec{a}$ и $\vec{b}' \parallel \vec{b}$. Поскольку $\vec{a}$, $\vec{b} \neq \vec{0}$, по критерию коллинеарности векторов $\vec{a}' = t_1 \vec{a}$ и $\vec{b}' = t_1 \vec{b}$ для некоторых чисел $t_1$ и $t_2$.
    Тогда $\vec{x} = t_1 \vec{a} + t_2 \vec{b}$.
    Осталось доказать единственность. Пусть $\vec{x} = s_1 \vec{a} + s_2 \vec{b}$ для некоторых чисел $s_1$ и $s_2$.
    Вычитая это равенство из равенства $\vec{x} = t_1 \vec{a} + t_2 \vec{b}$
    имеем $(t_1 - s_1) \vec{a} + (t_2 - s_2) \vec{b} = \vec{0}$.
    Если $t_1-s_1 \neq 0$, то $\displaystyle \vec{a} = - \frac{t_2 - s_2}{t_1 - s_1} \cdot \vec{b} \parallel \vec{b}$, противоречие.
    Следовательно, $t_1 - s_1 = 0$, то есть $t_1 = s_1$. Доказано


    \subsection*{Действия с векторами в координатной форме}

	Ниже полезные формулки в ПСК:

    \includegraphics[width=18cm]{t2}

    \newpage \begin{center}
                 \begin{Large}
                     \fbox{БИЛЕТ 9}
                 \end{Large}
    \end{center}
    \subsection*{Компланарность векторов}
    \textbf{Компланарные векторы} — это векторы, которые параллельны одной плоскости или лежат на одной плоскости.

    Условия компланарности векторов:\begin{itemize}
                                        \item Для 3-х векторов выполняется условие: если смешанное произведение 3-х векторов равно нулю, то эти три вектора компланарны
                                        \item Для 3-х векторов выполняется условие: если три вектора линейно зависимы, то они компланарны.
                                        \item если среди векторов не более 2-х линейно независимых векторов, то они компланарны.
    \end{itemize}
    \subsection*{Базис пространства}
    \textbf{Базисом пространства }называется упорядоченная тройка некомпланарных
    векторов

    \begin{htheorem}
        \textbf{Теорема о разложении вектора по базису в пространстве}

        Пусть $( \vec{a}, \vec{b}, \vec{c})$ --- базис пространства, а $\vec{x}$ --- произвольный вектор. Тогда существуют единственные $t_1$, $t_2$, $t_3$ такие, что
        \begin{equation}
            \vec{x} = t_1 \vec{a} + t_2 \vec{b} + t_3 \vec{x}
        \end{equation}
    \end{htheorem}


    \begin{hproof}
        \textbf{Доказательство}

        Отложим вектора $\vec{a}$, $\vec{b}$  и $\vec{c}$ от некоторой точки $О$ и обозначим концы полученных направленных отрезков через $A, B, C$ и $M$ соответственно.

        Поскольку $\vec{a}$ и $\vec{b}$ неколлинеарны, существует единственная плоскость $\pi$, проходящая через точки $O, A$ и $B$. Спроектируем точку M на плоскость $\pi$ параллельно прямой $OC$ и на прямую $OC$ параллельно плоскоси $\pi$.

        \includegraphics[width=12cm]{t3}

        Обозначим полученные точки как $M'$ и $C'$ и положии $\vec{x}' := \overrightarrow{OM}'$ и $\vec{c}' := \overrightarrow{OC}'$. По теореме о разложении вектора по базису на плоскости  $\vec{x} = t_1 \vec{a} + t_2 \vec{b}$ для некоторых $t_1$ и $t_2$. Далее $\vec{c}' \parallel \vec{c} \neq \vec{0}$, откуда $\vec{c}' = t_3 \vec{c}$ для некоторого $t_3$. Тогда $\vec{x} = \vec{x'} + \vec{c}' = t_1 \vec{a} + t_2 \vec{b} + t_3 \vec{c}$. Существование чисел $t_1$, $t_2$, $t_3$ с требуемыми свойствами доказано.
        Осталось доказать их единственность. Пусть $\vec{x} = s_1 \vec{a} + s_2 \vec{b} + s_3 \vec{c}$ для некоторых $s_1$, $s_2$ и $s_3$. Вычитая это равенство их равенства $\vec{x} = t_1 \vec{a} + t_2 \vec{b} + t_3 \vec{c}$, получим
        \begin{equation}
        (t_1 -s_1)
            \vec{a} + (t_2 - s_2)\vec{b} + (t_3 - s_3) \vec{c} = \vec{0}
        \end{equation}
        Если $\displaystyle t_1 -s_1 \neq    0$, то $\displaystyle \vec{a} = - \frac{t_2 - s_2}{t_1 - s_1} \cdot \vec{b} - \frac{t_3 - s_3}{t_1 - s_1} \cdot \vec{c}$. Но тогда вектора $\vec{a}$, $\vec{b}$ и $\vec{c}$ компланарны, что противоречит условию $\Rightarrow t_1 - s_1 = 0 \Rightarrow t_1 = s_1$. Аналогично $t_2 = s_2$ и $t_3 = s_3$.
    \end{hproof}


    \newpage \begin{center}
                 \begin{Large}
                     \fbox{БИЛЕТ 10}
                 \end{Large}
    \end{center}
    \subsection*{Скалярное произведение векторов}

    \textbf{Скалярным произведением} ненулевых векторов называется число, равное
    произведению длин этих векторов на косинус угла между ними. Скалярное
    произведение нулевого вектора на любой вектор по определению равно 0.
    Скалярное произведение $\vec{a}$ и $\vec{b}$ обозначается через $\vec{a} \vec{b}$
    \begin{equation}
        \vec{a} \vec{b} = |\vec{a}| \cdot | \vec{b} | \cdot \cos(\widehat{\vec{a}, \vec{b}})
    \end{equation}

    Свойства:
    \begin{itemize}
        \item $\vec{a} \vec{b} = \vec{b} \vec{a}$
        \item $(\vec{a} + \vec{b}) \vec{c} = \vec{a} \vec{c} + \vec{b} \vec{c}$
        \item $(t \vec{a}) \vec{b} = t (\vec{a} \vec{b})$
        \item $\vec{a} \vec{a} \geq 0$, причём $\vec{a} \vec{a} = 0$ только тогда, когда $\vec{a} = \vec{0}$.
    \end{itemize}

    \subsection*{Компонента вектора на прямую и проекция вектора на ось
    }
    Если $\vec{a} = x \vec{i} + y \vec{j} + z \vec{k}$, то $x \vec{i}$, $y \vec{j}$ и $z \vec{k}$ --- \textbf{компоненты} этого вектора.

    \textbf{Проекция} вектора на ось --– это вектор, началом и концом которого являются соответственно проекции начала и конца заданного вектора.

    \subsection*{Свойства компоненты, проекции и скалярного произведения}

    Свойства проекций (пусть векторы $\vec{a}$ и $\vec{b}$ проецируются на прямую $l$):
    \begin{itemize}
        \item $pr_l (\vec{a} + \vec{b}) = pr_l \vec{a} + pr_l \vec{b}$
        \item $pr_l (t \vec{a}) = t pr_l \vec{a}$
    \end{itemize}


    \newpage \begin{center}
                 \begin{Large}
                     \fbox{БИЛЕТ 11}
                 \end{Large}
    \end{center}
    \subsection*{Векторное и смешанное произведения векторов}

    \textbf{Упорядоченная тройка} некомпланарных векторов $(\vec{u}, \vec{v}, \vec{w})$  называется
    правой, если из конца вектора $\vec{w}$ поворот от $\vec{u}$ к $\vec{v}$ по наименьшему углу
    выглядит происходящим против часовой стрелки, и левой – в противном случае. Правую тройку векторов называют также положительно
    ориентированной, а левую – отрицательно ориентированной.

    \textbf{Векторным произведением} неколлинеарных векторов $\vec{a}$ и $\vec{b}$ называется
    вектор $\vec{c}$ такой, что:
    \begin{itemize}
        \item $| \vec{c}| = | \vec{a} | \cdot | \vec{b} | \cdot \sin (\widehat{\vec{a}, \vec{b}})$
        \item $\vec{c}$ ортогонален к векторам $\vec{a}$ и $\vec{b}$
        \item тройка $(\vec{a}, \vec{b}, \vec{c})$ --- правая.
    \end{itemize}

    Если $\vec{a}, \vec{b}$ и $\vec{c}$ --- произвольные вектора, а $t$ - произвольное число, то
    \begin{itemize}
        \item $\vec{a} \times \vec{b} = -\vec{b} \times \vec{a}$ (антикоммутативность)
        \item $(t \vec{a}) \times \vec{b} = \vec{a} \times (t \vec{b}) = t ( \vec{a} \times \vec{b})$
        \item $(\vec{a} +\vec{b}) \times \vec{c} = \vec{a} \times \vec{c} + \vec{b} \times \vec{c}$
        \item $\vec{a} \times (\vec{b} + \vec{c}) = \vec{a} \times \vec{b} +\vec{a} \times \vec{c}$
    \end{itemize}

    \textbf{Смешанным произведением} векторов $\vec{a}, \vec{b}$ и $\vec{c}$ называется число, равное
    скалярному произведению векторного произведения векторов $\vec{a}$ и $\vec{b}$
    на вектор $\vec{c}$ (обозначается $\vec{a} \vec{b} \vec{c}$.
    Таким образом, $\vec{a} \vec{b} \vec{c} := (\vec{a} \times \vec{b}) \vec{c}$.

    \begin{htheorem}
        \textbf{Критерий компланарности векторов}. Вектора $\vec{a}, \vec{b}$ и $\vec{c}$ компланарны тогда и только тогда, когда их смешанное
        произведение равно нулю.
    \end{htheorem}

    \begin{hproof}
        \textbf{Доказательство.} Необходимость. Предположим, что вектора $\vec{a}, \vec{b}$ и $\vec{c}$ компланарны. Если $\vec{a} \parallel \vec{b}$, то $\vec{a} \times \vec{b} = \vec{0}$, и потому $\vec{a} \vec{b} \vec{c} = (\vec{a} \times \vec{b} ) \vec{c} = \vec{0}$. \\
        Пусть теперь $\vec{a} \nparallel \vec{b}$. Отложим вектора $\vec{a}, \vec{b}$ и $\vec{c}$ от одной точки. Тогда они будут лежать в некоторой плоскости. Вектор $\vec{a} \times \vec{b}$ ортогонален этой плоскости а значит и вектору $\vec{c}$. Следовтельно $\vec{a} \vec{b} \vec{c} = (\vec{a} \times \vec{b} ) \vec{c} = \vec{0}$.

        Достаточность. Если $\vec{a} \parallel \vec{b}$, то компланарность очевидна Пусть теперь $\vec{a} \nparallel \vec{b}$. Будем считать что вектора $\vec{a}, \vec{b}$ и $\vec{c}$ отложены от отдной и той же точки. Пусть $\vec{a} \vec{b} \vec{c} = 0$. Это означает что $(\vec{a} \times \vec{b}) \vec{c} = 0$. Следовательно $\vec{a} \times \vec{b}$ ортогонален вектору $\vec{c}$. Но вектор $\vec{a} \times \vec{b}$ ортогонален плоскости $\delta$, образованной векторами $\vec{a}$ и $\vec{b}$. Поскольку $\vec{c}$ ортогонален этому вектору, то он лежитт в $\delta$. А это означает, что вектора $\vec{a}, \vec{b}$ и $\vec{c}$ компланарны.
    \end{hproof}


    \begin{htheorem}
        \textbf{Теорема  (геометрический смысл смешанного произведения)} Объем параллелепипеда, построенного на трех некомпланарных векторах,
        равен модулю их смешанного произведения.
    \end{htheorem}


    \begin{hproof}
        \textbf{Доказательство}. Пусть $\vec{a}, \vec{b}$ и $\vec{c}$ --- три некомпланарных вектора. Предположим сначала, что тройка $(\vec{a}, \vec{b}, \vec{c})$ --- правая.

        \includegraphics[width=11cm]{t4}


        Отложим вектора $\vec{a}, \vec{b}$ и $\vec{c}$ от точки O. Пусть точка С такая, что $\overrightarrow{OC} = \vec{c}$, а D --- проекция точки C на плоскость векторов $\vec{a}$ и $\vec{b}$, которую обозначим через $\sigma$. Учитывая, что $\alpha + \beta = \frac{\pi}{2}$ и потому $\sin \alpha = \cos \beta$, и юзая геометрический смысл векторного произведения, имеем
        \begin{equation}
            V = S \cdot h =  | \vec{a} \times \vec{b} | \cdot | CD | = |\vec{a} \times \vec{b}| \cdot | \vec{c} | \cdot \sin \alpha = |\vec{a} \times \vec{b}| \cdot | \vec{c} | \cdot \cos \beta = ( \vec{a} \times \vec{b}) \vec{c} = \vec{a} \vec{b} \vec{c}.
        \end{equation}

        Пусть теперь тройка $\vec{a}, \vec{b}$ и $\vec{c}$ левая. Тогда $\alpha = \beta - \frac{\pi}{2}$, откуда $\sin \alpha = - \cos \beta$.

        \includegraphics[width=10cm]{t5}

        \begin{equation}
            V = S \cdot h = | \vec{a} \times \vec{b} | \cdot | CD | = | \vec{a} \times \vec{b} | \cdot | \vec{c} | \cdot \sin \alpha = - | \vec{a} \times \vec{b} | \cdot | \vec{c} | \cdot \cos \beta = - (\vec{a} \times \vec{b}) \vec{c} = -\vec{a} \vec{b} \vec{c}
        \end{equation}

        \begin{itemize}
            \item $\vec{a} \vec{b} \vec{c} = V > 0$, если тройка правая
            \item $\vec{a} \vec{b} \vec{c} = -V < 0$, если тройка левая
        \end{itemize}

    \end{hproof}


    Вот еще свойства (пусть $\vec{a}, \vec{b}, \vec{c}, \vec{d}$ - произвольные вектора, а $t$ --- число)
    \begin{itemize}
        \item $\vec{a} \vec{b} \vec{c} = \vec{b} \vec{c} \vec{a} = \vec{c} \vec{a} \vec{b} = -\vec{a} \vec{c} \vec{b} = -\vec{c} \vec{b} \vec{a} = -\vec{b} \vec{a} \vec{c}$
        \item $(t\vec{a}) \vec{b} \vec{c} = \vec{a} (t \vec{b}) \vec{c} = \vec{a} \vec{b} (t \vec{c}) = t (\vec{a} \vec{b} \vec{c})$
        \item $(\vec{a} + \vec{b}) \vec{c} \vec{d} = \vec{a} \vec{c} \vec{d} + \vec{b} \vec{c} \vec{d} $
        \item $\vec{a} (\vec{b} + \vec{c}) \vec{d} = \vec{a} \vec{b} \vec{d} + \vec{a} \vec{c} \vec{d}$
        item $\vec{a} \vec{b} (\vec{c} + \vec{d}) = \vec{a} \vec{b} \vec{c} + \vec{a} \vec{b} \vec{d}$
    \end{itemize}

    \newpage \begin{center}
                 \begin{Large}
                     \fbox{БИЛЕТ 12}
                 \end{Large}
    \end{center}

    \subsection*{Системы координат на плоскости и в пространстве}
    \textbf{Координатами} точки $М$ называются координаты её радиус-вектора.

    Точка $А$ делит отрезок $M_0 M_1$ внутренним образом в отношении $\lambda$, если $\displaystyle \frac{M_0 A}{A M_1} = \lambda$,\newline внешним образом, если $\displaystyle \frac{M_0 A}{A M_1} = -\lambda$

    \begin{equation}
        \displaystyle A \left( \frac{x_0 + \lambda x_1}{1 + \lambda}; \frac{y_0 + \lambda y_1}{1 + \lambda}; \frac{z_0 + \lambda z_1}{1+\lambda}\right)
    \end{equation}

    \newpage \begin{center}
                 \begin{Large}
                     \fbox{БИЛЕТ 13}
                 \end{Large}
    \end{center}

    \subsection*{Виды уравнений прямой на плоскости}

    \includegraphics[width=11cm]{t6}

    Любая точка на прямой может быть задана как $\vec{r} = \vec{r_0} + t\vec{a}, t \in \mathbb{R}, \vec{r_0} = (x_0, y_0), \vec{r} = (x,y), \vec{a} = (r, s)$

    Уравнения прямой:
    \begin{enumerate}


        \item Параметрическое: $
        \begin{cases}
            x = x_0 + tr,
            \\
            y = y_0 + ts
        \end{cases}$

        \item Каноническое: $\displaystyle \frac{x-x_0}{r} = \frac{y-y_0}{s}$
        \item Общее: $Ax + By + C = 0, A^2 + B^2 \neq 0$
    \end{enumerate}

    \textbf{Определение.} Пусть прямая $l$ задана уравнением $Ax + By + C = 0$. Тогда вектор $\vec{n} = (A, B)$ называется \textbf{главным вектором} прямой $l$.

    \begin{htheorem}
        \textbf{Замечание.} Главный вектор прямой не коллинеарен этой прямой.
    \end{htheorem}


    \begin{hproof}
        \textbf{Доказательство}. Пусть прямая $l$ задана уравнением $Ax + By + C = 0$, $\vec{n} = (A, B), M_0 (x_0, y_0) \in l$, то есть $Ax_0 + By_0 + C = 0$. Отложим вектор $\vec{n}$ от точки $M_0$. Концом соответствующего направленного отрезка будет точка $M_1(x_0 + A, y_0 + B)$. Подставив координаты этой точки в левую часть уравнения прямой, получим $A (x_0 + A) + B(y_0 + B) + C = Ax_0 + By_0 + C + A^2 + B^2 = A^2 + B^2 \neq 0$.

        Таким образом, $M_1 \notin l$. Поскольку $M_0 \in l$, а $\overrightarrow{M_0 M_1} = \vec{n}$, это означает, что вектор $\vec{n}$ и прямая $l$ не коллинеарны.

    \end{hproof}

    \newpage \begin{center}
                 \begin{Large}
                     \fbox{БИЛЕТ 14}
                 \end{Large}
    \end{center}
    \subsection*{Взаимное расположение прямых на плоскости}


    \begin{htheorem}
        \textbf{Теорема}. Пусть прямые $l_1$ и $l_2$ заданы уравнениями \begin{itemize}
                                                                            \item $A_1x+B_1y+C_1 = 0$
                                                                            \item $A_2x+B_2y+C_2 = 0$
        \end{itemize}. Тогда \begin{enumerate}
                                 \item $l_1$ и $l_2$ пересекаются $\displaystyle \Leftrightarrow \frac{A_1}{A_2} \neq \frac{B_1}{B_2}$
                                 \item $\displaystyle l_1 \parallel l_2$ и $\displaystyle l_1 \neq l_2 \Leftrightarrow \frac{A_1}{A_2} = \frac{B_1}{B_2} \neq \frac{C_1}{C_2}$
                                 \item $\displaystyle l_1 = l_2 \Leftrightarrow \frac{A_1}{A_2} = \frac{B_1}{B_2} = \frac{C_1}{C_2}$
        \end{enumerate}
    \end{htheorem}


    \begin{hproof}
        \textbf{Доказательство} (в ПСК).

        $\displaystyle \frac{A_1}{A_2} = \frac{B_1}{B_2}$ (условие параллельности нормальных векторов в ПСК).

        Направляющие вектора: $\vec{a_1} = (-B_1, A_1), \vec{a_2} = (-B_2, A_2)$. \newline $l_1 \parallel l_2$ или $\displaystyle l_1 = l_2 \Leftrightarrow a_1 \parallel a_2 \Leftrightarrow \frac{-B_1}{-B_2} = \frac{A_1}{A_2}$, откуда следует утверждение 1.

        Пусть $t = \frac{A_1}{A_2} = \frac{B_1}{B_2}$.

        $A_1 = tA_2$

        $B_1 = tB_2$

        $t \neq 0$, иначе $A_1 = B_1 = 0$

        Получаем $
        \begin{cases}
            A_1x+B_1y+C_1 = 0
            \\
            A_2x + B_2y+C_2 = 0
        \end{cases}
        $

        $
        \begin{cases}
            tA_2x+tB_2y+C_1 = 0
            \\
            A_2x + B_2y+C_2 = 0
        \end{cases}
        $

        $
        \begin{cases}
            tA_2x+tB_2y+C_1 = 0
            \\
            tA_2x + tB_2y+tC_2 = 0
        \end{cases}
        $

        $C_1 - tC_2 = 0$

        $C_1 = tC_2$, то есть только при $\displaystyle t=\frac{C_1}{C_2}$ система имеет решение, откуда следуют утверждения 2 и 3.
    \end{hproof}


    \newpage \begin{center}
                 \begin{Large}
                     \fbox{БИЛЕТ 15}
                 \end{Large}
    \end{center}
    \subsection*{Нормальное уравнение прямой на плоскости}
    ??????????????????????????????????????????????

    \subsection*{Отклонение точки от прямой}

    \begin{htheorem}
        \textbf{Теорема (о полуплоскостях)}. Пусть $M(x', y')$ --- точка плоскости. Если $M \in \lambda$, то $Ax' + By' + C > 0$, а если $M \in \mu $, то $Ax' + By' + C < 0$
    \end{htheorem}



    \includegraphics[width=11cm]{t7}

    \begin{hproof}
        \textbf{Доказательство}. Пусть $M \in \lambda$. Через точку М проведём прямую, коллинеарнубю $\vec{n}$. Мы знаем, что \textit{главный вектор прямой не коллинеарен этой прямой}. Значит наша прямая пересечёт $l$. Пусть точка пересечения это $N(x'', y'')$. Очевидно $Ax'' + By'' + C = 0$. $\overrightarrow{NM}$ и $\vec{n}$ сонаправлены, то есть $\overrightarrow{NM} = t \vec{n}, t>0$. Получаем, что $x' -x'' = tA, y'-y'' = tB \Rightarrow x' = x'' + tA, y' = y'' + tB \Rightarrow$ \newline \begin{equation}
                                                                                                                                                                                                                                                                                                                                                                                                                                                                                                                  Ax' + By' + C = A(x'' + tA) + B(y'' + tB) + C = Ax'' + By'' + C + t(A^2 + B^2) = t(A^2+B^2)>0
        \end{equation}
        Мы доказали, что если $M \in \lambda$, то $Ax' + By' + C > 0$.

        Ребят, ну давайте второе утверждение докажете сами плз.
        \newline
        Точки $P(x_1, y_1)$ и $Q(x_2, y_2)$ лежат по одну сторону от прямой $Ax+By+C=0$ тогда и только тогда, когда $sgn (Ax_1+By_1+C) = sgn (Ax_2+By_2+C)$ и по разные стороны, когда $sgn (Ax_1+By_1+C) \neq sgn (Ax_2+By_2+C)$
    \end{hproof}

    \newpage \begin{center}
                 \begin{Large}
                     \fbox{БИЛЕТ 16}
                 \end{Large}
    \end{center}
    \subsection*{Плоскость}

    $\sigma$ --- плоскость, $M_0(x_0, y_0, z_0)$ --- точка в $\sigma$, $\vec{a_1} = (q_1, r_1, s_1)$ и $\vec{a_2} = (q_2, r_2, s_2)$ --- направляющие вектора, не коллинеарные между собой.
    $\overrightarrow{M_0M_1} = u\vec{a_1} + v\vec{a_2}$, где $u, v \in \mathbb{R}$.


    Уравнения плоскости: \begin{enumerate}
                             \item Параметрическое; $
                             \begin{cases}
                                 x = x_0 + q_1u + q_2v
                                 \\
                                 y = y_0 + r_1u + r_2v
                                 \\
                                 z = z_0 + s_1u + s_2v
                             \end{cases}
                             $
                             \item Каноническое: $\begin{vmatrix}
                                                      x-x_0 & y-y_0 & z-z_0 \\
                                                      q_1   & r_1   & s_1   \\
                                                      q_2   & r_2   & s_2
                             \end{vmatrix} = 0$

                             \item Общее: из канонического можем получить $A = \begin{vmatrix}
                                                                                   r_1 & s_1 \\
                                                                                   r_2 & s_2
                             \end{vmatrix}$, $B = \begin{vmatrix}
                                                      q_1 & s_1 \\
                                                      q_2 & s_2
                             \end{vmatrix}$, $C = \begin{vmatrix}
                                                      q_1 & r_1 \\
                                                      q_2 & r_2
                             \end{vmatrix}$. Имеем $A(x-x_0)+B(y-y_0)+C(z-z_0) = 0$ и $(A,B,C) -$  главный вектор плоскости.
    \end{enumerate}

    \begin{htheorem}
        \textbf{Теорема}. Любая плоскость представима в виде уравнения $Ax+By+Cz+D=0$. И наоборот, любое уравнение $Ax+By+Cz+D=0$ задаёт плоскость.
    \end{htheorem}


    \begin{hproof}
        \textbf{Доказательство}. \begin{enumerate}
                                     \item Любая плоскость представима каноническим уравнением $\begin{vmatrix}
                                                                                                    x-x_0 & y-y_0 & z-z_0 \\
                                                                                                    q_1   & r_1   & s_1   \\
                                                                                                    q_2   & r_2   & s_2
                                     \end{vmatrix} = 0$\newline

                                     $\begin{vmatrix}
                                          r_1 & s_1 \\
                                          r_2 & s_2
                                     \end{vmatrix}(x-x_0)+ \begin{vmatrix}
                                                               q_1 & s_1 \\
                                                               q_2 & s_2
                                     \end{vmatrix} (y-y_0)+ \begin{vmatrix}
                                                                q_1 & r_1 \\
                                                                q_2 & r_2
                                     \end{vmatrix} (z-z_0) = 0$, где $A(x-x_0)+B(y-y_0)+C(z-z_0) = 0$ и $(A,B,C)$.

                                     \item Возьмём уравнение $Ax+By+Cz+D=0, A^2 + B^2 + C^2 \neq 0$.
                                     \begin{enumerate}
                                         \item Возьмём точку $(x_0, y_0, z_0)$, удовлетворяющую данному уравнению.

                                         Если $A\neq 0$, то берём $y_0 = z_0 = 0$ и получаем $\displaystyle x_0 = \frac{D}{A}$ (аналогично для $A=0$, тогда либо $B \neq 0$, либо $C \neq 0$.

                                         \item Возьмём 2 вектора:

                                         \begin{itemize}
                                             \item $\vec{a_1} = (-B, A, 0)$
                                             \item $\vec{a_2} = (-C, 0, A)$
                                         \end{itemize}

                                         Составим каноническое уравнение плоскости, проходящей через $M_0$ с направляющими векторами $a_1$ и $a_2$.

                                         $\begin{vmatrix}
                                              x-x_0 & y-y_0 & z-z_0 \\
                                              -B    & A     & 0     \\
                                              -C    & 0     & A
                                         \end{vmatrix} = 0$

                                         \begin{equation}
                                             A^2(x-x_0) + AB(y-y_0)+AC(z-z_0) = 0|:A \Rightarrow A(x-x_0) + B(y-y_0)+C(z-z_0) = 0
                                         \end{equation}
                                         \begin{equation}
                                             Ax+By+Cz - Ax_0 - By_0 -Cz_0 = 0
                                         \end{equation}
                                         Здесь $D = - Ax_0 - By_0 -Cz_0$.

                                     \end{enumerate}
        \end{enumerate}
    \end{hproof}

    \newpage \begin{center}
                 \begin{Large}
                     \fbox{БИЛЕТ 17}
                 \end{Large}
    \end{center}
    \subsection*{Взаимное расположение плоскостей}

    \begin{htheorem}
        Пусть плоскости заданы уравнениями \begin{enumerate}
                                               \item $\pi_1: A_1x+B_1y+C_1z+D_1 = 0$
                                               \item $\pi_2: A_2x+B_2y+C_2z+D_2 = 0$
        \end{enumerate}

        Тогда\begin{enumerate}
                 \item $\pi_1$ и $\pi_2$ пересекаются $\displaystyle \Leftrightarrow \frac{A_1}{A_2} \neq \frac{B_1}{B_2}$ или $\displaystyle \frac{A_1}{A_2} \neq \frac{C_1}{C_2}$

                 \item $\pi_1 \parallel \pi_2$ и $\pi_1 \neq \pi_2 \Leftrightarrow \displaystyle \frac{A_1}{A_2} = \frac{B_1}{B_2} = \frac{C_1}{C_2} \neq \frac{D_1}{D_2}$
        \end{enumerate}
    \end{htheorem}

    \begin{hproof}
        \textbf{Доказательство (в общем случае)}. Рассмотрим систему уравнений
        \begin{equation}
            \begin{cases}
                A_1x+B_1y+C_1z+D_1 = 0
                \\
                A_2x+B_2y+C_2z+D_2 = 0
            \end{cases}
        \end{equation}
        Пусть для определения $\displaystyle \frac{A_1}{A_2} \neq \frac{B_1}{B_2}$.

        Давайте сделаем $z_0=0$, тогда получим систему \begin{equation}
                                                           \begin{cases}
                                                               A_1x+B_1y = -D_1
                                                               \\
                                                               A_2x+B_2y= -D_2
                                                           \end{cases} (*)
        \end{equation}
        Эта система по правилу Крамера имеет единственное решение $(x_0, y_0)$. Значит невозможно $\pi_1 = \pi_2$.

        Если $\pi_1 = \pi_2$, то имеется другое решение системы, что противоречит с тем, что система $(*)$ имеет только одно решение.

        $\begin{cases}
             A_1x+B_1y+C_1z+D_1 = 0
             \\
             A_2x+B_2y+C_2z+D_2 = 0
        \end{cases}$

        $\displaystyle \frac{A_1}{A_2} = \frac{B_1}{B_2} = \frac{C_1}{C_2} = t$, откуда $\begin{cases}
                                                                                             tA_2x+tB_2y+tC_2z+D_1 = 0
                                                                                             \\
                                                                                             tA_2x+tB_2y+tC_2z+tD_2 = 0
        \end{cases} \Rightarrow D_1 = tD_2 = 0 \Rightarrow \displaystyle t = \frac{D_1}{D_2}$. Если это не так, то решений $\infty$.
    \end{hproof}


    \newpage \begin{center}
                 \begin{Large}
                     \fbox{БИЛЕТ 18}
                 \end{Large}
    \end{center}
    \subsection*{Нормальное уравнение плоскости}

    ?????????????????????????????????

    \subsection*{Отклонение точки от плоскости}

    \begin{htheorem}
        \textbf{Теорема (о полупространствах)}. Пусть $M(x', y', z')$ - произольная точка пространства. Если $M \in \lambda$, то $Ax' + By' + Cz' + D > 0$, а если $M \in \mu$, то $Ax' + By' + Cz' + D < 0$.

        Точки $P(x_1, y_1, z_1)$ и $Q(x_2, y_2, z_2)$ расположены по одну сторону от плоскости\newline $Ax' + By' + Cz' + D = 0$ тогда и только тогда, когда\newline $sgn(Ax_1 + By_1 +Cz_1+D) = sgn(Ax_2 + By_2 + Cz_2 + D)$, и по разные стороны, когда\newline $sgn(Ax_1 + By_1 +Cz_1+D) \neq sgn(Ax_2 + By_2 + Cz_2 + D)$.
    \end{htheorem}


    \newpage \begin{center}
                 \begin{Large}
                     \fbox{БИЛЕТ 19}
                 \end{Large}
    \end{center}
    \subsection*{Виды уравнений прямой в пространстве}

    Пусть на прямой $l$ лежит точка $M_0(x_0,y_0,z_0)$, $\vec{a} = (q,r,s) \neq \vec{0}$ - направляющий вектор прямой $l$. $\vec{r_0}$ --- радиус-вектор точки $M_0$.

    \includegraphics[width=10cm]{t8}

    Точка $M$ лежит на $l$ тогда и только тогда, когда $\vec{a}$ коллинеарен $\overrightarrow{M_0M}$, то есть $\overrightarrow{M_0M}  = t\vec{a}$.

    $M \in l \Leftrightarrow \vec{r} = \vec{r_0} + t\vec{a}$.

    Виды уравнений прямой в пространстве:
    \begin{enumerate}
        \item Векторное: $\vec{r} = \vec{r_0} + t\vec{a}$
        \item Параметрическое: $\begin{cases}
                                    x=x_0+qt
                                    \\
                                    y=y_0+rt
                                    \\
                                    z=z_0+st
        \end{cases}$
        \item Каноническое: $\displaystyle \frac{x-x_0}{q} = \frac{y-y_0}{r} = \frac{z-z_0}{s}$
        \item По двум точкам: $\displaystyle \frac{x-x_0}{x_1-x_0} = \frac{y-y_0}{y_1-y_0} = \frac{z-z_0}{z_1-z_0}$
        \item Общие уравнения (как пересечение двух плоскостей): $\begin{cases}
                                                                      A_1x+B_1y+C_1z+D_1=0,
                                                                      \\
                                                                      A_2x+B_2y+C_2z+D_2=0
        \end{cases}$
    \end{enumerate}

    \begin{htheorem}
        \textbf{Теорема}. Любая прямая в пространстве представима общим уравнением.
    \end{htheorem}

    \begin{hproof}
        \textbf{Доказательство.}

        У нас плосоксти пересекаются, поэтому нормальные векторы плоскостей непараллельны.

        Общий случай. Предположим $\displaystyle \frac{A_1}{A_2} \neq \frac{B_1}{B_2}$.

        Перепишем систему в виде $\begin{cases}
                                      A_1x+B_1y=-C_1z-D_1,
                                      \\
                                      A_2x+B_2y=-C_2z-D_2
        \end{cases}
        $ Зафиксируем $z$ и скажем, что $z=t$: $\displaystyle \begin{cases}
                                                                  A_1x+B_1y=-C_1t-D_1,
                                                                  \\
                                                                  A_2x+B_2y=-C_2t-D_2
        \end{cases}
        $

        Поскольку $\displaystyle \begin{vmatrix}
                                     A_1 & A_2 \\
                                     B_1 & B_2
        \end{vmatrix} \neq 0$, то при любом $t$ система имеет единственное решение по правилу Крамера: \begin{equation}
                                                                                                           \displaystyle
                                                                                                           \begin{cases}
                                                                                                               x=\frac{\begin{vmatrix}
                                                                                                                           -C_1t-D_1 & A_2 \\
                                                                                                                           -C_2t-D_2 & B_2
                                                                                                               \end{vmatrix}}{\begin{vmatrix}
                                                                                                                                  A_1 & A_2 \\
                                                                                                                                  B_1 & B_2
                                                                                                               \end{vmatrix}} = \frac{\displaystyle t(-B_2C_1+A_2C_2) - B_2D_1+A_2D_2}{\begin{vmatrix}
                                                                                                                                                                                           A_1 & A_2 \\
                                                                                                                                                                                           B_1 & B_2
                                                                                                               \end{vmatrix}} = \frac{\displaystyle -B_2D_1+A_2D_2}{\begin{vmatrix}
                                                                                                                                                                        A_1 & A_2 \\
                                                                                                                                                                        B_1 & B_2
                                                                                                               \end{vmatrix}} + t\frac{\displaystyle -B_2C_1+A_2C_2}{\begin{vmatrix}
                                                                                                                                                                         A_1 & A_2 \\
                                                                                                                                                                         B_1 & B_2
                                                                                                               \end{vmatrix}},
                                                                                                               \\
                                                                                                               y=\frac{\begin{vmatrix}
                                                                                                                           -A-1 & C_1t-D_1 \\
                                                                                                                           -B_1 & C_2t-D_2
                                                                                                               \end{vmatrix}}{\begin{vmatrix}
                                                                                                                                  A_1 & A_2 \\
                                                                                                                                  B_1 & B_2
                                                                                                               \end{vmatrix}} = \frac{\displaystyle t(-A_1C_2+B_1C_1) - A_1D_2+B_1D_1}{\begin{vmatrix}
                                                                                                                                                                                           A_1 & A_2 \\
                                                                                                                                                                                           B_1 & B_2
                                                                                                               \end{vmatrix}} = \frac{\displaystyle -A_1D_2+B_1D_1}{\begin{vmatrix}
                                                                                                                                                                        A_1 & A_2 \\
                                                                                                                                                                        B_1 & B_2
                                                                                                               \end{vmatrix}} + t\frac{\displaystyle -A_1C_2+B_1C_1}{\begin{vmatrix}
                                                                                                                                                                         A_1 & A_2 \\
                                                                                                                                                                         B_1 & B_2
                                                                                                               \end{vmatrix}},
                                                                                                               \\
                                                                                                               z = t
                                                                                                           \end{cases}
        \end{equation}

        \begin{equation}
            \displaystyle
            \begin{cases}
                x=\frac{\displaystyle -B_2D_1+A_2D_2}{\begin{vmatrix}
                                                          A_1 & A_2 \\
                                                          B_1 & B_2
                \end{vmatrix}} + t\frac{\displaystyle -B_2C_1+A_2C_2}{\begin{vmatrix}
                                                                          A_1 & A_2 \\
                                                                          B_1 & B_2
                \end{vmatrix}},
                \\
                y=\frac{\displaystyle -A_1D_2+B_1D_1}{\begin{vmatrix}
                                                          A_1 & A_2 \\
                                                          B_1 & B_2
                \end{vmatrix}} + t\frac{\displaystyle -A_1C_2+B_1C_1}{\begin{vmatrix}
                                                                          A_1 & A_2 \\
                                                                          B_1 & B_2
                \end{vmatrix}}
                \\ z = t
            \end{cases}
        \end{equation}


        В ПСК можно доказать обратное: любое уравнение задаёт некоторую прямую\newline $\begin{cases}
                                                                                            A_1x+B_1y+C_1z+D_1=0,
                                                                                            \\
                                                                                            A_2x+B_2y+C_2z+D_2=0
        \end{cases}$.

        Поскольку плоскости непараллельны, то пусть $\displaystyle \frac{A_1}{A_2} \neq \frac{B_1}{B_2}$. Тогда берём $z=0$ и получаем систему $\begin{cases}
                                                                                                                                                    A_1x+B_1y=-D_1,
                                                                                                                                                    \\
                                                                                                                                                    A_2x+B_2y=-D_2
        \end{cases}$, которая по правилу Крамера имеет единственное решение $(x_0, y_0)$.\newline Таким образом, точка с координатами $M(x_0, y_0, z_0)$ лежит на данной прямой.

        $\vec{a} = \vec{n_1} \times \vec{n_2}$

        $\vec{a_1} \perp \vec{n_1}, \vec{a_2} \perp \vec{n_2}$

        Таким образом, из уравнения плоскостей мы получаем напраляющий вектор данной прямой. По точке $M_0$ и направляющему вектору мы сможем восстановить прямую:
        \begin{equation}
            \displaystyle \frac{x-x_0}{B_1C_2-B_2C_1} = \frac{y-y_0}{A_1C_2-A_2C_1} = \frac{z}{A_2B_2-A_2B_1}
        \end{equation}
    \end{hproof}

    \newpage \begin{center}
                 \begin{Large}
                     \fbox{БИЛЕТ 20}
                 \end{Large}
    \end{center}
    \subsection*{Взаимное расположение прямых в пространстве}
    Пусть даны $l_1$ и $l_2$, а $\vec{a_1} = (q_1, r_1, s_1)$ и $\vec{a_2} = (q_2, r_2, s_2)$ --- направляющие векторы для этих прямых соответственно. Возьмём по одной точке $M_1(x_1, y_1, z_1)$ и $M_2(x_2, y_2, z_2)$ с каждой прямой.

    Если прямые \textit{лежат в одной плоскости (либо совпадают, либо пересекаются)}, то смешанное произведение $\overrightarrow{M_1M_2}, \vec{a_1}, \vec{a_2}$ компланарны, то есть смешанное произведение равно нулю: $\displaystyle \begin{vmatrix}
                                                                                                                                                                                                                                           x_2-x_2 & y_2-y_2 & z_2-z_1 \\
                                                                                                                                                                                                                                           q_1     & r_1     & s_1     \\
                                                                                                                                                                                                                                           q_2     & r_2     & s_2
    \end{vmatrix} = 0$.

    Прямые \textit{скрещиваются}$\displaystyle \Leftrightarrow \begin{vmatrix}
                                                                   x_2-x_2 & y_2-y_2 & z_2-z_1 \\
                                                                   q_1     & r_1     & s_1     \\
                                                                   q_2     & r_2     & s_2
    \end{vmatrix} \neq 0$.

    Прямые \textit{параллельны или совпадают}: $\displaystyle \vec{a_1} \parallel \vec{a_2} \Leftrightarrow \frac{q_1}{q_2} = \frac{r_1}{r_2} = \frac{s_1}{s_2}$.

    \newpage \begin{center}
                 \begin{Large}
                     \fbox{БИЛЕТ 21}
                 \end{Large}
    \end{center}
    \subsection*{Взаимное расположение прямой и плоскости в пространстве}

    \begin{htheorem}
        \textbf{Теорема}. Предположим, что дана плоскость $Ax+By+Cz+D=0$, где $A^2+B^2+C^2+D^2\neq 0$, и прямая $\displaystyle l: \begin{cases}
                                                                                                                                      x=x_0+qt,
                                                                                                                                      \\
                                                                                                                                      y=y_0+rt,
                                                                                                                                      \\
                                                                                                                                      z=z_0+st
        \end{cases}$. \textbf{Тогда прямая и плоскость пересекаются $\mathbf{\Leftrightarrow Aq+Br+Cs \neq 0}$}
    \end{htheorem}

    \begin{hproof}
        $A(x_0 + qt) + B(y_0 + rt) + C(z_0 + st) = 0 \Leftrightarrow \vec{n} \perp \vec{a}$

        $Ax_0 + By_0 + Cz_0 + D + (Aq+Br+Cs)t=0$

        Если $Aq+Br+Cs \neq 0$, то решение единственное: $\displaystyle t = \frac{-(Ax_0 + By_0 + Cz_0 + D)}{Aq+Br+Cs}$ и прямая с плоскостью имеют одну общую точку.
    \end{hproof}

    Итак: \begin{itemize}
              \item $l$ лежит в плоскости $\Leftrightarrow \begin{cases}
                                                               Ax_0 + By_0 + Cz_0 + D = 0,
                                                               \\
                                                               Aq+Br+Cs=0,
              \end{cases}$
              \item $l$ параллельна плоскости $\Leftrightarrow \begin{cases}
                                                                   Ax_0 + By_0 + Cz_0 + D \neq 0,
                                                                   \\
                                                                   Aq+Br+Cs=0,
              \end{cases}$
              \item $l$ пересекается с плоскостью $\Leftrightarrow Aq+Br+Cs\neq 0$
    \end{itemize}


    \newpage \begin{center}
                 \begin{Large}
                     \fbox{БИЛЕТ 22}
                 \end{Large}
    \end{center}
    \subsection*{Построение поля комплексных чисел}
    Пусть $a,b \in \mathbb{R}$.

    Рассмотрим множесто пар вида $(a,b)$.

    Введем операции сложения и умножения $z_1 + z_1 = (x_1+x_2,y_1+y_2), z_1 z_2 = (x_1 x_2 -y_2 y_2, x_1 y_2 + x_2 y_1)$

    \begin{htheorem}
        \textbf{Теорема}. Относительно введенных операций множество $\mathbb{C}$ является полем:
    \end{htheorem}

    \begin{hproof}
        \begin{itemize}
            \item
            сложение:\begin{itemize}
                         \item коммутативность
                         \item ассоциативность
                         \item $(0,0)$ --- нейтральный
                         \item $(-x,-y)$ --- противоположный элемент
            \end{itemize}

            \item
            умножение: \begin{itemize}
                           \item коммутативность
                           \item ассоциативность
                           \item $(1,0)$ --- нейтральный
                           \item обратный элемент существует, если $(x,y) \neq (0,0)$ или $\displaystyle x^2+y^2 \neq 0$: $z^{-1} = \left( \frac{x}{x^2+y^2},-\frac{y}{x^2+y^2} \right)$
            \end{itemize}


        \end{itemize}
    \end{hproof}

    \subsection*{Алгебраическая форма комплексного числа}
    $(x,y) = (x,0) + y(0,1) = x+iy$ --- общепринятая запись комплексного числа.

    $x+iy$ (x --- вещественная часть, y --- мнимая часть)

    Для каждого $z = x+iy$ существует $\overline{z} = x-iy$ --- сопряжённое.

    \newpage \begin{center}
                 \begin{Large}
                     \fbox{БИЛЕТ 23}
                 \end{Large}
    \end{center}
    \subsection*{Тригонометрическая форма комплексного числа}

    Любая точка плоскости однозначно задается парой $(r, \phi)$, где $r$ --- расстояние от точки до начала координат.

    $\displaystyle \begin{cases}
                       x = r \cos \phi,
                       \\
                       y = r \sin \phi
    \end{cases} \Rightarrow z = |z| (\cos \phi + i \sin \phi)$

    $\phi$ называется аргументом числа $z$ ($\phi = argZ$)

    \subsection*{Действия с числами в тригонометрической форме}

    При умножении комплексных чисел модули умножаются, а углы складываются.

    При делении модули делятся, а углы вычитаются.



    \subsection*{Формула Муавра}
    $z^n = (r(\cos \phi + i \sin \phi))^n = r^n ( \cos(n \phi) + i \sin(n \phi)$

    \newpage \begin{center}
                 \begin{Large}
                     \fbox{БИЛЕТ 24}
                 \end{Large}
    \end{center}

    \subsection*{Извлечение корней из комплексных чисел
    }
    \textbf{Определение}. Корнем n-ой степени комплексного числа z называется число $w$ такое,\newline что $w^n = z$.


    Пусть $z = r(\cos \phi + i \sin \phi)$

    $w = \rho (\cos \psi + i \sin \psi)$

    У нас должно быть $w^n = z$: $w^n = \rho^n (\cos (n \psi) + i \sin (n \psi))$

    $\rho^n (\cos (n \psi) + i \sin (n \psi)) = r (\cos  \phi + i \sin \phi)$

    Числа равны, а значит равны их модули $\rho^n = r \Rightarrow \rho = \sqrt[n]{r}$

    $\cos (n \psi) + i \sin (n \psi) = \cos  \phi + i \sin \phi \Rightarrow \begin{cases}
                                                                                \cos (n \psi) = \cos \phi,
                                                                                \\
                                                                                \sin (n \psi) = \sin \phi
    \end{cases}$.

    Если у углов одинаковы $\cos$ и $\sin$, то углы различаются на $2\pi k, k \in \mathbb{Z}$:
    \begin{equation}
        n \psi = \phi + 2\pi k \Rightarrow \psi = \frac{\phi + 2 \pi k}{n} = \frac{\phi}{n} + \frac{2 \pi k}{n}
    \end{equation}

    $\displaystyle \psi_0 = \frac{\phi}{n}$

    $\displaystyle \psi_1 = \frac{\phi}{n} + \frac{2\pi}{n}$

    $\displaystyle \psi_2 = \frac{\phi}{n} + \frac{4\pi}{n}$

    ...

    $\displaystyle \psi_{n-1} = \frac{\phi}{n} + \frac{2\pi}{n}n = \frac{\phi}{n} + 2\pi$

    Таким образом, получаем, что аргументов для $w$, дающих разные корни $n-1$ степени в точности $n$ штук при $k =0, 1, ..., n-1$

    $\psi = \frac{\phi}{n} + \frac{2\pi k}{n}, k = 0, 1, ..., n-1$.


    Пусть $z = r(\cos \phi + i \sin \phi)$

    \begin{equation}
        \displaystyle w = \sqrt[n]{z} = \sqrt[n]{r} \left( \cos \frac{\phi + 2\pi k}{n} + i \sin \frac{\phi + 2\pi k}{n} \right), k = 0,1,...,n-1
    \end{equation}

    \textbf{В поле комплексных чисел любое комплексное число $z \neq 0$ имеет в точности $n$ корней $n$-ой степени (предыдущая формула)}

    \textit{Пример. $\mathit{1 = \cos 0 + i \sin 0}$}

    $\displaystyle \sqrt[3]{1} = \cos \frac{0+2\pi k}{3} + i \sin \frac{0 + 2\pi k}{3} = \cos \frac{2 \pi k}{3} + i \sin \frac{2 \pi k}{3}, k = 0,1,2$

    $\displaystyle w_0 = \cos \frac{2 \pi 0}{3} + i \sin \frac{2 \pi 0}{3} = 1$

    $\displaystyle w_1 = \cos \frac{2 \pi}{3} + i \sin \frac{2 \pi}{3} = -\frac{1}{2} + i\frac{\sqrt{3}}{2}$

    $\displaystyle w_2 = \cos \frac{4 \pi}{3} + i \sin \frac{4 \pi}{3} = -\frac{1}{2} - i\frac{\sqrt{3}}{2}$\\

    \textit{Пример. Корни n-ой степени из 1}

    В случае $\mathbb{R}$: $\sqrt[n]{1} = 1$

    В случае $\mathbb{C}$ мы имеем n корней:

    $1 = 1(\cos 0 + i \sin 0)$
    \begin{equation}
        \displaystyle \sqrt[n]{1} = \sqrt[n]{1} \left( \cos \frac{2\pi k}{n} + i \sin \frac{2 \pi k}{n} \right), k=0,1,...,n-1
    \end{equation}


    При $k=0$: $w_0 = 1$

    При $k=1$: $\displaystyle w_1 = \cos \frac{2\pi}{n} + i \sin \frac{2\pi}{n}$

    $\displaystyle w_k = \cos \frac{2\pi k}{n} + i \sin \frac{2 \pi k}{n} = \left( \cos \frac{2\pi}{n} + i \sin \frac{2 \pi}{n} \right)^k$


    \newpage \begin{center}
                 \begin{Large}
                     \fbox{БИЛЕТ 25}
                 \end{Large}
    \end{center}
    \subsection*{Линейное пространство}
    \textbf{Определение}. Множество $V$ называется \textbf{линейным пространстом} над полем $\mathbb{F}$, если для каждой пары элементов V определена операция сложения и для каждого элемента $x \in V$ определена операция умножения на число $t \in \mathbb{F}$. При этом элементы V называются векторами, а элементы $\mathbb{F}$ называются скалярами.

    Аксиомы:
    \begin{enumerate}
        \item $x+(y+z) = (x+y)+z$
        \item $x+y=y+x$
        \item $\exists 0: \forall x \in V 0 + x = x + 0 = x$
        \item $\forall x \exists -x: x + (-x) = -x + x = 0$
        \item $t(x+y) = tx + ty$
        \item $(t+s)x = tx + sx$
        \item $t(sx) = (ts)x$
        \item $1x = x$
    \end{enumerate}


    Свойства:
    \begin{enumerate}
        \item Нулевой вектор единственный.
        \item Противоположный элемент единственный
    \end{enumerate}

    Примеры линейных пространств:
    \begin{enumerate}
        \item Множество векторов плоскости
        \item Множество векторов пространства
        \item Множество последовательностей длиный $n$ из элементов $\mathbb{R}$
        \item Многочлены от одной переменной, степени которых не превосходят n
        \item Матрицы $n \cdot m$
    \end{enumerate}

    \newpage \begin{center}
                 \begin{Large}
                     \fbox{БИЛЕТ 26}
                 \end{Large}
    \end{center}
    \subsection*{Линейная зависимость векторов}
    \begin{htheorem}
        \textbf{Лемма}. Если система векторов $x_1, x_2, ..., x_m$ содержит нулевой вектор, то она линейно зависима.
    \end{htheorem}

    \begin{hproof}
        \textbf{Доказательство}.

        Пусть $\vec{x_k} = \vec{0}$
        Тогда можно взять линейную комбинацию $0x_1 + 0x_2 + ... + 1x_k + ... + 0x_m = \vec{0} \Rightarrow$ система векторов линейно зависима чтд.

    \end{hproof}


    \begin{htheorem}
        \textbf{Лемма}. Если к линейно зависимой системе добавить новые векторы, то система останется линейно зависимой.
    \end{htheorem}

    \begin{hproof}
        \textbf{Доказательство}.

        Пусть  $x_1, x_2, ..., x_m$ --- линейно зависимая система. Тогда некоторый $\vec{x_k}$ выражается через остальные $x_1, ..., x_{k-1}, x_{k+1}, ..., x_m$. Если мы добавим еще какие-то векторы в системы, то $\vec{x_k}$ всё равно будет выражаться через остальные вектора. Пусть мы добавили векторы $y_1, y_2, y_l$. Тогда $\vec{x_k} = t_1x_1 + ... + t_mx_m + 0y_1 +0y_2 + ... + 0y_l$ чтд.
    \end{hproof}

    \begin{htheorem}
        \textbf{Лемма}. Если система ненулевых векторов $a_1, ..., a_m$ линейно зависима, то найдется $a_k$ который выражается через предыдущие векторы.
    \end{htheorem}

    \begin{hproof}
        \textbf{Доказательство}. По условию существуют скаляры $t_1, t_2, ..., t_k$, по крайней мере один из которых не равен 0, такие, что $t_1a_1 + t_2a_2 + ... + t_ka_k = 0$.

        Пусть j --- наибольший индекс, для которого $t_j \neq 0$. Если $j=1$, то равенство \newline  $t_1a_1 + t_2a_2 + ... + t_ka_k = 0$ сводится к $t_1a_1 = 0$, откуда $a_1 = 0$, противоречие. Тогда $j>1$.

        Перенося последнее слагаемое в другую часть и деля на $t_j \neq 0$, получаем \begin{equation}
                                                                                        \displaystyle a_j = -\frac{t_1}{t_j}\cdot a_1 - ... - \frac{t_{j-1}}{a_j} \cdot a_{j-1}
        \end{equation}
        чтд браток.
    \end{hproof}

    \newpage \begin{center}
                 \begin{Large}
                     \fbox{БИЛЕТ 27}
                 \end{Large}
    \end{center}
    \subsection*{Системы образующих}
    \textbf{Определение}. Система векторов $\Sigma$ векторного пространства $V$ называется \textbf{системой образующих} этого пространства, если любой вектор из V линейно выражатеся через какие-то вектора из системы $\Sigma$.

	\begin{htheorem}
		\textbf{Лемма о прополке.} Если $\Sigma$ - система образующих векторного пространства $V$ и $a \in \Sigma$ линейно выражается через другие вектора системы $\Sigma$, то и $\Sigma \backslash{} \{ a \}$ является системой образующих пространства $V$.
	\end{htheorem}
	
	\begin{hproof}
		\textbf{Доказательство.}
		
		$x = t_1a_1 + ... + t_ka_k$. Если среди $a_i$ нет $a$, то чтд. Иначе вместо $a_i$ подставим его выражение через другие вектора системы $\Sigma$.
	\end{hproof}

	\begin{htheorem}
		\textbf{Теорема о существовании конечного базиса.} Если в ненулевом векторном пространстве $V$ есть конечная система образующих, то в $V$ есть и конечный базис.
	\end{htheorem}
	
	\begin{hproof}
		\textbf{Доказательство.} 
		
		Если $\Sigma$ - конечная система образующих, то по лемме о прополке, пока в ней есть векторы, линейно выражающиеся через другие, мы их можем выбрасывать. В какой-то момент таких векторов не останется и система станет линейно независимой.
	\end{hproof}


    \subsection*{Базис линейного пространства}
    \textbf{Определение. Базисом} векторного пространства называется линейно независимая
    система образующих.

    \begin{htheorem}
        \textbf{Замечание 1.} Система векторов $e_1, e_2, ..., e_n$ линейно независима.
    \end{htheorem}


    \begin{hproof}
        \textbf{Доказательство.} Предположим, что $x_1e_1+x_2e_2 + ... + x_ne_n = 0$, для некоторых $x_1, ..., x_n \in \mathbb{F}$.

        $x_1e_1+x_2e_2 + ... + x_ne_n = (x_1, x_2, ..., x_n)$, то есть $(x_1, x_2, ..., x_n) = 0 \Leftrightarrow x_1 = x_2 = ... = x_n = 0$
    \end{hproof}

    \begin{htheorem}
        \textbf{Замечание 2.} Если $x = (x_1, x_2, ..., x_m) $ --- произвольный вектора из $F^n$, то \newline $x = x_1e_1 + x_2e_2 + ... + x_ne_n$
    \end{htheorem}


    \begin{htheorem}
        \textbf{Замечание 3.} Вектора $e_1, e_2, ..., e_n$ образуют базис пространства $F^n$.
    \end{htheorem}

    \begin{hproof}
        \textbf{Доказательство.} В силу 1 и 2 замечания эти вектора линейно независимы и являются системой образующих пространства $F^n$.
    \end{hproof}

    \textbf{Определение}. Система векторов $e_1, e_2, ..., e_n$ называется \textbf{стандартным базисом} пространства $F^n$.

    \subsection*{Равномощность базисов}

    \begin{htheorem}
        \textbf{Теорема}. Если в векторном пространстве есть базис из $n$ векторов, то и любой базис этого пространства содержит ровно $n$ векторов.
    \end{htheorem}

    \begin{hproof}
        \textbf{Доказательство.} Пусть $A := (a_1, a_2, ..., a_n) --- $ базис пространства, а $B := (b_1, b_2, ..., b_k)$ --- другой базис. Чтобы доказать, что $n=k$, в силу симметрии достаточно проверить, что $k \leq n$. Пусть $k > n$.

        Рассмотрим систему $b_1, a_1, a_2, ..., a_n$.

        Это линейно зависимая система ненулевых векторов, так как вектор $b_1$ выражается через систему образующих $А$. По лемме о правом крайнем в $b_1, a_1, a_2, ..., a_n$ есть вектор, который линейно выражается через предыдущие. Это не может быть вектор $b_1$ --- у него нет предыдущих. Значит, это какой-то $a_i$. Выкинув его, получим систему $b_1, a_1, a_2, ..., a_{i-1}, a_{i+1}, ..., a_n $ (*), которая останется системой образующих согласно лемме о прополке.

        Теперь рассмотрим $b_2, b_1, a_1, a_2, ..., a_n$ (**).

        Это линейно независимая система ненулевых векторов, так как вектор $b_2$ выражается через систему образующих (*). По лемме о правом крайнем в (**) есть вектор, выражаюшийся через предыдущие. Это не может быть ни $b_1$, ни $b_2$ (у $b_2$ нет предыдущих, а $b_1$ не выражается через $b_2$, так как система B линейно независимая). Значит, это какой-то $a_j, j \neq i$.

        Выкинув его из (**), получим систему $b_2, b_1, a_1, a_2, ..., a_{i-1}, a_{i+1}, ..., a_{j-1}, a_{j+1}, ...,a_n$, которая останется системой образующих согласно лемме о прополке. Продолжая добавлять вектора из B и удалять вектора из А, будем получать системы из n образующих, в которых всё больше векторов из B и всё меньше из А. Поскольку $k>n$, то через n шагов мы придём к системе образующих $b_n, ..., b_2, b_1$. Но тогда вектор $b_{n+1}$ выражается через эту систему обазующих, что противоречит линейной незавимисости B. чтд браток.
    \end{hproof}

    \begin{htheorem}
        \textbf{Следствия.}
        \begin{itemize}
            \item Если у векторного пространства $V$ есть система из n образующих, то любая линейно независимая система в $V$ содержит не больше n векторов.
            \item  Если в $V$ есть линейно независимая система из n векторов, то любая система образующих пространства $V$ содержит не менее n векторов.
        \end{itemize}
    \end{htheorem}




    \subsection*{Размерность пространства}
    Если у векторного пространства есть конечный базис, то число векторов в базисе называется \textbf{размерностью} этого пространства.
    Размерность пространства $V$ обозначается через $dim V$.

    \begin{htheorem}
        \textbf{Теорема о разложении вектора по базису}. Пусть $V$ --- ненулевое векторное пространство, $a_1, a_2, ..., a_n$ --- базис. Тогда $\forall x \in V$ существуют единственные $t_1, t_2, ..., t_n$ такие, что $x = t_1a_1 + t_2a_2 + ... + t_na_n (*)$.
    \end{htheorem}


    \begin{hproof}
        \textbf{Доказательство.} Сущестование $t_1, t_2, ..., t_n$ ясно, поскольку базис --- система образующих. Предположим, что наравне с (*) выполняется $x = s_1a_1 + s_2a_2 + ... + s_na_n$ для некоторых скаляров $s_i$. Вычтем одно равенство из другого и получим: \begin{equation}
        (t_1-s_1)
                                                                                                                                                                                                                                                                             a_1 + (t_2-s_2)a_2 + ... + (t_n-s_n)a_n = 0
        \end{equation}

        Поскольку вектора $a_i$ линейно независимы, получаем, $t_i-s_i = 0 \Rightarrow t_i = s_i$ чтд браток.
    \end{hproof}



    \subsection*{Координаты вектора}
    \textbf{Определение}. Равенство $x = t_1a_1 + t_2a_2 + ... + t_na_n$ называется \textbf{разложением вектора} $x$ \textbf{по базису} $a_1, a_2, ..., a_n$. Скаляры $t_1, t_2, ..., t_n$ называются координатами вектора $x$ в базисе $a_1, a_2, ..., a_n$. Записывается $\vec{x} = (t_1, t_2, ..., t_n)$.

    \subsection*{Действия с векторами в координатной форме}


    \newpage \begin{center}
                 \begin{Large}
                     \fbox{БИЛЕТ 28}
                 \end{Large}
    \end{center}
    
    \subsection*{Подпространства линейного пространства}
    \textbf{Определение.} Непустое подмножество $M$ векторного пространства $V$ над полем $F$ называется \textbf{подпространством} пространства $V$, если выполняются следующие условия:
    \begin{enumerate}
        \item если $x,y \in M$, то $x+y \in M$ (замкнутость подпространства относительно сложения векторов).
        \item если $x \in M, t \in F$ то $tx \in M$ (замкнутость подпространства относительно умножения вектора на скаляр).
    \end{enumerate}

	Примеры:
	\begin{itemize}
		\item $V$ --- подпространство $V$.
		\item $\{ 0 \}$ - подпространство $V$.
		\item $V$ --- трёхмерное пространство, $M$ - множество векторов, коллинеарных одной плоскости.
		\item В $F^n$ подпространством будет, например, множество строк, у которых $i$-тая координата равна 0.
		\item В пространстве функций $\mathbb{R} \rightarrow \mathbb{R}$ --- множество непрерывных функций.
	\end{itemize}

	\begin{htheorem}
		\textbf{Теорема.} $M$ --- линейное пространство $V \Leftrightarrow M$ --- подпространство.
	\end{htheorem}
	
	\begin{hproof}
		\textbf{Доказательство}
		 
		$\Rightarrow$.
		\begin{enumerate}
			\item $\forall x, y \in M \quad x+y \in M$
			\item $\forall x \in M, t \in F \quad tx \in M$
		\end{enumerate}
		
		Надо проверить, что $\vec{0} \in M$ и $\forall x \in M \quad (-x) \in M$.
		
		При $t=0 \quad \forall x \in M \quad 0 \cdot x = \vec{0} \in M$.
		
		При $t=-1 \quad \forall x \in M \quad (-1)x = -x \in M$.
		
		$\Leftarrow$ следует из того, что операции не выводят за пределы линейного пространства.
	\end{hproof}	
	
	\textbf{Определение.} Пусть дано множество векторов $A = \{ a_1, ..., a_s \}$, не обязательно линейно независимых. Множество линейных комбинаций данного множества $A$ называется \textbf{линейной оболочкой} множества $A$ и обозначается $<A> = <a_1, ..., a_s>$.
	
	\begin{htheorem}
		\textbf{Утверждение.} Линейная оболочка множества $A$ векторов $<A> = <a_1, ..., a_n>$
		 \begin{enumerate}
		 	\item является подпространством, в котором $A$ является системой образующих.
		 	\item $<A>$ является наименьшим линейным подпространством, содержащим множество $A$.
		 \end{enumerate}
	\end{htheorem}
	
	\begin{hproof}
		\textbf{Доказательство.} 
		\begin{enumerate}
			\item $x, y \in A \Rightarrow x = t_1a_1 + ... + t_sa_s, y = u_1a_1 + ... + u_sa_s, x+y = (t_1+u_1)a_1 + ... + (t_s+u_s)a_s \in A$
			
			$x \in A, t \in F \Rightarrow tx \in A$
			
			\item $A \subseteq M \Rightarrow a_1, ..., a_s \in M \Rightarrow$ все линейные комбинации векторов $a_1, ..., a_s$ тоже лежат в $M$. Тогда $<A> \subseteq M$. Убито!
		\end{enumerate}
	\end{hproof}
	
    \textbf{Определение}. Векторные пространства $V_1$ и $V_2$ над одним и тем же полем $F$ изоморфны, если существует биекция $f$ из $V_1$ на $V_2$ (называемая изоморфизмом) такая, что $f$ сохраняет операции, т.е.


    \begin{equation}
        \forall x_1, x_2 \in V_1 \forall t \in F \quad f(x_1+x_2) = f(x_1) + f(x_2) \quad \& \quad f(tx) = t \cdot f(x)
    \end{equation}

    \begin{htheorem}
        \textbf{Теорема об изоморфизме векторных пространств}. Любое $n$-мерное векторное пространство $V$ над полем $F$ изоморфно
        пространству $F^n$.
    \end{htheorem}

    \begin{hproof}
        \textbf{Доказательство.} Пусть $a_1, a_2, ..., a_n$ --- базис пространства $V, b \in V, (t_1, ..., t_n) --- $ координаты вектора $b$ в этом базисе. Определим отображение $f: V \rightarrow F^n$ правилом: $f(b) := t_1, ..., t_n)$. Поскольку координаты определяют вектор однозначно, то отображение $f$ инъективно. Сюръективность $f$ очевидна: если $y = (s_1, ..., s_n) \in F^n$, то $y = f(x)$, где $x = s_1a_1 + s_2a_2 + ... + s_na_n$.

        Наконец, сохранение операций вытекает из замечания о координатах суммы векторов и произвежения вектора на скаляр.

        Таким образом, $f$ --- изоморфизм из $V$ на $F^n$. чтд браток.
    \end{hproof}

    \subsection*{Операции над подпространствами и их свойства}

    Нулевой вектор содержится в любом подпространстве M пространства V.
    \textbf{Доказательство.} Если $x$ --- произвольный вектор из М, то по второму условию из определения подпространства $0 = 0 \cdot x \in M$.

    \begin{htheorem}
        \textbf{Замечание о подпространстве, порождённом набором векторов}. Пусть V --- векторное пространство и $a_1, a_2, ..., a_k \in V$. Тогда $<a_1, ..., a_k>$ --- наименьшее подпространство пространства V, содержащее вектора $a_1, ..., a_k$.
    \end{htheorem}


    \begin{hproof}
        \textbf{Доказательство}. Пусть $M$ – подпространство пространства $V$, содержащее вектора $a_1, a_2, ..., a_k$. По определению подпространства любая линейная комбинация векторов$a_1, a_2, ..., a_k$ лежит в $M$. Следовательно, $<a_1, ..., a_k> \subseteq M$. чтд браток.
    \end{hproof}


    \begin{htheorem}
        \textbf{Предложение о размерности подпространства}.
        Пусть $M$ - подпространство векторного пространства $V$. Тогда
        $dim M \leq dim V$ , причем $dim M = dim V$ тогда и только тогда, когда
        M = V.
    \end{htheorem}

    \begin{hproof}
        \textbf{Доказательство}. Если $M$ или $V$ –-- нулевое пространство, то оба
        утверждения предложения выполняются тривиальным образом. Будем
        поэтому считать, что $M$ и $V$ – ненулевые пространства. Пусть $dim M = k$,
        $dim V = n$. Неравенство $k \leq n$ следует из того, что базис $M$ –-- это линейно
        независимая система в $V$, а любую линейно независимую систему
        векторов из $V$ можно дополнить до базиса $V$ по теореме о продолжении. При этои для дополнения нужно $n-k$ векторов. Поэтому если $n=k$, то базис $M$ уже является базисом $V$ , т.е. $M = V$ . Обратное утверждение очевидно.
    \end{hproof}

    \textbf{Определение}. Пусть $V$ – векторное пространство, а $M_1$ и $M_2$ – его подпространства.
    Сумма подпространств $M_1$ и $M_2$ – это множество $M_1$ + $M_2$ всех сумм
    векторов из $M_1$ с векторами из $M_2$:
    \begin{equation}
        M_1+M_2 := \{ x_1 + x_2| x_1 \in M_1, x_2 \in M_2 \}
    \end{equation}

    \begin{htheorem}
        \textbf{Замечание о сумме и пересечении подпространств}. Если $M_1$ и $M_2$ --- подпространства $V$, то $M_1+M_2$ и $M_1 \cap M_2$ также являются подпространствами $V$.
    \end{htheorem}

    \begin{hproof}
        \textbf{Доказательство.} В силу замечания о нулевом векторе и подпространствах,
        каждое из подпространств $M_1$ и $M_2$ содержит нулевой вектор. Следовательно, $0=0+0 \in M_1+M_2$ и $0 \in M_1 \cap M_2$. В частности, множества $M_1+M_2$ и $M_1 \cap M_2$ непустые.

        Пусть $x, y \in M_1+M_2$ и $t$ --- скаляр. Тогда $x=x_1+x_2, y=y_1+y_2$ для некоторых $x_1, y_1 \in M_1$ и $x_2, y_2 \in M_2$. Получаем \begin{equation}
                                                                                                                                                  x+y = (x_1+x_2) + (y_1+y_2) = (x_1+y_1)+(x_2+y_2) \in M_1 + M_2, tx = t(x_1 + x_2) = tx_1 + tx_2 \in M_1 + M_2
        \end{equation}

        Итак, $M_1+M_2$ --- подпространство в $V$. Далее пусть $x, y \in M_1 \cap M_2$ и $t$ --- скаляр. Тогда $x, y \in M_1$ и $x, y \in M_2$. При этом имеем $x+y \in M_1, x+y \in M_2, tx \in M_1, tx \in M_2 \Rightarrow x+y \in M_1, x+y \in M_2, tx \in M_1 \cap M_2$, то есть $M_1 \cap M_2$ --- подпространство $V$. чтд браток.
    \end{hproof}

    \begin{htheorem}
        \textbf{Замечание о сумме подпространств}. Если $M_1$ и $M_2$ –-- подпространства пространства $V$ , то $M_1 + M_2$ –--
        наименьшее подпространство в $V$, содержащее $M_1$ и $M_2$.
    \end{htheorem}


    \begin{hproof}
        \textbf{Доказательство}. Если $x \in M_1$, то $x \in M_1 + M_2$, поскольку $x = x + 0, 0 \in M_2$. Следовательно, $M_1 \subseteq M_1 + M_2$. Аналогично, $M_2 \subseteq M_1 + M_2$. Тогда $x = x_1 + x_2$ для некоторых $x_1 \in M_1$ и $x_2 \in M_2$. Следовательно, $x_1, x_2 \in M$, откуда $x=x_1+x_2 \in M$. Итак $M_1 + M_2 \subseteq M$. чтд браток.
    \end{hproof}


    \newpage \begin{center}
                 \begin{Large}
                     \fbox{БИЛЕТ 29}
                 \end{Large}
    \end{center}
    
    \begin{htheorem}
        \textbf{Теорема о размерности суммы и пересечения подпространств}. Пусть $V$ --– векторное пространство, а $M_1$ и $M_2$ – его подпространства.
        Тогда размерность суммы подпространств $M_1$ и $M_2$ равна сумме размерностей этих подпространств минус размерность их пересечения.
    \end{htheorem}

    \begin{hproof}
        \textbf{Доказательство.} Из предложения о размерности подпространства $dim(M_1 \cap M_2) \leq dim M_1$ и $dim(M_1 \cap M_2) \leq dim M_2$.

        Положим $dim(M_1 \cap M_2) = k, dim M_1 = k + l, dim M_2 = k + m$. Если $M_1 = \{ 0 \}$, то очевидно $dim(M_1 \cap M_2) = \{ 0 \}, M_1 + M_2  = M_2$ и потому
        \begin{equation}
            dim(M_1 + M_2) = dim M_2 = dim M_1 + dim M_2 - dim(M_1 \cap M_2)
        \end{equation}
        Аналогично разбирается случай $M_2 = \{ 0 \} $.

        Далее можно считать, что $M_1$ и $M_2$ ненулевые и $M_1 \cap M_2 \neq \{ 0 \}$. Пусть $a_1, ..., a_k$ --- базис $dim(M_1 \cap M_2)$.

        По теореме о продолжении $a_1, ..., a_k$ можно дополнить как до базиса $M_1$, так и до $M_2$. Пусть $a_1, ..., a_k, b_1, ..., b_l$ --- базис $M_1$, а $a_1, ..., a_k,c_1, c_2, ..., c_m$ --- базис $M_2$.

        Докажем, что базис $a_1, ..., a_k, b_1, ..., b_l, c_1, ..., c_m$ является базисом пространства $M_1 + M_2$. Этого достаточно для доказательства теоремы, так как число векторов в этом наборе равно \begin{equation}
                                                                                                                                                                                                              k + l + m = (k+l) + (k+m) - k = dim M_1 + dim M_2 -dim (M_1 \cap M_2)
        \end{equation}

        Пусть $x \in M_1 + M_2$. Тогда $x = x_1 + x_2$. $x_1 $--- линейная комбинация векторов $a_1, ..., a_k, b_1, .., b_l$, а $x_2$ --- линейная комбинация векторов $a_1, ..., a_k,c_1, c_2, ..., c_m$.

        Отсюда $x$ --- линейная комбинация $a_1, ..., a_k, b_1, ..., b_l, c_1, ..., c_m$.

        Таким образом, $a_1, ..., a_k, b_1, ..., b_l, c_1, ..., c_m$ --- система образующих пространства $M_1 + M_2$. Осталось доказать, что эта система линейно независима.

        Предположим, что
        \begin{equation}
            t_1a_1 + t_2a_2 + ... +t_ka_k + s_1b_1 + ... + s_lb_l + ... + r_1c_1 + r_2c_2 + ... + r_mc_m = 0
        \end{equation}
        Нужно доказать, что все эти скаляры равны нулю.

        Положим $y = s_1b_2 +s_2b_2 + ... + s_lb_l$. Очев брат $y \in M_1$. С другой стороны из (28) вытекает, что \begin{equation}
                                                                                                                  y = -t_1a_1 - t_2a_2 -... -t_ka_k - r_1c_1 -r_2c_2 -.. -r_mc_m \in M_2
        \end{equation}

        Следовательно $y \in M_1 \cap M_2$. Тогда $y$ --- это линейная комбинация $a_1, ..., a_k$. То есть существуют такие скаляры $q_1, q_2, ..., q_k$, что
        \begin{equation}
            y = s_1b_1+s_2b_2+...+s_lb_l = q_1a_1 + q_2a_2 + ... + q_ka_k
        \end{equation}
        Следовательно,
        \begin{equation}
            q_1a_1 + q_2a_2 + ... + q_ka_k - s_1b_1 - s_2b_2-... - s_lb_l = 0
        \end{equation}

        Поскольку $a_1, a_2, ..., a_k, b_1, ..., b_l$ образуют базис пространства $M_1$, то они линейно независимы. Поэтому линейная комбинация (31) тривиальна. Следовательно, равенство (28) принимает вид $t_1a_1 +t_2a_2 + ... + t_ka_k +r_1c_1 + r_2c_2 + ... + r_mc_m = 0$.

        Учитывая, что вектора $a_1, .., a_k, c_1, ..., c_m$ образуют базис пространства $M_2$, получаем, что $t_1 = t_2 = ... = t_k = r_1 = ... = r_m = 0$. чтд браток.
    \end{hproof}

    \newpage \begin{center}
                 \begin{Large}
                     \fbox{БИЛЕТ 30}
                 \end{Large}
    \end{center}

    \textbf{Опредление}. Пусть $V$ – векторное пространство, а $M_1$ и $M_2$ – его подпространства.
    Говорят, что сумма подпространств $M_1$ и $M_2$ является их \textbf{прямой суммой},
    если $M_1 \cap M_2 =\{ 0 \}$. Прямая сумма подпространств $M_1$ и $M_2$
    обозначается через $M_1 \oplus M_2$.

    \begin{htheorem}
        \textbf{Замечание о базисе прямой суммы подпространств}. Если $V = M_1 \oplus M_2, b_1, b_2, ..., b_l - $базис $M_1$, а $c_1, c_2, ..., c_m$ - базис $M_2$, то $b_1, ..., b_l, c_1, ..., c_m$ --- базис пространства $V$.
    \end{htheorem}

    \begin{htheorem}
        \textbf{Теорема о прямой сумме подпространств}. Пусть $V$ – векторное пространство, а $M_1$ и $M_2$ – его подпространства. Следующие условия эквивалентны \begin{enumerate}
                                                                                                                                                                    \item $M_1 + M_2$ является прямой суммой подпространств $M_1$ и $M_2$.
                                                                                                                                                                    \item $dim(M_1+M_2) = dim M_1 + dim M_2$
                                                                                                                                                                    \item любой вектор из $M_1+M_2$ единственным образом представим в виде суммы вектора из $M_1$ и вектора из $M_2$.
                                                                                                                                                                    \item нулевой вектор пространства V единственным образом представим в виде суммы вектора из $M_1$ и вектора из $M_2$.
        \end{enumerate}
    \end{htheorem}

    \begin{hproof}
        \textbf{Доказательство.} Эквивалентность условий $1)$ и $2)$ непосредственно
        вытекает из теоремы о размерности суммы и пересечения и того факта,
        что размерность нулевого пространства равна $0$. Импликация $3) \Rightarrow 4)$
        очевидна. Поэтому достаточно доказать импликации $1) \Rightarrow 3)$ и $4) \Rightarrow 1)$.

        \begin{itemize}
            \item $1) \Rightarrow 3)$. Пусть $x \in M_1 + M_2$. По определению суммы подпространств $x = x_1 + x_2, x_1 \in M_1, x_2 \in M_2$. Остаётся доказать, то такое представление вектора x единственно. Предположим, что $x = y_1 + y_2, y_1 \in M_1, y_2 \in M_2$. Тогда мы имеем $x_1-y_1 = y_2 - x_2$. Ясно, что $x_1 - y_1 \in M_1, y_2-x_2 \in M_2$. Следовательно $x_1 - y_1 = y_2 - x_2 \in M_1 \cap M_2$.
            Но $M_1 \cap M_2 = \{ 0 \}$. Поэтому $x_1 - y_1 = y_2 - x_2 = 0$, откуда $x_1 = y_1, x_2 = y_2$. чтд браток.

            \item $4) \Rightarrow 1)$. Предположим, что $M_1 \cap M_2 \neq \{ 0 \}$, то есть существует ненулевой вектор $x \in M_1 \cap M_2$. Тогда вектор 0 может быть двумя различными способами представлен в виде суммы вектора из $M_1$ и вектора из $M_2$: $0 = x+(-x)$ и $0 = 0+0$. Мы получили противоречие с условием 4). чтд браток.
        \end{itemize}
    \end{hproof}

    \begin{htheorem}
        \textbf{Замечание о прямой сумме подпространств} \begin{equation}
                                                             V = M_1 \oplus M_2 \Leftrightarrow dim(M_1 + M_2) = dim M_1 + dim M_2 = dim V
        \end{equation}
        Необходимость сразу следует из теоремы о прямой сумме подпространств.
        Достаточность следует из теоремы о размерности сумм и пересечения. чтд браток.
    \end{htheorem}

    \textbf{Определение.}  Пусть $V = M_1 \oplus M_2$, $x \in V$. В силу теоремы о прямой сумме подпространств существуют однозначно определенные векторы $x_1 \in M_1$ и
    $x_2 \in M_2$ такие, что $x = x_1 + x_2$. Вектор $x_1$ называется проекцией x на $M_1$
    параллельно $M_2$, а вектор $x_2$ – проекцией x на $M_2$ параллельно $M_1$.



    \textbf{Алгоритм нахождения проекции вектора на подпространство.} Пусть $V = M_1 \oplus M_2$, $x \in V$. Предположим, что нам известен базис $a_1, ..., a_k$ подпространства $M_1$ и базис $b_1, ..., b_l$ подпространства $M_2$. В силу замечания о базисе прямой суммы подпространств $a_1, a_2, ..., a_k, b_1, b_2, ... ,b_l$ --- базис пространства $V$. Найдем координаты вектора x в этом базисе. Пусть они имеют вид $(t_1, ..., t_k, s_1, ..., s_l)$. Тогда $t_1a_1 + ... + t_ka_k + s_1b_1 + ... + s_lb_l$ --- проекция x на $M_1$ параллельно $M_2$, а $s_1b_1 + ... + s_lb_l$ --- проекция x на $M_2$ параллельно $M_1$.


    \textbf{Определение.} Пусть $V$ – векторное пространство, $x_0 \in V$ , а M – подпространство в $V$.
    Множество всех векторов вида $x_0 + y$, где $y \in M$, называется \textbf{линейным
    многообразием} в $V$ и обозначается через $x_0 + M$. Вектор $x_0$ называется
    \textbf{начальным вектором }многообразия $x_0 + M$, а подпространство $M$ –--
    \textbf{направляющим подпространством} этого многообразия. Размерность
    подпространства $M$ называется размерностью многообразия $x_0 + M$.

    Примеры
    \begin{itemize}
        \item Если $x_0 = 0$, то $x_0 + M = M$. Таким образом, всякое
        подпространство пространства V является линейным многообразием в V
        \item Если $M = \{ 0 \}$, то $x_0 + M = \{ x_0 \}$. Таким образом, всякий
        вектор из $V$ является линейным многообразием в $V$ (размерности $0$).
        \item Обычные прямые и плоскости трехмерного пространства –
        линейные многообразия.
    \end{itemize}

    \newpage \begin{center}
                 \begin{Large}
                     \fbox{БИЛЕТ 31}
                 \end{Large}
    \end{center}
    \subsection*{Понятие линейного отображения}

    Пусть $V$ и $W$ – векторные пространства над одним и тем же полем $F$.
    Отображение $A: V \rightarrow W$ называется \textbf{линейным оператором}, если
    для любых векторов $x1, x2 \in V$ и любого скаляра $t \in F$ выполняются
    равенства $A(x_1+x_2) = A(x_1) + A(x_2), A(tx_1) = tA(x_1)$

    Относительно первого равенства говорят, что $A$ сохраняет сумму векторов,
    относительно второго – что $A$ сохраняет произведение вектора на скаляр.
    Линейные операторы иначе называют линейными отображениями.

    Важный специальный случай возникает, когда пространства $V$ и $W$
    совпадают, т.е. $W = V$ . Тогда говорят, что A – линейный оператор
    на пространстве $V$ или что $A$ – линейный оператор пространства $V$.
    Линейные операторы на $V$ иначе называют линейными преобразованиями.

    \begin{htheorem}
        \textbf{Свойства линейного оператора}
        Пусть $V$ и $W$ – векторные пространства над полем $F$, а $A: V \rightarrow W$ –
        линейный оператор. Тогда:
        \begin{enumerate}
            \item $A(0) = 0$
            \item $A(\lambda_1v_1 + \lambda_2v_2 + ... + \lambda_mv_m) = \lambda_1A(v_1)+...+\lambda_m A(v_m)$ для любых векторов $v_1,v_2, ..., v_m \in V$ и любых скаляров $\lambda_1, ..., \lambda_m \in \mathbb{F}$.
        \end{enumerate}
    \end{htheorem}

    \begin{hproof}
        \textbf{Доказательство.} Первое свойство вытекает из того, что $A(0) = A(0 \cdot 0) = 0 \cdot A(0) = 0$.

        Второе свойство выводится из определения линейного оператора очевидной индукцией по $m$. чтд браток доказательство огонь.
    \end{hproof}


    \begin{htheorem}
        \textbf{Теорема существования и единственности линейного оператора}

        Пусть $V$ и $W$ – векторные пространства над полем $F$, причем $dim V = n > 0$. Пусть $P = \{ p_1, ..., p_n \}$ --- базис пространства $V$, а $w_1, ..., w_n$ --- произвольные вектора из $W$. Тогда существует
        единственный линейный оператор $A: V \rightarrow W$ такой, что $A(p_i) = w_i$ для всех $i = 1, 2, ..., n$.
    \end{htheorem}


    \begin{hproof}
        \textbf{Доказательство. Существование.} Пусть $x \in V$, а $(x_1, ..., x_n) --- $ координаты вектора $x$ в базисе $P$. Определим оператор $A: V \rightarrow W$ правилом $A(x) := x_1w_1+x_2w_2+...x_nw_n$. В силу единственности координат вектора в базисе это определение корректно (т. е. образ вектора x под действием $A$ определен однозначно). Из свойств координат суммы векторов и произведения вектора на скаляр вытекает, что этот оператор
        линеен. Осталось заметить, что для всякого $i = 1,...n$ вектор $p_i$ имеет в базисе $P$ координаты $(0,....,0,1,0,...,0)$, где единичка стоит на $i-$ом месте, и потому $A(p_i) = w_i$.

        \textbf{Единственность.} Пусть $B: V \rightarrow W$ --- линейный оператор такой, что $B(p_i) = w_i$ для всех $i = 1,...,n$. Пусть $x \in V$, а $(x_1, ..., x_n) --- $ координаты вектора $x$ в базисе $P$. Тогда $x = x_1p_1 + ... + x_np_n$. В силу замечания о свойствах линейного оператора имеем
        \begin{equation}
            B(x) = B(x_1p_1 + ... + x_np_n) = x_1B(p_1) + ... + x_nB(p_n) = x_1w_1 + ... + x_nw_n = A(x)
        \end{equation}

        Следовательно $B=A$. чтд детка.
    \end{hproof}



    \subsection*{Произведение отображений}
    \textbf{Определение} Пусть $V$ и $W$ – векторные пространства над полем $F$, $A: V \rightarrow W$ –
    линейный оператор, а $t \in \mathbb{F}$. Произведением оператора $A$ на скаляр $t$
    называется оператор $B: V \rightarrow W$, задаваемый правилом $B(x) := tA(x)$ для
    всех $x \in V$ . Произведение оператора A на скаляр t обозначается через $tA$.


    \subsection*{Линейное пространство линейных отображений}
    \textbf{Определение.} Пусть $V$ и $W$ – векторные пространства над полем $F$, а $A$ и $B$ – линейные операторы из V в W. Суммой операторов $A$ и $B$ называется оператор $S : V \rightarrow W$, задаваемый правилом $S(x) := A(x) + B(x)$ для всех $x \in V$ .
    Сумма операторов $A$ и $B$ обозначается через $A + B$.

    Множество всех линейных операторов из $V$ в $W$ обозначается $Hom(V, W)$.

    \begin{htheorem}
        \textbf{Предложение о пространстве линейных операторов}.

        Произведение линейного оператора на скаляр является линейным
        оператором. Множество $Hom(V, W)$ с операциями сложения операторов
        и умножения оператора на скаляр является векторным пространством.
    \end{htheorem}


    \begin{hproof}
        \textbf{Доказательство}. Пусть $A,B \in Hom(V), x,y \in V, t, s \in \mathbb{F}$. Тогда
        \begin{equation}
        (tA)(x+y)
            = t(A(x+y)) = t(A(x) + A(y)) = tA(x) + tA(y) = (tA)(x) + (tA)(y)
        \end{equation}

        \begin{equation}
        (tA)(sx)
            = t(A(sx)) = t(sA(x)) = (ts)(A(x)) = s(tA(x)) = s((tA)(x))
        \end{equation}

        Следовательно, $tA$ --- линейный оператор.

        Похожим образом получаем, что $1 \cdot A = A$. С учетом свойств суммы операторов, мы получаем, что в $Hom(V, W)$ выполнены все аксиомы векторного пространства. Уничтожено!
    \end{hproof}

    \begin{htheorem}
        \textbf{Теорема о пространствах линейных операторов и матриц}.

        Если $V$ и $W$ – векторные пространства над полем $F$, $dim V = n$ и
        $dim W = k$, то векторные пространства $Hom(V,W)$ и $F^{k\times n}$ изоморфны.
    \end{htheorem}


    \begin{hproof}
        \textbf{Доказательство}. Зафиксируем в V базис $P = \{ p_1, p_2, . . . , p_n \}$, а в W –
        базис $Q = \{ q_1, q_2, . . . , q_k \}$. Определим отображение
        $\phi: Hom(V, W) \rightarrow F^{k \times n}$
        правилом: если $A: V \rightarrow W$ – линейный оператор,
        то $\phi(A)$ – матрица оператора A в базисах P и Q. Пусть $A, B \in Hom(V)$ и
        $t \in F$. Надо проверить, что отображение $\phi$ биективно и выполнены равенства \begin{equation}
                                                                                              \phi (A+B) = \phi(A) + \phi(B)
        \end{equation}
        \begin{equation}
            \phi(tA) = t \phi(A)
        \end{equation}

        В матрице $\phi(A+B)$ по столбцам записаны координаты векторов
        $(A + B)(p_i)$ в базисе Q, а в матрицах $\phi(A)$ и $\phi(B)$ – координаты векторов
        $A(p_i)$ и $B(p_i)$ соответственно в том же базисе. Поскольку
        $(A + B)(p_i) = A(p_i) + B(p_i)$, координаты вектора $(A + B)(p_i)$ равны
        сумме координат векторов $A(p_i)$ и $B(p_i)$. Первое из равенств \begin{equation}
                                                                             \phi (A+B) = \phi(A) + \phi(B)
        \end{equation}
        \begin{equation}
            \phi(tA) = t \phi(A)
        \end{equation}
        Доказано. Второе из них проверяется вполне аналогично.

        Проверим, что отображение $\phi$ биективно. Если $A, B \in Hom(V, W)$ и
        $\phi(A) = \phi(B)$, то из определения матрицы линейного оператора вытекает,
        что операторы $A$ и $B$ одинаково действуют на базисных векторах
        пространства $V$. Но тогда $A = B$, так как линейный оператор однозначно
        определяется своим действием на базисных векторах. Следовательно,
        отображение $\phi$ инъективно.

        Осталось доказать, что $\phi$ сюръективно. Пусть $A = (a_{ij}$) – произвольная
        матрица размера $k \times n$. Для всякого $j = 1, 2, . . . , n$ положим
        $w_j = a_{1j}q_1 + a_{2j}q_2 + · · · + a_{kj}q_k$. В силу теоремы существования и
        единственности линейного оператора существует линейный оператор $A$
        такой, что $A(p_i) = w_i$ для всех $i = 1, 2, . . . , n$. Из определения матрицы
        оператора вытекает, что $A_{P,Q} = A$, т.е. $\phi(A) = A$. Следовательно, отображение $\phi$ сюръективно. Убито фух.
    \end{hproof}

    \begin{htheorem}
        \textbf{Следствие о размерности пространства линейных операторов}

        Если $V$ и $W$ – векторные пространства над полем $F$, $dim V = n$ и
        $dim W = k$, то $dim Hom(V, W) = kn$.
    \end{htheorem}

    Если $W = V, Q = P$, то говорят о \textbf{матрице оператора в базисе} $P$.

    \begin{htheorem}
        \textbf{Предложение о свойствах суммы операторов}.


        Сумма линейных операторов является линейным оператором. Множество
        $Hom(V, W)$ с операцией сложения операторов является абелевой группой.
    \end{htheorem}

    \begin{hproof}
        \textbf{Доказательство.} Пусть $A, B \in Hom(V), S = A+B$. $\forall x, y \in V, t \in \mathbb{F}$ имеем  \begin{equation}
                                                                                                                     S(x+y) = A(x + y) + B(x + y) = A(x) + A(y) + B(x) + B(y) = (A(x) + B(x)) +(A(y) + B(y))
                                                                                                                     = S(x) + S(y)
        \end{equation}

        \begin{equation}
            S(tx) = A(tx) + B(tx) = tA(x) + tB(x) = t(A(x) + B(x))= tS(x)
        \end{equation}

        Следовательно, оператор $S$ линеен.

        Далее, если $A, B, C \in Hom(V,W)$, то \begin{equation}
        (A+B)(x)
                                                   = A(x) + B(x) = B(x) + A(x) = (B+A)(x)
        \end{equation}
        \begin{equation}
            \begin{matrix}
                ((A+B)+C)(x) = (A+B)(x) + C(x) = (A(x) + B(x)) + C(x)= \\
                = A(x) + (B(x) + C(x)) = A(x) + (B+C)(x) = (A+(B+C))(x)
            \end{matrix}
        \end{equation}

        Получается $A+B = B+A, (A+B)+C = A+(B+C)$. Нейтральным элементом по сложению является нулевой оператор $O$, поскольку $(A+O)(x) = A(x) + O(x) = A(x) + 0 = A(x)$

        Обратным по сложению является оператор $-A$.

        Предложение доказано. туда его.
    \end{hproof}


    \subsection*{Свойства произведения}

    \newpage \begin{center}
                 \begin{Large}
                     \fbox{БИЛЕТ 32}
                 \end{Large}
    \end{center}
    \subsection*{Матрица линейного отображения}

    \textbf{Определение}. Пусть $V$ и $W$ --- векторные пространства над полем $F$,\newline причем $dim V = n>0, dim W = k > 0$. Пусть $P = \{ p_1, ..., p_n \}$ --- базис пространства $V$, а $Q = \{ q_1, ..., q_k \}$ - базис пространства $W$. \textbf{Матрицей линейного оператора } $A: V \rightarrow W$ \textbf{в базисах} $P$ и $Q$ называется $k \times n$ --- матрица, итый столбец которой состоит из координат вектора $A(p_i)$ в базисе $Q, i = 1, 2, ..., n$. Эта матрица обозначается $A_{P,Q}$ или просто $A$, если базисы зафиксированы.

    Итак если \begin{equation}
                  \begin{matrix}
                      A(p_1) = a_{11}q_1 + a_{21}q_2 + ... + a_{k1}q_k, \\
                      A(p_2) = a_{12}q_1 + a_{22}q_2 + ... + a_{k2}q_k, \\
                      ...                                               \\
                      A(p_n) = a_{1n}q_1 + a_{2n}q_2 + ... + a_{kn}q_k,
                  \end{matrix}
    \end{equation}
    то $\displaystyle A_{P,Q} = \begin{pmatrix}
                                    a_{11} & a_{12} & ... & a_{1n} \\
                                    a_{21} & a_{22} & ... & a_{2n} \\
                                    ...    & ...    & ... & ...    \\
                                    a_{k1} & a_{k2} & ... & a_{kn} \\
    \end{pmatrix}$

    $y = A(x) = A(x_1p_1 + ... + x_np_n) = x_1(a_{11}q_1 + ... + a_{m1}q_m) + ... + x_n(a_{1n}q_1 + ... + a_{mn}q_m) = (x_1a_{11} + ... + x_na_{1n})q_1 + ... + (x_1a_{21} + ... + x_na_{2n}) + ...$

    $\displaystyle \begin{pmatrix}
                       y_1 = x_{11} + x_2a_{12} + ... + x_na_{1n} \\
                       ...                                        \\
                       y_m = x_{m1} + x_2a_{m2} + ... + x_na_{mn} \\
    \end{pmatrix}$

    \begin{equation}
        \displaystyle \begin{pmatrix}
                          y_1 \\
                          ... \\
                          y_n \\
        \end{pmatrix} = \begin{pmatrix}
                            a_{11} & ... & a_{1n} \\
                            ...    & ... & ...    \\
                            a_{m1} & ... & a_{mn} \\
        \end{pmatrix} \begin{pmatrix}
                          x_1 \\
                          ... \\
                          x_n \\
        \end{pmatrix}
    \end{equation}

    $\left[y \right]_Q = A_{P,Q} \left[ x \right]$

    \begin{htheorem}
        \textbf{Теорема}. Композиция линейных отображений --- линейное отображение.
    \end{htheorem}

    \begin{hproof}
        \textbf{Доказательство.}

        $(f \circ g)(x+y) = g(f(x+y)) = g(f(x) + f(y) = g(f(x)) + g(f(y)) = (f \circ g)(x) + (f \circ g)(y)$

        $(f \circ g)(\alpha x) = g(f(\alpha x)) = g(\alpha f(x)) = \alpha g(f(x)) = \alpha (f \circ g(x))$
    \end{hproof}

    Композиция однозначна и всюду определена, так как $f$ и $g$ однозначны и всюду определены.

    Так как линейные отображения образуют абелеву группу по сложению и дистрибутивны отнасительно сложения, то линейные отображения образуют кольцо.

    Линейные преобразования линейного пространства образуют ассоциативное кольцо с единицей.

    \begin{htheorem}
        \textbf{Теорема.} $[f \circ g] = [g][f]$ (f и g --- линейные операторы)
    \end{htheorem}

    \begin{hproof}
        $a_1 = (1,0,...,0)^T$

        $f(a_1) = \alpha_{11}b_1 + ... + \alpha_{n1}b_n$

        $\displaystyle (f \circ g)(a_1) = g(f(a_1)) = g(\alpha_{11}b_1 + ... + \alpha_{n1}b_n) = \alpha_{11}g(b_1) + ... + \alpha_{n1}g(b_n) =
        \alpha_{11} \begin{pmatrix}
                      \beta_{11} \\
                      ...        \\
                      \beta_{q1} \\
        \end{pmatrix} + ... + \alpha_{n1} \begin{pmatrix}
                                              \beta_{1n} \\
                                              ...        \\
                                              \beta_{qn} \\
        \end{pmatrix} = [g] \begin{pmatrix}
                                \alpha_{11} \\
                                ...         \\
                                \alpha_{n1} \\
        \end{pmatrix} = [g(f(a))] = [g][f](a)$


    \end{hproof}

    \begin{htheorem}
        Умножение матриц дистрибутивно относительно сложения.
    \end{htheorem}

    \begin{hproof}
        $B(A+C) = [g]([f]+[h]) = [g]([f+h]) = [(f+h) \circ g] = [f \circ g + h \circ g] = [g][f] + [g][h] = BA+BC$
    \end{hproof}


    \begin{htheorem}
        Умножение матриц ассоциативно.
    \end{htheorem}

    \begin{hproof}
        $A(CB) = A([h \circ g]) = [(h \circ g) \circ f] = [h \circ (g \circ f)] = [g \circ f][h] = ([f][g])[f]$
    \end{hproof}

    \newpage \begin{center}
                 \begin{Large}
                     \fbox{БИЛЕТ 33}
                 \end{Large}
    \end{center}

    \textbf{Определение. Образ} $A$ --- множество \textit{ImA} всех $y \in W$ таких, что $A(x) = y \forall x \in V$.

    \textbf{Ядро А} --- множество \textit{kerA} всех $x \in V$ таких, что $A(x) = 0$.

    \begin{htheorem}
        \textbf{Замечание.} $A: V \rightarrow W$. $ImA$ --- подпространство в $W$, $kerA$ --- подпространство в $W$.
    \end{htheorem}

    \begin{hproof}
        \textbf{Доказательство.} Пусть $y_1, y_2 \in ImA, t \in F$

        $\exists x_1, x_2 \in V: A(x_1) = y_1$ и $A(x_2) = y_2 \Rightarrow y_1+y_2 = A(x_1)+A(x_2) = A(x_1+x_2)$ и

        $ty_1 = tA(x_1) = A(tx_1) \Rightarrow y_1+y_2, ty_1 \in ImA \Rightarrow ImA$ --- подпространство.

        Пусть $x_1, x_2 \in kerA, t \in F \Rightarrow A(x_1 + x_2) = A(x_1) + A(x_2) = 0+0 = 0$

        $A(tx_1) = tA(x_1) = t \cdot 0 = 0 \Rightarrow x_1 + x_2, tx_1 \in kerA \Rightarrow kerA$ --- подпространство.
    \end{hproof}

    \begin{htheorem}
        \textbf{Замечание.} $A: V \rightarrow W$. $V$ --- конечномерное $\Rightarrow ImA$ конечномерное.
    \end{htheorem}

    \begin{hproof}
        \textbf{Доказательство.} Если $dimV = 0$, доказывать нечего.
        Пусть тогда $dimV = n>0$ и $P = \{p_1, ..., p_n\}$ --- какой-то базис V. Покажем, что подпространство $ImA$ порождается векторами $A(p_1), A(p_2), ..., A(p_n)$

        Возьмём любой $y \in ImA$. $\exists x \in V: A(x) = y$. Если $(x_1, ..., x_n)$ --- координаты вектора $х$ в базисе $Р$, то $x = x_1p_1 + ... + x_np_n \Rightarrow y = A(x_1p_1 + ... + x_np_n) = x_1A(p_1) + ... + x_nA(p_n)$

        $ImA$ имеет конечную систему образующих $\Rightarrow$ оно конечномерно.
    \end{hproof}

    \textbf{Определение. Ранг} $А$ --- размерность образа линейного оператора $А$ (обозначается как $r(A)$.

    \textbf{Дефект} $А$ --- размерность ядра оператора А ($d(A)$).

    \begin{htheorem}
        \textbf{Теорема о сумме ранга и дефекта.} $r(A) + d(A) = dimV$
    \end{htheorem}

    \begin{hproof}
        \textbf{Доказательство.} Пусть $r(A) = r, d(A) = d$. Выберем базис ядра $a_1, ..., a_d$ и базис $c_1, ..., c_r$.

        $\forall c_i \exists b_i: A(b_i) = c_i$. Докажем, что $a_1, ..., a_d, b_1, ..., b_r$ --- базис $V$.

        $A(x) = t_1c_1 + ... + t_rc_r, x' = t_1b_1 + ... + t_rb_r\\ \Rightarrow A(x) = A(x') \Rightarrow x-x' \in kerA, x-x' = s_1a_1 + ... + s_da_d$.

        Поэтому $x=(x-x')+x' = s_1a_1 + ... s_da_d + t_1b_1 + ... + t_rb_r$

        Покажем линейную независимость $a_1, ..., a_d, b_1, ..., b_r$:

        Пусть $s_1a_1 + ... + s_da_d + t_1b_1 + ... +t_rb_r = 0$

        Применим оператор к обеим частям равенства: $A(s_1a_1 + ... + s_da_d + t_1b_1 + ... +t_rb_r) = tc_1 + t_2 + ... + t_rc_r = 0$

        $c_1, ..., c_r$ линейно независимы $\Rightarrow \forall i \quad  t_i = 0$

        $s_1a_1 + ... + s_da_d = 0$, но $a_1, ..., a_d$ тоже линейно независимы $\Rightarrow \forall i \quad  a_i = 0 \Rightarrow a_1, ..., a_d, b_1, ..., b_r ---$ базис V $\Rightarrow dimV = r+d$.
    \end{hproof}


    \newpage \begin{center}
                 \begin{Large}
                     \fbox{БИЛЕТ 34}
                 \end{Large}
    \end{center}

    \textbf{Определение.} Отображение $V \times V \rightarrow F$ для $x, y \in V$, обозначаемый $xy$ или $(x,y)$ или $<x,y>$ называется \textbf{скалярным произведением}, если
    \begin{enumerate}
        \item $\forall x, y \in V xy =  \overline{yx}$
        \item $\forall x,y \in V \forall \alpha \in F (\alpha x) y = \alpha (xy)$
        \item $\forall x, y, z \in V (x+y)z = xz + yz$
        \item $\forall x \in V \quad xx \geq 0$
    \end{enumerate}
    \textbf{Примеры}

    \begin{enumerate}
        \item Трёхмерное пространство аналитической геометрии.
        \item Пространство многочленов над $\mathbb{R}$
    \end{enumerate}

    \begin{htheorem}
        \textbf{Ослабленный закон сокращения.} Если $V$ --- пространство со скалярным произведением, а $a,b \in V$ таковы, что $\forall x \in V: ax = bx$, то $a=b$
    \end{htheorem}

    \begin{hproof}
        \textbf{Доказательство.} $(a-b)x = 0$, а значит и $(a-b)(a-b) = 0, a-b = 0, a=b$
    \end{hproof}

    \textbf{Определение.} \textbf{Длина} вектора x --- $|x| = \sqrt{xx}$

    \textbf{Определение.} $\displaystyle \frac{x}{|x|}$ --- \textbf{орт} вектора х.
    \begin{htheorem}
        \textbf{Замечание.} Если $x \neq 0$, то $\displaystyle \left| \frac{x}{|x|} \right| = 1$
    \end{htheorem}

    \begin{hproof}
        \textbf{Доказательство.}

        $|\alpha x| = |\alpha||x|$

        $\displaystyle  \left| \frac{x}{|x|} \right| =  \left| \frac{1}{|x|} \cdot x \right| =  \left| \frac{1}{|x|} \right| |x| =   \frac{1}{|x|} |x| = 1$
    \end{hproof}

    \begin{htheorem}
        \textbf{Теорема (неравенство Коши-Буняковского).} Пусть $V$ --- пространство со скалярным произведением и $x,y \in V$. Тогда $|xy| \leq |x||y|$
    \end{htheorem}

    \begin{hproof}
        \textbf{Доказательство.} При $y=0$ очевидно.

        Рассмотрим случай $y \neq 0$. Рассмотрим вектор $x -\alpha y$

        $(x-\alpha y)(x-\alpha y) \geq 0$

        $xx - \alpha y x - \overline{\alpha} x y + \alpha \overline{\alpha} y y \geq 0$

        $\displaystyle \alpha = \frac{xy}{yy}$

        $\displaystyle 0 \leq xx - \frac{xy}{yy} yx - \frac{\overline{xy}}{yy} xy +  \frac{xy}{yy} \frac{\overline{xy}}{yy} yy = xx - \frac{|xy|^2}{yy}$ $\Rightarrow |xy|^2 \leq xx \cdot yy$

        $xx = |x|^2, yy = |y|^2 \Rightarrow |xy| \leq |x||y|$.

        Если $x$ и $y$ линейно независимы, то $x- \alpha y \neq 0 \forall \alpha$ и $(a-\alpha y)(x-\alpha y) > 0$

        Тогда $|xy| < |x||y| \Rightarrow |xy| \leq |x||y| \Leftrightarrow$ $x$ и $y$ линейно зависимы.

        Обратно. Пусть $x$ и $y$ линейно зависимы: $y \neq 0, x = \lambda y$.

        $|xy| = |(\lambda y) y| = |\lambda (yy)| = |\lambda| |yy| = |\lambda||y||y| = |\lambda y||y| = |x||y|$
    \end{hproof}

    \begin{htheorem}
        \textbf{Теорема.} $|x+y| \leq |x|+|y|$. сли $x$ и $y$ линейно независимы, то $|x+y| < |x|+ |y|$
    \end{htheorem}

    \begin{hproof}
        \textbf{Доказательство.}
        $|x+y|^2 = (x+y)(x+y) = |(x+y)(x+y)| = |xx+xy+yx+yy| \leq |xx| + |xy| + |yx| + |yy| = |x|^2 + 2|xy| = |y|^2 \leq |x|^2 + 2|x||y| + |y|^2 = (|x| + |y|)^2$

        $|x+y|^2 \leq (|x| + |y|)^2$

        $|x+y| \leq |x| + |y|$
    \end{hproof}

    \newpage \begin{center}
                 \begin{Large}
                     \fbox{БИЛЕТ 35}
                 \end{Large}
    \end{center}

    \textbf{Определение.} Два вектора \textbf{ортогональны} $\Rightarrow$ скалярное произведение равно 0.

    Базис называется \textbf{ортогональным}, если любые его векторы ортогональны друг другу.

    Базис называется \textbf{ортонормированным}, если он единичный и ортогональный.

    \begin{htheorem}
        \textbf{Утверждение.} Пусть $x_1, ..., x_n$ --- ортонормированная система и $\forall i \quad x_i \neq 0 \Rightarrow x_1, ..., x_n$ --- линейно независима.
    \end{htheorem}

    \begin{hproof}
        \textbf{Доказательство.} От противного. $\alpha_1 x_1 + ... \alpha_n x_n = 0$. Пусть б.о.о. $\alpha_1 \neq 0$

        $\alpha_1 x_1 x_1 + ... \alpha_n x_n x_1 = 0$

        $\alpha_1 x_1 x_1 = 0$. Но $\alpha_1 x_1 x_1 \neq 0$. Противоречие.
    \end{hproof}


    \begin{htheorem}
        \textbf{Свойство.} $a \perp b \Rightarrow |a+b|^2 = |a|^2 + |b|^2$
    \end{htheorem}

    \begin{htheorem}
        \textbf{Теорема.} Пусть $Р$ --- ортонормированный базис V со скалярным произведением.

        Тогда $\forall x, y \in V \quad xy = [x]_p^T \cdot \overline{[y]_P}$
    \end{htheorem}

    \begin{hproof}
        \textbf{Доказательство.} Обозначим координаты $x$ и $y$ как $(x_1, ..., x_n)$ и $(y_1, ..., y_n)$. Пусть базис $Р$ состоит из векторов $a_1, ..., a_n$.

        $y = x_1a_1+...+x_na_n$

        $y = y_1a_1 + ... + y_na_n$

        $\displaystyle xy = (x_1a_1+...+x_na_n)(y_1a_1 + ... + y_na_n) = \sum_{i,j=1}^n(x_ia_i)(y_ja_j) = \sum_{i=1}^n x_i \overline{y_i}a_i^2 = \sum_{i=1}^n x_i \overline{y_i} = [x]_P^T \cdot \overline{[y]_P}$
    \end{hproof}

    \begin{htheorem}
        \textbf{Теорема (процесс ортогонализации Грама-Шмидта).} Пусть $a_1, ..., a_n$ --- линейно независимая система векторов пространства со скалярным произведением $V$. Тогда в $V$ найдётся ортогональная система ненулевых векторов $b_1, ..., b_n$, линейная оболочка которых совпадает с линейной оболочкой системы $a_1, ..., a_n$.
    \end{htheorem}

    \begin{hproof}
        \textbf{Доказательство.}
        Индукция по $n$. Для $n=1$ положим $b_1=a_1$.

        Пусть $1 \leq i < n$ и уже найден ортогональный набор ненулевых $b_1, ..., b_n$, линейная оболочка которого совпадает с оболочкой $a_1, ..., a_n$.

        Ищем вектор $b_{i+1}$ в виде $b_{i+1} = \alpha_1 b_1 + ... + \alpha_i b_i + a_{i+1} (*)$.

        Чтобы найти $\alpha_1$, умножим скалярно обе части равенства на $b_1$ справа:

        $b_{i+1} b_1 = \alpha_1 b_1 b_1 + a_{i+1} b_1$

        $0 = \alpha_1 b_1 b_1 + a_{i+1} b_1$

        $\displaystyle \alpha_1 = \frac{a_{i+1}b_1}{b_1b_1}$.

        Аналогично умножая на $b_i$, можно найти $\alpha_i$.

        $b_i$ являются линейными комбинациями векторов $a_1, ..., a_n$. Поэтому (*) даёт $b_{i+1} = t_1a_1 + ... + t_na_n$. Поскольку $a_1, ..., a_n$ линейно независимы, то никакая их нетривиальная линейная комбинация не может быть нулевым вектором. Отсюда $b_{i+1} \neq 0$.
    \end{hproof}

    \begin{htheorem}
        \textbf{Следствие.} В любом конечномерном пространстве со скалярным произведением существует ортогональный базис.
    \end{htheorem}

    \begin{hproof}
        \textbf{Доказательство.}

        Пусть $dimV = n$. Возьмём произвольный базис в $V$ и применим к нему процесс Грама-Шмидта. Получим ортогональную систему из $n$ векторов, порождающую V, а следовательно, оротогональный базис в $V$. В силу замечния об орте вектора, разделив каждый вектор этого базиса на его длину, получим ортонормированный базис пространства $V$.
    \end{hproof}

    \begin{htheorem}
        \textbf{Следствие.} Любую ортогоналную систему ненулевых векторов конечномерного пространства со скалярным произведением можно дополнить до ортогонального базиса этого пространства.
    \end{htheorem}

    \begin{hproof}
        \textbf{Доказательство.} Пусть $dimV = n$, $a_1, ..., a_k$ --- ортогональный набор ненулевых векторов из $V$. Тогда $a_1, ..., a_k$ линейно независимы, и их можно дополнить какими-то векторами $a_{k+1}, ..., a_n$ до базиса $V$. Применив к базису $a_1, ..., a_k$ процесс Грама-Шмидта, получим ортогональный базис в $V$. На первых k шагах процесс будет возвращать именно вектора $a_1, ..., a_k$.
    \end{hproof}

    \textbf{Определение.} Пусть $S$ --- подпространство в $V$. Множество всех векторов, ортогональных к произвольному вектору из $S$, называется \textbf{ортогональным дополнением} подпространства $S$.

    \begin{htheorem}
        \textbf{Предложение.} Пусть $S$ --- подпространство пространства со скалярным роизведением V, а $S^{\bot}$ --- ортогональное дополнение S. Тогда
        \begin{enumerate}
            \item $S^{\bot}$ --- подпространство V.
            \item Если $a_1, ..., a_k$ --- базис $S$, то  $x \in S^{\bot} \Leftrightarrow xa_1 = ... = xa_k = 0$
        \end{enumerate}
    \end{htheorem}

    \begin{hproof}
        \textbf{Доказательство.}

        \begin{enumerate}
            \item Если $x, y \in S^{\bot}, a \in S, t \in F$, то $(x+y)a = xa+ya = 0+0=0$ и $t(xa) = t \cdot 0 = 0$

            \item Если $a_1, ..., a_k$ --- базис $S$, а $x \in S^{\bot}$, то x ортогонален $a_1, ..., a_k$. Предположим, что x ортогонален $a_1, ..., a_k$. Пусть $a \in S$. Тогда $a = t_1a_1 + ... +t_ka_k$.

            Тогда $ax = (t_1a_1 + ... + t_ka_k)x = t_1(a_1x) + ... + t_k(a_kx) = t_10 + ... + t_k0 = 0$ и поэтому $x \in S^{\bot}$.
        \end{enumerate}
    \end{hproof}


    \newpage \begin{center}
                 \begin{Large}
                     \fbox{БИЛЕТ 36}
                 \end{Large}
    \end{center}

    \begin{htheorem}
        \textbf{Теорема об ортогональном разложении.}

        Если $V$ --- пространство со скалярным произведением, а $S$ --- подпространство в $V$, то $V = S \oplus S^{\bot}$
    \end{htheorem}

    \begin{hproof}
        \textbf{Доказательство.} Если $x \in S \cap S_{\bot}$, то $xx=0 \Rightarrow x=0 \Rightarrow S \cap S_{\bot} = \{ 0 \}$, откуда следует, что сумма $S + S^{\bot}$ --- прямая.

        Проверим, что $S + S^{\bot} = V, dimV = n, dimS = k$.

        Возьмём ортонормированный базис $a_1, ..., a_k$ подпространства $S$ и дополним его до ортонормированного базиса $V$. Пусть $a_{k+1}, ..., a_n$ --- вектора, использованные для дополнения. Каждый из этих $n-k$ векторов ортогонален $a_1, ..., a_k \Rightarrow a_{k+1}, ..., a_n in S^{\bot}$.

        Итак,  $S + S^{\bot}$ содержит $a_1, ..., a_n$, составляющие базис V,  откдуа $S + S^{\bot} = V$
    \end{hproof}


    \begin{htheorem}
        \textbf{Свойства ортогонального дополнения.} Пусть $V$ --- пространство со скалярным произведением, а $S, S_1, S_2$ --- его подпространства. Тогда
        \begin{enumerate}
            \item $V^{\bot} = \{ 0 \}$, а $\{ 0 \} = V^{\bot}$
            \item $(S^{\bot})^{\bot} = S$
            \item $S_1 \subseteq S_2 \Rightarrow S_2^{\bot} \subseteq S_1^{\bot}$
            \item $(S_1+S_2)^{\bot} = S_1^{\bot} \cap S_2^{\bot}, (S_1 \cap S_2)^{\bot} = S_1^{\bot} + S_2^{\bot}$
            \item $V = S_1 \oplus S_2 \Rightarrow V = S_1^{\bot} \oplus S_2^{\bot}$
        \end{enumerate}
    \end{htheorem}

    \begin{hproof}
        \textbf{Доказательство.}
        \begin{enumerate}
            \item Если $x \in V^{\bot}$, то $xy = 0 \forall y \in V$. В частности$xx = 0\Rightarrow x = 0 \Rightarrow V^{\bot} = \{ 0 \}$.

            $\{ 0 \}^{\bot} = V$ вытекает из ортогональности нулевого вектора.

            \item Из определения ортогонального дополнения вытекает, что если $x \in S$, то $x$ ортогонален любому $y \in S^{\bot} \Rightarrow S \subseteq (S^{\bot})^{\bot}$

            Пусть $dimS = l, dimV = n$. В силу теоремы об ортогональном разложениии $dim (S^{\bot})^{\bot} = n - dim S^{\bot} = n-(n-k) = k = dimS \Rightarrow S$ - подпространство $(S^{\bot})^{\bot}$ и $dimS = dim (S^{\bot})^{\bot} \Rightarrow S = (S^{\bot})^{\bot}$

            \item Пусть $S_1 \subseteq S_2, x \in S_2^{\bot}, x \in S_2^{\bot}$. Тогда x ортогонален любому вектору из $S_2$, а значит и любому из $S_1 \Rightarrow x \in S_1^{\bot} \Rightarrow S_2^{\bot} \subseteq S_1^{\bot}$

            \item Пусть $x \in S_1^{\bot} \cap S_2^{\bot}, y \in S_1+S_2$. Тогда $y=y_1+y_2, y_1 \in S_1, y_2 \in S_2$. В силу выбора х имеем $xy_1 = xy_2 = 0 \Rightarrow xy = x(y_1+y_2) = xy_1 + xy_2 = 0+0=0 \Rightarrow x \in (S_1 + S_2)^{\bot} \Rightarrow S_1^{\bot} \cap S_2^{\bot} \subseteq (S_1+S_2)^{\bot}$



            Обратно. Пусть $x \in (S_1+S_2)^{\bot}$. Так как $S_1 \subseteq S_1+S_2, S_2 \subseteq S_1+S_2$, из 3) вытекает, что $x \in S_1^{\bot}, x\ \in S_2^{\bot} \Rightarrow x \in S_1^{\bot} \cap S_2^{\bot}$ и потому $(S_1 + S_2)^{\bot} \subseteq (S_1+S_2)^{\bot}$.

            Используя 2): $S_1^{\bot} + S_2^{\bot} = ((S_1^{\bot} + S_2^{\bot})^{\bot})^{\bot} = ((S_1^{\bot})^{\bot} \cap (S_2^{\bot})^{\bot})^{\bot} = (S_1 \cap S_2)^{\bot}$.

            \item $S_1 \cap S_2 = \{ 0 \}$. Из 1) и 4) вытекает, что $S_1^{\bot} + S_2^{\bot} = (S_1 \cap S_2)^{\bot} = \{ 0 \}^{\bot} = V$

            $S_1+S_2 = V \Rightarrow S+1^{\bot} \cap S_2^{\bot} = (S_1 + S_2)^{\bot} = V^{\bot} = \{ 0 \}$. Итак, $V = S_1^{\bot} \oplus S_2^{\bot}$.
        \end{enumerate}
    \end{hproof}
    

\newpage \begin{center}
                 \begin{Large}
                     \fbox{БИЛЕТ 37}
                 \end{Large}
         \end{center} 
         
         Отображение $f: M_1 \rightarrow M_2$ \textbf{обратимо} тогда и только тогда, когда $f$ - взаимно однозначное отображение.
         
         \begin{htheorem}
         	\textbf{Предложение.}
         	
         	Если $f: V_1 \rightarrow V_2$ - взаимно однозначное линейное отображение $V_1$ на $V_2$, то $f^{-1}: V_1 \rightarrow V_2$ также линейное.
         \end{htheorem}
         
         \begin{hproof}
         \textbf{Доказательство.} Рассмотрим произвольные $y_1, y_2 \in V_2$ и пусть $x_1 := f^{-1}(y_1), x_2 = f^{-1}(y_2)$. Тогда $f(x_1+x_2) = f(x_1)+f(x_2) = y_1+y_2$, откуда $f^{-1}(y_1+y_2) = x_1+x_2 = f^{-1}(y_1) + f^{-1}(y_2)$. Аналогично проверятся, что $f^{-1}(ty) = tf^{-1}(y) \quad \forall y \in V_2 \quad \forall t \in F$.
         \end{hproof}
                 
         У изоморфных пространств одинаковы размерности, поэтому матрица обратимого линейного отображения будет \textbf{квадратной}.
         
         
          Если $f: V_1 \rightarrow V_2$ - взаимно однозначное линейное отображение векторного пространства $V_1$ на векторное пространство $V_2$, то а $f^{-1}: V_2 \rightarrow V_1$ — обратное отображение, то композиция $f \circ f^{-1}$ —
тождественное отображение пространства $V_1$, а композиция $f^{-1} \circ f$ —
тождественное отображение пространства $V$.

\begin{htheorem}
\textbf{Теорема.} Матрица $n \times n$ имеет обратную $\Leftrightarrow$ её ранг равен $n$.
\end{htheorem}

\begin{hproof}
\textbf{Доказательство.}

Матрице $A$ составим линейный оператор, который каждый столбец $x_j$ этой матрицы переводит в вектор $Ax_j: F(x) = Ax$.

Матрица $A$ обратима $\Leftrightarrow$ обратим оператор $F$.

С каждой матрицей $A$ связан линейный оператор $F$ пространства стобцов высоты $n$, определяемый правилом $F(x) = Ax$ для любого вектора столбца $x$.

При этом $A$ будет матрицей оператора $F$. Если $F$ обратим, то его образ совпадает со всем пространством стобцов, а значит ранг $F$ равен $n$.

Если ранг $A$ равен $n$, то ранг оператора $F$ равен $n \Rightarrow$ образ $F$ совпадает со всем пространством столбцов, то есть $F$ - отображение на себя. По теореме о ранге и дефекте ядро $F$ нулевое.

Предположим, что $F(x) = F(y)$ для некоторых векторов-столбцов $x$ и $y$.

Тогда $F(x-y) = F(x) - F(y)$, откуда $x-y=0 \Rightarrow x=y \Rightarrow F$ - взаимно однозначное отображение столбцов на себя $\Rightarrow$ это обратимый оператор. Это было жоско.
\end{hproof}

\textbf{Определение.} \textbf{Рангом матрицы по строкам/столбцам} называется размерность подпространства, порождённого набором строк/столбцов матрицы $A$.

\begin{htheorem}
\textbf{Теорема.} Ранги произвольной матрицы по строкам и столбцам совпадают.
\end{htheorem}

\begin{htheorem}
\textbf{Лемма.} Умножение строки на ненулевое число и прибавление одной строки к другой не меняют ранг матрицы по строкам.
\end{htheorem}

\begin{hproof}
\textbf{Доказательство.} 

Пусть $A$ - произвольная матрица, $B$ - матрица, полученная преобразованием строки $A - a_1, ..., a_m, B - b_1, ..., b_m$.

Положим $V_A = <a_1, ..., a_m>, V_B = <b_1, ..., b_m>$. Докажем, что $dimV_A = dimV_B$ (и даже $V_A = V_B$).

\begin{enumerate}
\item $B$ получена умножением $i$-ой строки матрицы $A$ на $t \neq 0 \Rightarrow \forall j \neq i \quad b_j = a_j$ и $b_i = ta_i$.

$\forall i \quad b_i \in V_A \Rightarrow V_B \subseteq V_A$. С другой стороны $a_1, ..., a_m$ лежит в $V_B$ (кроме $a_i$, а для $a_i$ вытекает из того, что $\displaystyle a_i = \frac{1}{t} b_i$.  $V_A \subseteq V_B \subseteq V_A \Rightarrow V_A = V_B$.
\item $B$ получена прибавлением $j$-ой строки $A$ к её $i$-ой строке.

$b_k = a_k \quad \forall k \neq i$

$b_i = a_i+a_j$. Каждый из $b_1m ..., b_m$ лежит в $V_1 \Rightarrow V_B \subseteq V_A$.

Каждый $a_1, ..., a_m$, кроме $a_i$, лежит в $V_B$ ($a_i$ тоже лежит в $V_B$, так как \newline $a_i = b_i - b_j$) $\Rightarrow V_A \subseteq V_B \subseteq V_A \Rightarrow V_A=V_B$
\end{enumerate}
\end{hproof}

\newpage \begin{center}
                 \begin{Large}
                     \fbox{БИЛЕТ 38}
                 \end{Large}
    \end{center}
    
   \textbf{Определение.} Система называется \textbf{однородной}, если свободные члены всех уравнений системы нулевые:
   
$\begin{cases}
a_{11}x_1 + a_{12}x_2 + ... + a_{1n}x_n = 0,
\\
a_{21}x_1 + a_{212}x_2 + ... + a_{22n}x_n = 0,
\\
...,
\\
a_{k1}x_1 + a_{k2}x_2 + ... + a_{kn}x_n = 0

\end{cases}$.


Однородная система всегда совместна.

Система из $m$ линейных уравнений

\begin{equation}
\begin{cases}
a_{11}x_1 + a_{12}x_2 + ... + a_{1n}x_n = b_1,
\\
a_{21}x_1 + a_{22}x_2 + ... + a_{2n}x_n = b_2,
\\
...,
\\
a_{m1}x_1 + a_{m2}x_2 + ... + a_{mn}x_n = b_n

\end{cases}
\end{equation}
может быть представлена в матричном виде 

\begin{equation}
\begin{pmatrix}
a_{11}& a_{12}& ...& a_{1n}
\\
...& ...& ...& ...
\\
a_{m1}& a_{m2}& ...& a_{mn}
\end{pmatrix} \begin{pmatrix}
x_1
\\
...
\\
x_n
\end{pmatrix} =  \begin{pmatrix}
b_1
\\
...
\\
b_n
\end{pmatrix}
\end{equation}

Решение системы --- набор чисел, при подстановке которых в систему получается набор верных равенств.

Общее решение системы --- множество всех решений системы линейных уравнений.

\begin{htheorem}
\textbf{Теорема Кронекера-Капелли.} Система линейных уравнений совместна $\Leftrightarrow$ ранг матрицы системы равен рангу расширенной матрицы системы.
\end{htheorem}

\begin{hproof}
\textbf{Доказательство}

	$\Rightarrow$:
	
	Система совместна. По условию, найдётся хотя бы одно решение, то есть вектор $b$ выражается через столбцы $A \Rightarrow$ добавление к $A$ не меняет её ранга: $r(A) = r(B)$.
	
	$\Leftarrow$:
	
	В $A$ существует максимальная линейно независимая подсистема которая также является максимальной и в $B$ (к $A$ добавили вектор $b$ и получаем $B$ - ранг не увеличился $\Rightarrow$ данная подсистема осталась максимальной), причём $b$ в неё не входит $\Rightarrow b$ выражается через эту подсистему матрицы $A$, то есть выражается через $A \Rightarrow$ система совместна.
\end{hproof}

\begin{hproof}
\textbf{Более крутое доказательство.}

Обозначим расширенную матрицу как $B$.

Вектора-столбцы $A$: $a_1,.., a_n$. $V_A$ - пространство, порожденное векторами $a_1, ..., a_n$, $V_B$ - пространство, порождённое векторами из $B$.

Заметим, что система $\begin{cases}
a_{11}x_1 + a_{12}x_2 + ... + a_{1n}x_n = b_1,
\\
a_{21}x_1 + a_{22}x_2 + ... + a_{2n}x_n = b_2,
\\
...,
\\
a_{m1}x_1 + a_{m2}x_2 + ... + a_{mn}x_n = b_n

\end{cases} (*)$ может быть записана как $x_1a_1 + ... + x_na_n = b \Rightarrow$ "система совместна $\Leftrightarrow$ $b$ - линейная комбинация столбцов из $A$, то есть $b \in V_A$".

Пусть (*) совместна. Тогда $b \in V_A \Rightarrow$ вектора-столбцы из $B$ принадлежат $V_A$, то есть $V_B \subseteq V_A$.

Но столбцы $A$ являются столбцами $B \Rightarrow V_A \subseteq V_B \Rightarrow V_A = V_B \Rightarrow dimV_A = dimV_B$.

Предположим теперь, что $r(A) = r(B) = r$. Пусть базис $V_A$ состоит из $a_1, ..., a_r$. Следовательно $a_1, ..., a_r$ - базис в $V_B$. $v \in V_B$ и является комбинацией базисных векторов. Итак, $b$ - линейная комбинация $a_1, ..., a_n$, а значит и линейная комбинация всей системы векторов матрицы $A \Rightarrow$ (*) совместна.
\end{hproof}


\begin{htheorem}
\textbf{Замечание.}Если $x_0$ --- некоторое решение системы $Ax = b$, то вектор-столбец $x_1$ будет решением системы $Ax=b \Leftrightarrow x_1 = x_0 + y$, где $y$ --- решение соответствующей системы $Ax = 0$.
\end{htheorem}

\begin{hproof}
\textbf{Доказательство.} Если $x_1$ --- решение системы $Ax=b$, положим $y:=x_1-x_0$.

Тогда \begin{equation}
Ay = A(x_1-x_0) = Ax_1-Ax_0 = b-b=0
\end{equation}

Итак, $y$ --- решение системы $Ax=0$ и $x_1=x_0+y$.

Обратно, если $x_1=x_0+y$, где $y$ --- решение однородной ситемы, то \begin{equation}
Ax_1 = A(x_0+y) = Ax_0+Ay=b+0=b
\end{equation}

Отсюда $x_1$ --- решение системы $Ax = b$. Всё для тебя.
\end{hproof}



\begin{htheorem}
\textbf{Предложение.} Множество решений однородной системы $Ax=0$ образует подпространство в пространстве столбцов.
\end{htheorem}

\begin{hproof}
\textbf{Доказательство.} Если $A - k \times n$-матрицы, то правило $F(x) = Ax$ определяет линейный оператор $F: V_1 \rightarrow V_2$ из пространства столбцов высоты $n$ в пространство столбцов высоты $k$. При этом матрица $A$ будет матрицей этого оператора $F$, а множество решений $Ax=0$ будет ядром оператора $F$. Ядро линейного оператора является подпространством.

\end{hproof}

\textbf{Определение.} Если пространство решений однородной системы ненулевое, то любой базис этого пространства называется \textbf{фундаментальной системой решений}.

\begin{htheorem}
\textbf{Теорема  размерности пространства решений однородной системы.} Размерность пространства решений системы $Ax=0$ равна $n-r$, где $n$ - число неизвестных в системе, а $r$ --- ранг матрицы $A$.
\end{htheorem}

\begin{hproof}
\textbf{Доказательство.} Рассмотрим линейный оператор $F$ из пространства столбцов высоты $n$ в пространство столбцов высоты $k$, определяемый как умножение вектора-столбца на матрицу $A$ слева, и применим к $F$ теорему о ранге и дефекте. По этой теореме сумма ранга (размерности образа $F$) и дефекта (размерности ядра  $F$) равна размерности пространства столбцов высоты $n$, то есть $n$. Так как ранг линейного оператора совпадает с рангом его матрицы, ранг оператора $F$ равен $r$. Ядро оператора $F$ --- это пространство решений системы, поэтому размерность ядра равна $n-r$. Тудааа его
\end{hproof}
    



\end{document}




